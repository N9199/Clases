\documentclass[11pt]{book}
    \usepackage[spanish]{babel}
    \usepackage[utf8]{inputenc}
    \usepackage[margin=1in]{geometry}          
    \usepackage{graphicx}
    \usepackage{amsthm, amsmath, amssymb}
    \usepackage{mathtools}
    \usepackage{setspace}\onehalfspacing
    \usepackage[loose,nice]{units} 
    \usepackage{enumitem}
    \usepackage{hyperref}
    \hypersetup{
        colorlinks,
        citecolor=black,
        filecolor=black,
        linkcolor=black,
        urlcolor=black
    }
    
    \renewcommand{\d}[1]{\ensuremath{\operatorname{d}\!{#1}}}
    \renewcommand{\vec}[1]{\mathbf{#1}}
    \newcommand{\set}[1]{\mathbb{#1}}
    \newcommand{\func}[5]{#1:#2\xrightarrow[#5]{#4}#3}
    \newcommand{\contr}{\rightarrow\leftarrow}


    \newtheorem{thm}{Teorema}[section]
    \newtheorem{lem}[thm]{Lema}
    \newtheorem{prop}[thm]{Proposición}
    \newtheorem*{cor}{Corolario}

    \theoremstyle{definition}
    \newtheorem{defn}{Definición}[section]
    \newtheorem{obs}{Observación}[section]
    \newtheorem{ejm}[thm]{Ejemplo:}

\title{Introducción a la Geometría Algebraica}
\author{Nicholas Mc-Donnell}
\date{1er semestre 2019}

\pagenumbering{gobble}
\begin{document}
\maketitle
\newpage
\tableofcontents
\pagenumbering{arabic}
\newpage
\section*{Info}
\begin{itemize}
    \item[Libro:] ``Algebraic Curves'' William Fulton
    \item[Libro 2:] ``Introduction to Commutative Algebra'' Atiyah, Mac Donald
    \item[Notas:] Tareas
\end{itemize}
\chapter{Introducción}
\section{Motivación}
Estudio de objetos geométricos derivados de los polinomios (Variedades $\rightarrow$ Esquemas, etc). Los objetos son suaves o singulares.

\subsection{Resolución de singularidades para una curva}
Consideramos el siguiente polinomio:
\[
    \{(x,y)\in\set{C}^2: f(x,y):= y^2-x^2(x+1)=0\}=C
\]
%Gráfico de f(x,y) en R
\begin{defn}[Singularidad]
    Es $p\in\set{C}^2$ tal que $f(p)=0$, $f_x(p)=0$ y $f_y(p)=0$
\end{defn}
En el ejemplo el $(0,0)$ es el único punto singular.\\
Considerar el morfismo:
\begin{align*}
    \set{C}^2\xrightarrow{\makebox[1cm]{$\sigma$}}\set{C}^2 \\
    (u,v)\mapsto(uv,v)
\end{align*}
%Gráfico de la función

Vemos la pre-imagen:
\begin{align*}
    \sigma^{-1}(C) & =\{v^2-u^2v^2(uv+1)=0\}     \\
                   & =\{v^2=0\}\{1-u^2(uv+1)=0\}
\end{align*}

%Gráfico de la función sobre C

\begin{ejm}
    \begin{align*}
        \set{C}^2\xrightarrow{\makebox[1cm]{}}\set{C}^2 \\
        T(x,y)=(-x,-y)
    \end{align*}
    $T$ es automorfismo de $\set{C}^2$
    \[T\circ T=1\]%Look for Id
    Lo qye sucede es que el grupo $\{1,T\}=G$ actua en $\set{C}^2$.\\
    Mirar $\set{C}^2/G=$espacio de órbitas de $G$, lo cuál es una variedad algebraica\\
    Funciones regulares en $\set{C}^2=\set{C}[x,y]$.\\
    Queremos buscar lo siguiente:
    \[\set{C}[x,y]^G=\{f(x,y)\text{polinomio tal que }f(x,y)=f(-x,-y)\}=\set{C}[x^2,y^2,xy]\]
    \[\set{C}[x^2,y^2,xy]\simeq\set{C}[a,b,c]/(c^2-ab)\]
    \[\therefore\set{C}^2/G:=\{(a,b,c)\in\set{C}^3:c^2-ab=0\}\]
    %Graficar c^2-ab en R^3
    %Graficar cono con el punto (0,0,0) y curva similar que es el cono con un blow up en el (0,0,0)
\end{ejm}
\begin{ejm}
    \[\{(x,y)\in k^2: x^{2n}+y^{2n}=1\}=V(k)\]
    Cómo se ve $V(k)$? ($V(k)\neq\emptyset$)\\
    $n=1$
    \begin{itemize}
        \item[$k=\set{Q}$:] Circunferencia porosa ($x=\frac{t^2-1}{t^2+1}, y=\frac{2t}{t^2+1}$)(Viene de $\set{Z}$, aritmético) %Graficar
        \item[$k=\set{R}$:] Circunferencia completa (Viene de Análisis/límites) %Graficar
        \item[$k=\set{C}$:] Esfera sin puntos? %Graficar
    \end{itemize}
    $n\geq2$: $V(\set{Q})\subset V(\set{R})\subset V(\set{C})$
    \begin{itemize}
        \item[$V(\set{Q})$:] Ultimo Teorema de Fermat $\implies$ 4ptos %Graficar
        \item[$V(\set{R})$:] Algo que se acerca a un cuadrado con $n$ ``grande''%Graficar
        \item[$V(\set{C})$:] Objeto extraño con $g=(n-1)(2n-1)$ agujeros %Gráfico?
    \end{itemize}
    Variedades $=$ ceros de polinomios $\in k[x_1,...,x_n]$ donde $k=\overline{k}$
\end{ejm}
\section{Preliminares Algebraicos}
\begin{itemize}
    \item Anillos conmutativos con $1$, y morfismos de anillos, tal que el $1\mapsto1$
    \item Dominios (sin div. del cero) y cuerpos (todo $u\neq0$ es unidad)
    \item $R$ anillo $\rightarrow$ $R[x]$, grado, mónico. En general: $R[x_1,...,x_n]$
    \item Polinomios homogeneos: $F\in R[x_1,...,x_n]$ ssi $F(\lambda x_1,...,\lambda x_n)=\lambda^{\deg(F)}F(x_1,...,x_n)$
    \item $a\in R$ es \underline{irreducible} si $a$ no unidad, no cero y $a=bc\implies b$ o $c$ es unidad
    \item $a\in R$ es primeo si $a\mid bc\implies a\mid b$ o $a\mid c$
    \item $R$ es UFD (DFU): Todo elemento se factoriza de forma única salvo orden y unidades. ($R$ UFD$\implies R[x]$ UFD)
    \item Dado $R$ dominio existe $F=$ cuerpo de fracciones de $R\supset R$, $F=\{\frac{a}{b}:a,b\in R, b\neq 0\}$
    \item $f$ morfismo, $\ker f$ (ideal) $\Ima f$ (anillo)
    \item Ideal $\cong$ Kernel (Primer teorema de Isomorfismo)
    \item Para $S\subset R$ anillo, $<S>=$  Ideal generado por S
\end{itemize}
\begin{defn}[Ideal Primo]
    $p\subset R$ ideal primo ssi $ab\in p\implies a\in p\vee b\in p$
\end{defn}
\begin{thm}
    $p$ primo $\iff R/p$ dominio.
\end{thm}
\begin{proof}
    $p$ ideal primo
    \begin{align*}
             & ab=0              \\
        \iff & ab\in p           \\
        \iff & a\in p\vee b\in p \\
        \iff & a=0\vee b=0
    \end{align*}
\end{proof}
\begin{defn}[Ideal Maximal]
    $p\subset R$ es maximal ssi $p\subset m\subset R$, $m$ ideal $\implies p=m\vee m=R$
\end{defn}
\begin{thm}
    $m$ maximal $\iff R/m$  es cuerpo
\end{thm}
\begin{proof}
    $\implies$\\
    Sea $a\in R\setminus m$, por lo que $a\neq0$, luego ya que $m$ maximal, $<m, a>=R$. Dado esto, sabemos que $\exists b\in m, \exists c,d\in R: bc+ad=1$, y viendo esto en $R/m$ tenemos que $ad=1$, o sea, $a$ tiene inverso.\\
    $\impliedby$\\
    Por contradicción, existe $n$ ideal maximal que contiene a $m$
\end{proof}

\begin{prob}
    Sea $R$ un dominio.
    \begin{enumerate}
        \item Si $F,G$ son formas\footnote{Polinomios homogeneos} de grado $r,s$ respectivamente en $R[x_1,...,x_n]$, muestre que $FG$ es una forma de grado $r+s$
        \item Muestre que todo factor de una forma en $R[x_1,...,x_n]$ también es una forma
    \end{enumerate}
\end{prob}

\begin{prob}
    Sea $R$ un DFU, $K$ el cuerpo cociente de $R$. Muestre que todo elemento $z$ de $K$ se puede escribir
\end{prob}

\section{Conjuntos Algebraicos}
\begin{defn}[Espacio Afín]
    El \textbf{Espacio afín} de  dimensión $n$ es $\set{A}_k^n:=k^n$
\end{defn}
\begin{defn}[Hipersuperficie]
    Dado $F\in k[x_1,...,x_n]$, se define la \textbf{hipersuperficie}
    \[
        V(F):=\{(a_1,...,a_n)\in k^n: F(a_1,...,a_n)=0\}
    \]
\end{defn}

\begin{ejm}
    $V(y^2-x^2(x+1))\subset\set{A}_\set{R}^2$ %Graficar
\end{ejm}

\begin{ejm}
    $V(ax^2+by^2+1)\subset\set{A}_\set{R}^2=\emptyset$, dado $a,b>0$, distinto a $V(x^2+y^2+1)\subset\set{A}_\set{C}^2$
\end{ejm}

\begin{ejm}
    $V(y^2-x(x^2-1))\subset\set{A}_\set{R}^2$ %Graficar
\end{ejm}

\begin{ejm}
    $V(z^2-x^2-y^2)\subset\set{A}_\set{R}^3$ %Graficar
\end{ejm}

\begin{ejm}
    $V((x^2-y^2)(x^3-1)(y^3-1))\subset\set{A}_\set{R}^2$ %Graficar
\end{ejm}

\begin{defn}[Conjunto Algebraico]
    Sea $S\subset k[x_1,..,x_n]$. Un \textbf{conjunto algebraico afín}
    \begin{align*}
        V(S) & =\{p\in\set{A}^n_k:F(p)=0\forall F\in S\} \\
             & =\bigcap_{F\in S}V(F)
    \end{align*}
    $S=\{F_1,...,F_m\}, V(S)=V(F_1,...,F_m)$
\end{defn}
\begin{ppty}[Conjuntos Algebraicos]
    \
    \begin{enumerate}
        \item Si $I=<S>\implies V(S)=V(I)$
              \begin{proof}
                  Sea $p\in V(S)\implies F(p)=0\forall F\in S$.\\
                  Sea $G\in I\implies G=r_1F_1+...+r_mF_m$, $F_1,...,F_m\in S$ $r_1,...,r_m\in k[x_1,...,x_n]$\\
                  \[\therefore G(p)=r_1(p)F_1(p)+...+r_m(p)F_m(p)=0\implies p\in V(I)\]
                  Si $p\in V(I)\implies$ en particular $F(p)=0\forall F\in S\subset I\implies p\in V(S)$\\
                  \[\therefore V(I)=V(S)\]
              \end{proof}
        \item $\{I_\alpha\}_{\alpha\in J}$ familia de ideales $\implies V(\bigcup_{\alpha\in J}I_\alpha)=\bigcap_{\alpha\in J}V(I_\alpha)$
        \item $I\subset J$ ideales $\implies V(I)\supset V(J)$
        \item $V(FG)=V(F)\cup V(G)$ Sea $I,J$ ideales $\implies V(I)\cup V(J)=V(<\{FG:F\in I, G\in J\}>)$
        \item $V(\emptyset)=\set{A}_k^n$, $V(1)=\emptyset$
    \end{enumerate}
\end{ppty}
\begin{obs}
    La \textbf{unión} arbitraria de conjuntos algebraicos no es necesariamente conjunto algebraico:
    \[
        \set{N}=V(I)?
    \]
\end{obs}
\begin{obs}[Topología de Zariski]
    Los conjuntos algebraicos definen los \textbf{conjuntos cerrados} para una topología en $\set{A}_k^n$ ($\set{A}_k^n\setminus$cerrados $=$ abiertos). Los cerrados de esta topología son $\{\emptyset,\set{A}_k^n, \text{conj. finitos}\}$
\end{obs}
\begin{defn}[Ideal de un conjunto]
    Sea $X\subset\set{A}_k^n$. $I(X)=\{f\in k[x_1,...,x_n]: f(p)=0\forall p\in X\}$
\end{defn}
\begin{ppty}[Ideales de conjuntos]
    \
    \begin{enumerate}
        \item $I(X)$ es ideal:
              \begin{proof}
                  $f,g\in I(X)\implies f(p)+g(p)=0,\forall p\in X\implies f+g\in I(X)$\\
                  $r\in k[x_1,..,x_n], f\in I(X)\implies r(p)f(p)=r(p)\cdot 0=0\forall p\in X\implies rf\in I(X)$
              \end{proof}
        \item $X\subset Y\implies I(X)\supset I(Y)$
        \item $I(\emptyset)=k[x_1,...,x_n]$, $I(\set{A}_k^n)=(0)$ si $k$ es un cuerpo infinito. $I(\{a_1,...,a_n\})=(x_1-a_1,x_2-a_2,...,x_n-a_n)$ $a_i\in k$
        \item $I(V(S))\supset S$ $\forall$ conj. $S\subset k[x_1,..,x_n]$, $V(I(X))\supset X\forall X\subset \set{A}_k^n$
        \item $V(I(V(S)))=V(S)$ $\forall$ conj. de pol. $S$, $I(V(I(X)))=I(X)\forall X\subset\set{A}_k^n$
        \item Si $V=$ conj. alg. $\implies V=V(I(V))$, si $I=$ ideal $\implies I=I(V(I))$
    \end{enumerate}
\end{ppty}
\begin{obs}
    Si $I=I(X)$ y $\exists m\in\set{N}: F^m\in I$, entonces $F\in I$
\end{obs}

\begin{defn}[Ideal Radical]
    Si $I$ es ideal de $R$, entonces el ideal Radical es:
    \[
        \rad I=\{a\in R: \exists m\in\set{N} a^m\in I\}
    \]
\end{defn}

\section{Base de Hilbert}

\begin{thm}[Base de Hilbert]
    Todo conjunto algebraico es la intersección de un número finito de hipersuperficies.
\end{thm}

\begin{defn}[Anillo Noetheriano]
    Sea $R$ anillo. $R$ se dice \textbf{Noetheriano} ssi todo ideal de $R$ es finitamente generado.
\end{defn}

\begin{obs}
    Notar que $k$ cuerpo es Noetheriano, y que los DIP son Noetherianos.
\end{obs}

\begin{thm}[Hilbert]
    $R$ Noetheriano $\implies$ $R[x_1,...,x_n]$ Noetheriano
\end{thm}

\begin{proof}
    $F(x)=a_0+a_1x+...+a_dx^d\in R[x]$ $a_d\neq 0$, $a_d$ se llamará término líder de $F$ ($a_d=l(F)$).\\
    Sea $I\subset R[x]$ ideal, Sea $J\subset R$ el conjunto de todos los términos líderes de elementos en $I$
    \begin{obs}
        $J$ es ideal
    \end{obs}
    \noindent$\therefore$ Por hipótesis $J=<l(F_1,...,l(F_r))>$.\\
    Sea $N>\deg(F_i)\forall i=1,..,r$\\
    Para cada $m\leq N$, sea $J_m$ el conjunto de los coeficientes líderes de $F\in I$ con $\deg(F)\leq n$. Notamos que $J_m$ es ideal.\\
    $\therefore$ Por hipótesis, $J_m=<\underbrace{l(F_{m,i})}_\text{Finitos}>$ con $\deg(F_{m,i})\leq m$.\\
    \[
        I'=<F_1,...,F_r,\bigcup_{m=1}^N\{F_{m,i}\}>
    \]
    Notar que $I'\subseteq I$. Sea $G\in I\setminus I'$ tal que su grado es lo más pequeño posible.\\
    Caso 1: Si $\deg(G)>N\implies\exists$ polinomios $\{Q_i\}$ tal que $G$ y $\sum_{i=1}^rQ_i\cdot F_i$ tienen el mismo líder: Sea $l(G)=\sum_{i=1}^r\alpha_i\cdot l(F_i)$
    \[
        \therefore Q_i=\alpha_i\cdot x^{\deg(G)-\deg(F_i)}
    \]
    \[
        \therefore \deg(\underbrace{G-\sum Q_i\cdot F_i}_{\in I})<\deg(G)
    \]
    Y $G-\sum Q_i\cdot F_i\notin I'$ en otro caso $G\in I'$ $\contr$\\
    Caso 2: Si $\deg(G)\leq N\implies \deg(G)=m\leq N$\\
    $\therefore$ hacer lo mismo con $J_m$
\end{proof}

\begin{cor}
    $k[x_1,...x_m]$ es Noetheriano.
\end{cor}

\begin{defn}[Reducible/Irreducible]
    $V\subset\set{A}^n$ conjunto algebraico. Si $V=V_1\cup V_2$ donde $V_1,V_2$ son conjuntos algebraicos en $\set{A}^n$ y $V\neq V_i, i=1,2\implies$ se dice que $V$ es \textbf{reducible}. Si no, es \textbf{irreducible}
\end{defn}

\begin{ejm}
    $V(xy)\subset\set{A}^2_k$, $V(xy)=V(x)\cup V(y)\implies$ $V(xy)$  es reducible.
\end{ejm}
\begin{ejm}
    $\{p,q\}\subset\set{A}_k^n$, $V=\{p,q\}=\{p\}\cup\{q\}$
\end{ejm}
\begin{ejm}
    $V(x^2)\subset\set{A}^2$, $(x^2)=I$ no es primo
\end{ejm}

\begin{prop}
    $V$ irred. ssi $I(V)$ es primo
\end{prop}
\begin{proof}
    Si $I(V)$ no primo $\implies F_1,F_2\in k[x_1,...,x_2]$ con $F_1F_2\in I(V)$, pero $F_i\notin I(V)$ $i=1,2$
    \[
        \implies V=(V\cap V(F_1))\cup(V\cap V(F_2)), V\cap V(F_i)\nsubseteq V
    \]
    Si $p\in V\implies F_1(p)\cdot F_2(p)=0\implies F_1(p)=0\vee F_2(p)=0$.\\
    Luego $\exists q_i\in V$ tal que $F_i(q_i)\neq 0\implies q_i\notin V\cap V(F_i)$\\
    $\therefore V$ es reducible
\end{proof}

\end{document}