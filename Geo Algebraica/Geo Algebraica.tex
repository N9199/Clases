\input{../Templates/Notetaking}

\title{Introducción a la Geometría Algebraica}
\author{Nicholas Mc-Donnell}
\date{1er semestre 2019}

\pagenumbering{gobble}
\begin{document}
\maketitle
\newpage
\tableofcontents
\pagenumbering{arabic}
\newpage
\section*{Info}
\begin{itemize}
    \item[Libro:] ``Algebraic Curves'' William Fulton
    \item[Notas:] Tareas
\end{itemize}
\chapter{Introducción}
\section{Motivación}
Estudio de objetos geométricos derivados de los polinomios (Variedades $\rightarrow$ Esquemas, etc). Los objetos son suaves o singulares.

\subsection{Resolución de singularidades para una curva}
Consideramos el siguiente polinomio:
\[
    \{(x,y)\in\set{C}^2: f(x,y):= y^2-x^2(x+1)=0\}=C
\]
%Gráfico de f(x,y) en R
\begin{defn}[Singularidad]
    Es $p\in\set{C}^2$ tal que $f(p)=0$, $f_x(p)=0$ y $f_y(p)=0$
\end{defn}
En el ejemplo el $(0,0)$ es el único punto singular.\\
Considerar el morfismo:
\begin{align*}
    \set{C}^2\xrightarrow{\makebox[1cm]{$\sigma$}}\set{C}^2 \\
    (u,v)\mapsto(uv,v)
\end{align*}
%Gráfico de la función

Vemos la pre-imagen:
\begin{align*}
    \sigma^{-1}(C) & =\{v^2-u^2v^2(uv+1)=0\}     \\
                   & =\{v^2=0\}\{1-u^2(uv+1)=0\}
\end{align*}

%Gráfico de la función sobre C

\begin{ejm}
    \begin{align*}
        \set{C}^2\xrightarrow{\makebox[1cm]{}}\set{C}^2 \\
        T(x,y)=(-x,-y)
    \end{align*}
    $T$ es automorfismo de $\set{C}^2$
    \[T\circ T=1\]%Look for Id
    Lo qye sucede es que el grupo $\{1,T\}=G$ actua en $\set{C}^2$.\\
    Mirar $\set{C}^2/G=$espacio de órbitas de $G$, lo cuál es una variedad algebraica\\
    Funciones regulares en $\set{C}^2=\set{C}[x,y]$.\\
    Queremos buscar lo siguiente:
    \[\set{C}[x,y]^G=\{f(x,y)\text{polinomio tal que }f(x,y)=f(-x,-y)\}=\set{C}[x^2,y^2,xy]\]
    \[\set{C}[x^2,y^2,xy]\simeq\set{C}[a,b,c]/(c^2-ab)\]
    \[\therefore\set{C}^2/G:=\{(a,b,c)\in\set{C}^3:c^2-ab=0\}\]
    %Graficar c^2-ab en R^3
    %Graficar cono con el punto (0,0,0) y curva similar que es el cono con un blow up en el (0,0,0)
\end{ejm}
\begin{ejm}
    \[\{(x,y)\in k^2: x^{2n}+y^{2n}=1\}=V(k)\]
    Cómo se ve $V(k)$? ($V(k)\neq\emptyset$)\\
    $n=1$
    \begin{itemize}
        \item[$k=\set{Q}$:] Circunferencia porosa ($x=\frac{t^2-1}{t^2+1}, y=\frac{2t}{t^2+1}$)(Viene de $\set{Z}$, aritmético) %Graficar
        \item[$k=\set{R}$:] Circunferencia completa (Viene de Análisis/límites) %Graficar
        \item[$k=\set{C}$:] Esfera sin puntos? %Graficar
    \end{itemize}
    $n\geq2$: $V(\set{Q})\subset V(\set{R})\subset V(\set{C})$
    \begin{itemize}
        \item[$V(\set{Q})$:] Ultimo Teorema de Fermat $\implies$ 4ptos %Graficar
        \item[$V(\set{R})$:] Algo que se acerca a un cuadrado con $n$ ``grande''%Graficar
        \item[$V(\set{C})$:] Objeto extraño con $g=(n-1)(2n-1)$ agujeros %Gráfico?
    \end{itemize}
    Variedades $=$ ceros de polinomios $\in k[x_1,...,x_n]$ donde $k=\overline{k}$
\end{ejm}
\section{Preliminares Algebraicos}
\begin{itemize}
    \item Anillos conmutativos con $1$, y morfismos de anillos, tal que el $1\mapsto1$
    \item Dominios (sin div. del cero) y cuerpos (todo $u\neq0$ es unidad)
    \item $R$ anillo $\rightarrow$ $R[x]$, grado, mónico. En general: $R[x_1,...,x_n]$
    \item Polinomios homogeneos: $F\in R[x_1,...,x_n]$ ssi $F(\lambda x_1,...,\lambda x_n)=\lambda^{\deg(F)}F(x_1,...,x_n)$
    \item $a\in R$ es \underline{irreducible} si $a$ no unidad, no cero y $a=bc\implies b$ o $c$ es unidad
    \item $a\in R$ es primeo si $a\mid bc\implies a\mid b$ o $a\mid c$
    \item $R$ es UFD (DFU): Todo elemento se factoriza de forma única salvo orden y unidades. ($R$ UFD$\implies R[x]$ UFD)
    \item Dado $R$ dominio existe $F=$ cuerpo de fracciones de $R\supset R$, $F=\{\frac{a}{b}:a,b\in R, b\neq 0\}$
    \item $f$ morfismo, $\ker f$ (ideal) $\Ima f$ (anillo)
    \item Ideal $\cong$ Kernel (Primer teorema de Isomorfismo)
    \item Para $S\subset R$ anillo, $<S>=$  Ideal generado por S
\end{itemize}
\begin{defn}[Ideal Primo]
    $p\subset R$ ideal primo ssi $ab\in p\implies a\in p\vee b\in p$
\end{defn}
\begin{thm}
    $p$ primo $\iff R/p$ dominio.
\end{thm}
\begin{proof}
    $p$ ideal primo
    \begin{align*}
             & ab=0              \\
        \iff & ab\in p           \\
        \iff & a\in p\vee b\in p \\
        \iff & a=0\vee b=0
    \end{align*}
\end{proof}
\begin{defn}[Ideal Maximal]
    $p\subset R$ es maximal ssi $p\subset m\subset R$, $m$ ideal $\implies p=m\vee m=R$
\end{defn}
\begin{thm}
    $m$ maximal $\iff R/m$  es cuerpo
\end{thm}
\begin{proof}
    $\implies$\\
    Sea $a\in R\setminus m$, por lo que $a\neq0$, luego ya que $m$ maximal, $<m, a>=R$. Dado esto, sabemos que $\exists b\in m, \exists c,d\in R: bc+ad=1$, y viendo esto en $R/m$ tenemos que $ad=1$, o sea, $a$ tiene inverso.\\
    $\impliedby$\\
    Por contradicción, existe $n$ ideal maximal que contiene a $m$
\end{proof}

\begin{prob}
    Sea $R$ un dominio.
    \begin{enumerate}
        \item Si $F,G$ son formas\footnote{Polinomios homogeneos} de grado $r,s$ respectivamente en $R[x_1,...,x_n]$, muestre que $FG$ es una forma de grado $r+s$
        \item Muestre que todo factor de una forma en $R[x_1,...,x_n]$ también es una forma
    \end{enumerate}
\end{prob}

\begin{prob}
    Sea $R$ un DFU, $K$ el cuerpo cociente de $R$. Muestre que todo elemento $z$ de $K$ se puede escribir
\end{prob}

\end{document}