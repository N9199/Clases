\input{../Templates/Notetaking}

\title{Teoría de Modelos}
\author{Nicholas Mc-Donnell}
\date{Verano 2018-2019}

\pagenumbering{gobble}
\begin{document}
\maketitle
\newpage

\pagenumbering{arabic}
\tableofcontents
\newpage

\chapter{Base}
La lógica se interesa en el camino entre un conjunto de premisas y un conjunto de conclusiones. Para eso esta se fija en la forma  los argumentos.
\begin{ejm}
	\begin{align*}
		 & p\implies q          \\
		 & p                    \\
		 & \rule{1.6cm}{0.03cm} \\
		 & q
	\end{align*}
\end{ejm}
\begin{ejm}
	\begin{align*}
		 & p\implies q          \\
		 & \neg q               \\
		 & \rule{1.6cm}{0.03cm} \\
		 & \neg p
	\end{align*}
\end{ejm}
\section{Lógica proposicional}
Premisas que tienen un solo valor de verdad, Verdadero o Falso. Lenguaje es \textbf{muy} importante, no puede ser ambiguo.
\begin{defn}[Lenguaje ($\mathcal{L}$)]
	\
	\begin{itemize}
		\item letras proposicionales (infinitos numerables)
		      \[
			      p_1,p_2,...,p_n,...
		      \]
		\item conectivos lógicos
		\item paréntesis
	\end{itemize}
\end{defn}
\begin{defn}[Oración]
	\
	\begin{itemize}
		\item Oraciones atómicas:
		      \[
			      p_i
		      \]
		\item $\alpha, \beta$ son oraciones, las siguientes son oraciones:
		      \[
			      \neg\alpha, \alpha\implies\beta, \alpha\vee\beta,\alpha\wedge\beta, \alpha\iff\beta
		      \]
		\item Toda oración se obtiene de la manera anterior en un número finito de pasos.
	\end{itemize}
\end{defn}
\begin{defn}[Table de verdad]
	Es una función tal que
	\[
		v:\{p_i: i\in\set{N}\}\rightarrow\{0,1\}
	\]
	$v$ valuación: mundos posibles
\end{defn}
\begin{defn}[Consecuencia lógica ($\models$)]
	$\Gamma\models\varphi$, gama entraña phi (Gamma entails Phi), phi. Para cualquier $v$ si $v(\Gamma)=1$ entonces $v(\phi)=1$
\end{defn}
\begin{ejm}[Modus Ponens]
	\begin{align*}
		 & \text{Si }\Gamma\models\varphi\implies\psi \\
		 & \text{y  }\Gamma\models\varphi             \\
		 & \text{entonces }\Gamma\models\psi
	\end{align*}
\end{ejm}
\begin{ejm}[Teorema de Compacidad]
	Si $\Gamma\models\varphi$, entonces existe $\Gamma_0\subseteq\Gamma$ tal que $\Gamma_0$ es finito y $\Gamma_0\models\varphi$
\end{ejm}
\subsection{Sistema deductivo}
Axiomas y conectivos lógicos $(\neg,\implies)$
\begin{enumerate}
	\item $\vdash\varphi\implies(\psi\implies\varphi)$
	\item $\vdash\paren{\varphi\implies(\psi\implies\theta)}\implies\paren{(\varphi\implies\psi)\implies(\varphi\implies\theta)}$
	\item Hay dos opciones\begin{itemize}
		      \item $\vdash\paren{\neg\varphi\implies\psi}\implies\paren{(\neg\varphi\implies\neg\psi)\implies\varphi}$
		      \item $\vdash\paren{\neg\varphi\implies\neg\theta}\implies\paren{\psi\implies\theta}$
	      \end{itemize}
\end{enumerate}
Bajo Modus Ponens
\begin{proof}[$\vdash\varphi\implies\varphi$]
	\begin{align*}
		 & \vdash\varphi\implies((\varphi\implies\varphi)\implies\varphi)\implies\paren{(\varphi\implies(\varphi\implies\varphi))\implies(\varphi\implies\varphi)} \\
		 & \vdash\varphi\implies\paren{(\varphi\implies\varphi)\implies\varphi}                                                                                    \\
		 & \vdash(\varphi\implies(\varphi\implies\varphi))\implies(\varphi\implies\varphi)                                                                         \\
		 & \vdash\paren{\varphi\implies(\varphi\implies\varphi)}                                                                                                   \\
		 & \vdash\varphi\implies\varphi
	\end{align*}
\end{proof}
\begin{defn}[Demostración ($\Gamma\vdash\varphi$)]
	$\Gamma$ es consecuencia sintáctica de $\varphi$ si existe sucesión de oraciones de $\mathcal{L}$ $<\sigma_1,\sigma_2,...,\sigma_n>$
	\begin{itemize}
		\item $\sigma_n=\varphi$
		\item $\sigma_i$ es instancia de axiomas
		\item $\sigma_i\in\Gamma$
		\item $\sigma_i$ se obtiene por MP de $\sigma_j,\sigma_k;j,k<i$
	\end{itemize}
\end{defn}
\begin{thm}[Completitud]
	Si $\Gamma\models\varphi$, entonces $\Gamma\vdash\varphi$ (Dem: \cite{mendelson2009introduction})
\end{thm}

\begin{defn}[$\overline{\mathcal{L}}$]
	Son todas las oraciones de $\mathcal{L}$
\end{defn}

\begin{defn}[Álgebra totalmente libre]
	$<\mathcal{L},\vee,\wedge,\neg>$, dónde:
	\[\vee:\overline{\mathcal{L}}\times\overline{\mathcal{L}}\rightarrow\overline{\mathcal{L}}\]
	\[(\psi,\varphi)\mapsto(\psi\vee\varphi)\]
\end{defn}

\begin{defn}[Álgebra de Boole]
	Se toma la siguiente relación de congruencia
	\begin{align*}
		\varphi\sim\psi\iff & \models\varphi\iff\psi
	\end{align*}
	Con lo que $\overline{\mathcal{L}}\mid_\sim=\{[\varphi]:\varphi\in\overline{\mathcal{L}}\}$
	\[[\varphi]\vee[\psi]=[\varphi\vee\psi]\]
	\[[\varphi]\wedge[\psi]=[\varphi\wedge\psi]\]
	\[\neg[\psi]=[\neg\psi]\]
	Si se define $[\varphi]\leq[\psi]$ ssi $\models\varphi\implies\psi$, esto genera un orden en esta Álgebra.
\end{defn}
\begin{thm}[Teorema de la deducción]
	$(\Sigma\vdash\varphi\implies\psi)\impliedby(\Sigma,\varphi\vdash\psi)$
\end{thm}

\subsection{Lógica de Primer Orden}
Se cuantifica \textbf{sólo} sobre objetos, no sobre conjuntos de objetos.

\bibliographystyle{unsrt}
\bibliography{Modelos}

\end{document}