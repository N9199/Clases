\documentclass{notetaking}

\title{Ánalisis Funcional - MAT2555}
\author{Nicholas Mc-Donnell}
\date{}
\gdate{Segundo Semestre de 2019}

\setlist[enumerate]{label=\arabic*)}

\begin{document}
\maketitle
\tableofcontents
\newpage

\section*{Preliminares}

\subsection*{Contenidos}
\begin{enumerate}
    \item Espacios de Banach: Definiciones Básicas, Hahn-Banach, Consecuencias del Teorema de Bairi
    \item Espacios de Hilbert: Definiciones, Bases Hilbertianas, Proyección Dual de un Hilbert, Lax-Milgram
    \item Topologías débiles: Espacios reflexivos
    \item Teoría Espectral
\end{enumerate}

\subsection*{Textos}
\begin{itemize}
    \item Reed and Simon (Functional Analysis)
    \item Rudin (Functional Analysis)
    \item Hain Brenzin
\end{itemize}

\subsection*{Interrogaciones}
3 Interrogaciones + 1 Examen. Si hay exención sería con 6

\subsection*{Fechas}
\begin{itemize}
    \item[I1:] Semana 23-27/9
    \item[I2:] Semana 14-19/10
    \item[I3:] Semena 18-22/11
    \item[Ex:] Semana 2-6/12
\end{itemize}
\newpage

\part{Espacios de Banach}
\section{Introducción a los Espacios de Banach}
\begin{defn}[Espacio de Banach]
    Sea \(E\) un e.v., una función \(\norm{\cdot}\) tq
    \begin{itemize}
        \item \(\norm{x}\geq0\forall x\in E, \norm{x}=0\iff x=0\)
        \item \(\norm{x+y}\leq\norm{x}+\norm{y},\forall x,y,\in R\)
        \item \(\norm{\lambda x}=\abs{\lambda}\norm{x}\forall x\in E\forall\lambda\in k\)
    \end{itemize}
\end{defn}
\begin{ejm}
    En \(\set{C}^n\), si \(z\in\set{C}^n\), \(z=(z_1,\dots,z_n)\) \(\norm{z}_p=\paren{\sum_{j=1}^n\abs{z_j}^p}^{1/p}\)
\end{ejm}
\begin{ejm}
    Si \((X,\mathcal{B},\mu)\) es e. de medida y si \(1\leq p<\infty\), \(E=L^p(X)\); La norma es \(\norm{[f]}=\paren{\int_X\abs{f(x)}^p\d{x}}^{1/p}\)
\end{ejm}
\begin{obs}
    Si \(\norm{\cdot}\) es norma en \(E\), entonces \(d_E(x,y)=\norm{x-y}\) es una métrica o distancia en \(E\).
\end{obs}
\begin{defn}[Espacio de Banach]
    \(E\) e.v. con norma \(\norm{\cdot}\) se dice espacio de Banach si es completo con respecto a \(d_E\).
\end{defn}
\begin{ejm}
    Todos los anteriores son Banach
\end{ejm}
\begin{ejm}
    Sea \(\Omega\subseteq\set{R}^n\) abierto, y sea \(E=\{f:\Omega\rightarrow\set{R},\text{continúa tq }\int_\Omega\abs{f(x)}\d{x}<\infty\}\) en \(E\), \(\norm{f}_1=\int_\Omega\abs{f(x)}\d{x}<\infty\) es norma
\end{ejm}
\begin{ejm}
    Sea \(E\) un e.v. con norma, y sea \(x_n\in E\) tal que \(\sum_{k=1}^\infty\abs{x_n}<\infty\)\\
    Q: Si \(s_n=\sum_{k=1}^n x_k\), ¿qué podemos decir de \(s_n\)?\\
    Si \(1\leq m< n\) entonces \(s_n-s_m=\sum_{k=m+1}^nx_k\), luego \(\norm{s_n-s_m}\leq\sum_{k=m+1}^n\norm{x_k}\leq\sum_{k=m+1}^\infty\norm{x_k}\)\\
    De aquí no es difícil ver que, como \(\sum_{k=1}^\infty\norm{x_k}<\infty\). Entonces \(s_n\) es de Cauchy. Ciertamente \(s_n\) tiene límite en \(E\) cuando \(E\) es de Banach.
\end{ejm}
\begin{defn}[Convergencia Absoluta]
    Un \(E\) e.v. con norma, si \(x_n\in E\) es tq \(\sum_{k=1}^\infty\norm{x_k}<\infty\), diremos que la serie es absolutamente convergente
\end{defn}
\begin{defn}[Convergencia en Norma]
    Si \(s_n=\sum_{k=1}^nx_k\) es convergente en \(E\) converge respecto a \(d_E\), diremos que \(s_n\) converge en norma
\end{defn}
\begin{prp}
    Si \(E\) es Banach y \(\sum_{k=1}^\infty\norm{x_k}<\infty\), entonces \(s=\lim_{n\rightarrow\infty}s_n\) converge en norma. (Notación: \(s=\sum_{k=1}^\infty x_k\)) Recíprocamente si \(E\) e.v. con norma y si cada serie absolutamente convergente es también convergente en norma, entonces \(E\) es Banach.
\end{prp}
\begin{proof}
    \(\impliedby\): Listo anteriormente.\\
    \(\implies\): Sea \(x_n\) de Cauchy en \(E\). Claramente, basta encontrar \(x_{n_k}\) convergente.\\
    Como \(x_n\) es de Cauchy, existe \(x_{n_k}\) tq \(\norm{x_{n_k}-x_{n_{k-1}}}\leq\frac1{2^k}\) si esto es verdad.
    \[
        x_{n_k}-x_{n_1}=\sum_{j=2}^k(x_{n_j}-x_{n_{j-1}})
    \]
    Pero \(\sum_{j=2}^\infty\norm{x_{n_k}-x_{n_{k-1}}}\leq\sum_{j=2}^\infty\frac1{2^k}<\infty\) así que \(x_{n_k}-n_{n_1}\rightarrow x\implies x_{n_k}\rightarrow x+n_{n_1}\).\\
    Para ver que \(\exists x_{n_k}\) con \(\norm{x_{n_k}-x_{n_{k-1}}}\leq\frac1{2^k}\), sea \(k=1\), para \(\varepsilon=\frac12\exists n_1\) tq \(\norm{x_n-x_m}\leq\frac12\forall n,m\geq n_1\), esto da \(n_1\). Si \(1\leq n_1<\dots<n_k\) son tq \(\norm{x_{n_j}-x_{n_{j-1}}}\leq\frac1{2^j},\; j=1,\dots,k-1\), \(\norm{x_n-x_m}<\frac1{2^k}\forall n,m\geq n_k\), sea \(\varepsilon=\frac1{2^{k+1}}\). Sea \(n_{k+1}>n_k\) tq \(\norm{x_{n_{k+`'}}-x_{n_{k}}}\leq\frac1{2^{k+1}}\). Esto construye \(x_{n_k}\).
\end{proof}
\begin{ejm}
    \(M^n(\set{R})\) matrices de \(n\times n\) en \(\set{R}\), \(A\in M^n(\set{R})\) entonces \(\norm{A}=\paren{\tr(A^TA)}^{1/2}\)
\end{ejm}

\begin{defn}[Transformación Lineal]
    Sean \(E,F\) e.v. (sobre \(\set{C}\) o \(\set{R}\)). Una transformación lineal es una función \(T:E\rightarrow F\) tq \(T(x+\lambda y)=Tx+\lambda Ty\forall x,y\in E\forall\lambda\).
\end{defn}

\begin{thm}[Caracterización de continuidad de funciones lineales]
    Sean \(E,F\) e.v.n., y sea \(T:E\rightarrow F\) una transformación lineal. Las siguientes proposiciones son equivalentes:
    \begin{enumerate}
        \item \(T\) es continua en \(x\) para todo \(x\in E\).
        \item \(T\) es continua en \(0_E\)
        \item \(\sup_{\norm{x}_E=1}\norm{Tx}_F<\infty\)
        \item \(\exists c>0\forall x\in E:\norm{Tx}_F\leq c\norm{x}_E\)
    \end{enumerate}
\end{thm}

\begin{proof}
    Se demostrará \(1\implies 2\implies 3\implies 4\implies 1\)\\
    \(1\implies 2\) es trivial\\
    Para \(2\implies 3\), sea \(\varepsilon=1\), y un \(\delta>0\) tq
    \[
        \norm{x-0_E}_E<\delta\implies\norm{Tx-T(0)}_F<1
    \]
    Como \(T\) es lineal tenemos que \(Tx-T(0)=Tx\), y además se tiene que \(x-0_E=x\). Luego,
    \[
        \norm{x}_E<\delta\implies\norm{Tx}_F<1
    \]
    Ahora, para todo \(x\in E\) tq \(\norm{x}=1\), \(\norm{\delta x}=\delta\norm{x}\). Con esto,
    \[
        \norm{\frac\delta2x}=\frac\delta2<\delta
    \]
    Así, por lo anterior tenemos que
    \[
        \norm{T\paren{\frac\delta2x}}<1
    \]
    Eso significa que para todo \(x\in E\) tq \(\norm{x}=1\) se tiene que
    \[
        \norm{Tx}<\frac2\delta
    \]
    Con lo que tenemos lo pedido.\\
    Para \(3\implies 4\), sea \(c_0=\sup_{\norm{x}=1}\norm{Tx}\), entonces para todo \(x\in E\) distinto de cero, tenemos que
    \[
        \paren{\norm{\frac{x}{\norm{x}}}=1\implies\norm{T\paren{\frac{x}{\norm{x}}}}\leq c_0}\implies \norm{Tx}_F\leq c_0\norm{x}_E
    \]
    Con lo que se llega a lo que queríamos.\\
    Por último, para \(4\implies 1\), sea \(c>0\) tq \(\norm{Tx}_F\leq c\norm{x}_E\). Luego,
    \[
        \norm{Tx-Ty}_F=\norm{T(x-y)}_F\leq c\norm{x-y}_E
    \]
    Por lo que \(T\) es Lipschitz, por lo que es continua.
\end{proof}

\begin{defn}[Norma de operador/Funcional Acotado]
    Para \(E,V\) e.v.n \(T:E\rightarrow F\) que cumple 1-2-3-4 se llama funcional acotado (u operador lineal acotado); se define \(\norm{T}_{E,F}=\sup_{\norm{x}_E=1}\norm{Tx}_F=\inf\{c>0:\norm{Tx}\leq c\norm{x}\forall x\in E\}\)
\end{defn}

\begin{defn}
    Para \(E,V\) e.v.n. sea \(\mathcal{L}(E,F)=\{T:E\rightarrow F\text{ lineal, acotado}\}\)
\end{defn}

\begin{prp}
    \(\norm{\cdot}_{E,F}\) es norma en \(\mathcal{L}(E,F)\)
\end{prp}
\begin{proof}
    Claramente cumple todo en base a la definición
\end{proof}

\begin{prp}
    Si \(F\) es Banach, entonces \(\mathcal{L}(E,F)\) es Banach con  respecto a \(\norm{\cdot}_{E,F}\)
\end{prp}
\begin{proof}
    Sean \(T_n:E\rightarrow F\) lineales continuas, Cauchy con respecto a \(\norm{\cdot}_{E,F}\). Observemos que, para cada \(x\in R\)  fijo, \(y_n=T_nx\) es Cauchy en \(F\); pues \(\norm{y_n-y_m}_F=\norm{T_nx-T_mx}_F=\norm{(T_n-T_m)x}_F\leq\norm{T_n-T_m}\norm{x}\).\\
    Si \(x=0\), \(y_n\) es constante, por lo que es Cauchy.\\
    Si \(x\neq0\), sea \(\varepsilon>0\). \(T_n\) es Cauchy \(\implies\exists n_0:\norm{T_n-T_m}<\frac{\varepsilon}{\norm{x}}\forall n,m\geq n_0\). Así: \(y_n=T_nx\) es de Cauchy en \(F\), como \(F\) es completo, \(y_n\rightarrow y\equiv Tx\). En otras palabras, \(T_nx\rightarrow Tx\).\\
    Vamos a ver que \(T\in\mathcal{L}(E,F)\) y que \(\norm{T_n-T}\xrightarrow{n\rightarrow\infty}0\).\\
    \underline{Primero}: \(\forall x\in R\forall n\in\set{N}\) se tiene que \(T_n(x+\lambda y)=T_nx+\lambda T_ny\rightarrow T(x+\lambda y)=Tx+\lambda Ty\).\\
    \underline{Segundo}: (Ejercicio) Como \(T_n\) es Cauchy, \(\sup_{n\in\set{N}}\norm{T_n}=C<\infty\). Entonces \(\norm{T_nx}\leq\norm{T_n}\norm{x}\leq C\norm{x}\rightarrow\norm{Tx}\leq C\norm{x}\).\\
    \underline{Último}: Verificar que \(\norm{T_n-T}\xrightarrow{n\rightarrow\infty}0\), sea \(\varepsilon, n_0\) tq \(\norm{T_n-T_m}<\varepsilon\forall m,n\geq n_0\). Entonces \(\norm{T_n(x)-T_m(x)}\leq\varepsilon\norm{x}\;\forall n,m\geq n_0\forall x\in E\). Así \(\norm{Tx-T_nx}\leq\varepsilon\norm{x}\;\forall n\geq n_0\forall x\in E\) por lo que \(\norm{T-T_n}\leq\varepsilon\;\forall n\geq n_0\)
\end{proof}

\begin{defn}[Dual]
    Si \(E\) es e.v.n., definimos su dual (topológico) como:
    \[
        E^*=\mathcal{L}(E,\set{C})\text{ o }\mathcal{L}(E,\set{R})
    \]
\end{defn}

\begin{ejm}
    Tomemos \(E=\set{R}^n\), y sean \(S,T\in\mathcal{L}(E)\).
    \[
        \norm{S\circ T(x)}=\norm{S(Tx)}\leq\norm{S}\norm{Tx}\leq\norm{S}\norm{T}\norm{x}
    \]
    por lo que
    \[
        \norm{S\circ T}\leq\norm{T}\norm{S}
    \]
    Entonces \(\norm{T^k}\leq\norm{T}^k\)
\end{ejm}
\begin{prp}
    Si \(E\) es Banach, \(T\in\mathcal{L}(E),\norm{T}<1\), \(I-T\) es invertible, con inversa continua, entonces \((I-T)^{-1}=\sum_{k=0}^\infty T^k\)
\end{prp}
\begin{proof}
    Sale con truco típico.
\end{proof}

\begin{ejm}
    Sea \(\Omega\subseteq\set{R}^n\) abierto, y sea \(\kappa\in L^2(\Omega\times\Omega;\set{R})\). Definamos \(T_\kappa:L^2(\Omega)\rightarrow L^2(\Omega)\) donde \(f\mapsto T_\kappa(f)(x)=\int_\Omega \kappa(x,y)f(y)\d{y}\)\\
    \(T_\kappa\) es lineal. Veamos que \(T_\kappa(f)\in L^2(\Omega)\)
    \[
        \int_\Omega\abs{\int_\Omega\kappa(x,y)f(y)\d{y}}^2\d{x}\leq\int_\Omega\int_\Omega\kappa^2(x,y)\d{y}\int_\Omega f^2(y)\d{y}\d{x}
    \]
    O sea, ya que \(\int_\Omega\abs{T(f)(x)}^2\d{x}=\norm{T_\kappa f}^2_{L^2(\Omega)}\)
    \[
        \norm{T_\kappa f}\leq\norm{\kappa}\norm{f}
    \]
\end{ejm}

\begin{defn}
    Sea \(E\) e.v.n., \(p:E\rightarrow\set{R}\) se dice semi-norma si:
    \begin{itemize}
        \item \(p(x+y)\leq p(x)+p(y)\forall x,y\in E\)
        \item \(p(\lambda x)=\abs{\lambda}p(x)\forall x\in E\forall \lambda\)
    \end{itemize}
\end{defn}

\begin{ejm}
    Sea \(\Omega\subseteq\set{R}^n\), y sea \(\mathcal{C}'=\{f:\Omega\rightarrow\set{R}\text{ continuas, con derivadas continuas acotadas}\}\), \(p(f)=\int_\Omega\abs{\nabla f}\) es una semi-norma.
\end{ejm}

\subsection{Problema}
Sea \(E\) e.v.n. sobre \(\set{R}\), \(p:E\rightarrow\set{R}\) semi-norma, \(F\subseteq E\) s.e.v. y \(\varphi:F\rightarrow\set{R}\) lineal tq \(\varphi(x)\leq p(x)\forall x\in F\)

\subsubsection{Pregunta}
\(\exists\varphi_1:E\rightarrow\set{R}\) lineal tq \(\varphi_1(x)=\varphi(x)\forall x\in F\) y \(\varphi(x)\leq p(x)\forall x\in E\)?

\subsubsection{Primero}
Sea \(F\) s.e.v. de \(E\), \(F\subsetneq E\); tomemos \(x\in E\setminus F\) (en particular \(x\neq0\)), y consideramos \(F_1=\angled{\{x\}}\oplus F\). Entonces \(\forall y\in F_1\exists!\lambda\in\set{R},z\in F\) tq \(y=\lambda x+ z\). Si \(\varphi_1:F_1\rightarrow\set{R}\), entonces la linealidad implica que \(\varphi_1(y)=\lambda\varphi_1(x)+\varphi_1(z)\) y si \(\varphi_1\) es extensión de \(\varphi\), \(\varphi_1(y)=\lambda\varphi_1(x)+\varphi(z)\). En otras palabras, para extender \(\varphi\) a \(F_1\) basta escoger \(\varphi_1(x)\in\set{R}\), pero tenemos que escogerlo de modo que \(\lambda\varphi_1(x)+\varphi(z)\leq p(\lambda x+z)\,\forall z\in F\).
\begin{lem}
    Para \(E,p,F,\varphi\) como se acaban de describir.
    \[
        A=\sup_{\substack{z\in F\\\alpha>0}}\paren{\frac1\alpha\paren{-p(z-\alpha x)+\varphi(z)}}\leq\inf_{\substack{z\in F\\\alpha>0}}\paren{\frac1\alpha\paren{p(z+\alpha x)-\varphi(z)}}=B
    \]
\end{lem}
\begin{proof}
    Se tiene \(\forall\alpha,\beta>0\), \(\forall z_1,z_2\in F\)
    \begin{align*}
        \alpha\varphi(z_1)+\beta\varphi(z_2) & = (\alpha+\beta)\paren{\frac{\alpha z_1}{\alpha+\beta}+\frac{\beta z_2}{\alpha+\beta}}                      \\
                                             & = (\alpha+\beta)\paren{\frac{\alpha (z_1-\beta x)}{\alpha+\beta}+\frac{\beta (z_2+\alpha x)}{\alpha+\beta}} \\
                                             & \leq \alpha p(z_1=\beta x)+\beta p(z_2+\alpha x)
    \end{align*}
    Con esto se puede escribir
    \begin{align*}
        \alpha\varphi(z_1)-\alpha p(z_1-\beta x)       & \leq-\beta\varphi(z_2)+\beta p(z_2+\alpha x)           \\
        \frac1\beta\paren{\varphi(z_1)-p(z_1-\beta x)} & \leq\frac1\alpha\paren{-\varphi(z_2)+ p(z_2+\alpha x)} \\
        \sup_{\substack{\beta>0                                                                                 \\z_1\in F}}\frac1\beta\paren{\varphi(z_1)-p(z_1-\beta x)}&\leq\inf_{\substack{\alpha>0\\z_2\in F}}\frac1\alpha\paren{-\varphi(z_2)+ p(z_2+\alpha x)}
    \end{align*}
\end{proof}
\begin{cor}
    \(\exists\mu\in\set{R}\) tq si definimos \(\varphi_1(x)=\mu\), con \(x\in E\setminus F\), entonces \(\varphi_1(y)\leq p(y)\) \(\forall y\in F_1\)
\end{cor}
\begin{proof}
    Sea \(\mu\in[A,B]\), para \(y=\alpha x+z,\alpha>0,z\in F\),
    \begin{align*}
        \varphi_1(y) & =\varphi_1(\alpha x+z)                                                        \\
                     & =\alpha\paren{\frac1\alpha\paren{\varphi(z)-p(z+\alpha x)}+\mu}+p(\alpha x+z) \\
                     & \leq p(z+\alpha x)
    \end{align*}
    Entonces \(\varphi_1(\alpha x+z)\leq p(\alpha x+z)\, \forall \alpha>0\). Para \(-\alpha\), \(\alpha>0\) se usa una idea similar, pero con el supremo.
\end{proof}

\begin{thm}[Hahn-Banach real]
    Sea \(E\) e.v. sobre \(\set{R}\), \(F\subseteq E\) s.e.v., \(p\) semi-norma, \(\varphi:F\rightarrow\set{R}\) lineal tq \(\varphi(x)\leq p(x)\forall x\in F\). Entonces existe \(\varphi_1:E\rightarrow\set{R}\) lineal tq \(\varphi_1(x)=\varphi(x)\forall x\in F\), \(\varphi_1(x)\leq p(x)\forall x\in E\)
\end{thm}
\begin{proof}
    Sea \(E\) e.v. sobre \(\set{R}\), \(p\) semi-norma en \(E\), \(F\subseteq E\) s.e.v., si \(\varphi:F\rightarrow \set{R}\) es lineal y \(\varphi(x)\leq p(x)\forall x\in F\). Entonces, existe \(\varphi_1:E\rightarrow\set{R}\) lineal, extensión de \(\varphi\), tq \(\varphi_1(x)\leq p(x)\forall x\in E\)
\end{proof}
Definamos \(\mathcal{C}=\{(H,\psi): H\text{ s.e.v. de }E, F\subseteq H,\psi:H\rightarrow\set{R},\psi\mid_F=\varphi,\psi(x)\leq p(x) \forall x\in H\}\). Para \((H_1,\psi_1),(H_2,\psi_2)\in \mathcal{C}\). Se define \((H_1,\psi_1)\leq(H_2,\psi_2)\) ssi \(H_1\subseteq H_2\) y \(\psi_2\mid_{H_1}=\psi_1\). \(\leq\) es un orden parcial sobre \(\mathcal{C}\).
\begin{enumerate}
    \item \(\mathcal{C}\neq\emptyset\), pues \((F,\varphi)\in\mathcal{C}\).
    \item Si  \(\{(H_i,\psi_i)\}_{i\in I}\subseteq\mathcal{C}\) es totalmente ordenado, \(\exists (H_S,\psi_s)\) cota superior de \(\{(H_i,\psi_i)\}_{i\in I}\), o sea \(H_s\supseteq H_i\forall i\in I, \psi_s\mid_{H_i}=\psi_i\).
\end{enumerate}
Para esto:\\
Sea \(H_s=\bigcup_{i\in I}H_i\). \(H_s\) es s.e.v. de \(E\) pues si \(x,y\in H_s,\lambda\in\set{R}\) entonces \(\exists i,j\in I\) tq \(x\in H_i,y\in H_j\), como \(\{(H_i,\psi_i)\}\) es totalmente ordenado, se tiene que \(H_i\subseteq H_j\) o \(H_j\subseteq H_i\). Ahora, s.p.d.g. \(H_i\subseteq H_j\implies x,y\in H_j\implies x+\lambda y\in H_j\subseteq H_s\).\\
Definamos \(\psi_s:H_s\rightarrow\set{R}\) de la siguiente forma:
\[
    \psi_s(x)=\psi_i(x)\text{ si }x\in H_i
\]
Esto esta bien definido, pues \(\{(H_i,\psi_i)\}\) es totalmente ordenado. Además \(\psi(x)\leq p(x)\forall x\in H_s\) pues todo \(\psi_i\) satisface esto.\\
Esto prueba que \((H_s,\psi_s)\in\mathcal{C}\) y que \((H_i,\psi_i)\leq (H_s,\psi_s)\forall i\in I\). Ahora, por el lemma de Zorn existe \((H_*,\psi_*)\in\mathcal{C}\) que es maximal. Esto es \(\paren{(H,\psi)\in\mathcal{C}\wedge (H_*,\psi_*)\leq (H,\psi)}\implies (H_*,\psi_*)=(H,\psi)\).\\
\underline{Afirmación:} \(H_*=E\). Si no: \(\exists x\in E\setminus H_*\), por lo que podemos extender \(H_*\) y \(\psi_*\), lo que es una contradicción, ya que \((H_*,\psi_*)\) es maximal. Con esto se tiene que
\begin{cor}
    Sea \(E\) e.v.n. (sobre \(\set{R}\) por ahora), \(\forall x\in E, x\neq0,\exists \varphi\in E^*\) tq \(\varphi(x)=\norm{x}\) y \(\norm{\varphi}=1\)
\end{cor}
\begin{proof}
    Definamos \(\varphi(\lambda x)=\lambda\norm{x}\), \(\varphi:\angled{x}\rightarrow\set{R}\) es lineal y claramente \(\varphi(\lambda x)\leq\norm{\lambda x}\). Ahora usamos HB con \(p(x)=\norm{x}\).
\end{proof}
\begin{thm}[Hahn-Banach complejo]
    Hahn-Banach, pero sobre \(\set{C}\).
\end{thm}
\begin{proof}
    La misma demostración, pero cambiar la frase \(\forall x \varphi(x)\leq p(x)\) por \(\forall x\abs{\varphi(x)}\leq p(x)\) donde \(\abs{\cdot}\) es la norma en \(\set{C}\).
\end{proof}

\begin{cor}
    Si \(F\subseteq E\), \(\overline{F}\neq E\) \(\forall x\in E\setminus \overline{F}\,\exists\varphi\in E^*\) tq \(\varphi(z)=0\forall z\in F\) pero \(\varphi(x)\neq 0\)
\end{cor}
\begin{proof}
    Para todo \(A\subseteq E\) y \(x\in E\) se define:
    \[
        d(x,A)=\inf\{\norm{x-y}:y\in A\}
    \]
    Sea \(x\in\overline{F}\), y sea \(F_1=\angled{x}\oplus F\). Se toma \(y\in F_1\), \(y=\lambda x+z\), \(z\in F\). Se define \(\varphi(y)=\lambda d(x,F)\) para \(z\in F\) y \(\lambda\neq 0\). Luego,
    \begin{align*}
        \abs{\varphi(\lambda x+z)} & =\abs{\lambda}d(x,F)                      \\
                                   & \leq\abs{\lambda}\norm{x+\frac1\lambda z}
    \end{align*}
    Con lo que \(\abs{\varphi(\lambda x+z)}\leq\norm{\lambda x+z}\). O sea, \(\abs{\varphi(y)}\leq\norm{y}\) para \(y\in F_1\) y \(\varphi(z)=0\forall z\in F\)\\
    Luego por HB \(\exists\varphi':E\rightarrow\set{C}\), \(\norm{\varphi'(y)}\leq\norm{y}\forall y \in E\) y además \(\varphi'(x)=d(x,F)>0\) para \(x\notin\overline{F}\).
\end{proof}
\begin{obs}
    Esta proposición dice que si \(x\notin\overline{F}\) entonces \(\exists\varphi\in E^*\) tq \(\Re(\varphi(x))>\Re(\varphi(y))\) para todo \(y\in F\).
\end{obs}
En ese espíritu uno puede probar que:\\
Si \(E\) es e.v.n., \(A,B\) convexos cerrados y disjuntos, con \(B\) compacto, \(\exists\varphi\in E^*\), y \(\alpha,\beta\in\set{R}\) con \(\alpha<\beta\) tq \(\Re(\varphi(x))\leq\alpha<\beta\leq\Re(\varphi(y))\,\forall x\in A\,\forall y\in B\).

\section{Consecuencias del Lema de Baire}
\subsection{Recuerdo}
Sea \(X\) e.m. completo. Si \(O_n\subseteq X\) es abierto denso en \(X\forall n\in\set{N}\), entonces su intersección \(\bigcap_{n\in\set{N}}O_n\) es densa.\\
Esto es equivalente a:\\
Si \(F_n\) es cerrado de interior vacío en \(X\) para todo \(n\in\set{N}\). Entonces \(\bigcup_{n\in\set{N}}F_N\) tiene interior vacío.

\begin{thm}[Banach Steinhaus o Teorema de la cota uniforme]
    Sean \(E,F\) Banach, y sean \(T_i:E\rightarrow F\) lineales indexadas por \(i\in I\). Supongamos que \(\forall x\in E\,\exists c_x\geq 0\) tq \(\norm{T_ix}_F\leq c_x\,\forall i\in I\).\\
    En estas condiciones \(\exists c\geq 0\) tq
    \begin{align*}
         & \norm{T_ix}_F\leq C\norm{x}_E & \forall x\in E,\,\forall i\in I
    \end{align*}
    Equivalentemente \(\norm{T_i}_{\mathcal{L}(E,F)}\leq C\,\forall i\in I\)
\end{thm}
\begin{proof}
    Por hipótesis, \(\forall x\in E\) \(\exists c_x\geq 0\) tq \(\norm{T_ix}\leq c_x\) \(\forall i\in I\). Esto dice que si definimos
    \[
        B_n=\{x\in E\text{ tq }\norm{T_ix}\leq n\forall i\in I\}
    \]
    Entonces \(\bigcup_{n\in\set{N}}B_n=E\). Además, \(B_n\) son cerrados.\\
    Por el lema de Baire, algún \(B_n\) tiene interior no vacío, digamos \(B_{n_0}\). Entonces \(\exists x_0\in E\) y \(\delta>0\) tq \(B_{2\delta}(x_0)\subseteq B_{n_0}\).\\
    Entonces \(\forall x\in E\), \(x\neq0\),
    \begin{align*}
        T_i(x) & =\frac{\norm{x}}\delta T_i\paren{x_0+\frac{\delta x}{\norm{x}}-x_0}
    \end{align*}
    Como \(x_0+\frac{\delta x}{\norm{x}}\in B_{2\delta}(x_0)\) se tiene que
    \begin{align*}
        \norm{T_ix} & =\frac{\norm{x}}\delta\norm{T_i(x_0+\frac{\delta x}{\norm{x}})-T_ix_0}                         \\
                    & \leq\frac{\norm{x}}\delta\paren{\norm{T_i\paren{x_0+\frac{\delta x}{\norm{x}}}}+\norm{T_ix_0}} \\
                    & \leq\frac{2n_0}\delta\norm{x}
    \end{align*}
    Esto es, \(\norm{T_ix}\leq\frac{2n_0}\delta\norm{x}\) \(\forall x\in E,\,\forall i\in I\).
\end{proof}
\begin{thm}[Teorema de la aplicación abierta]
    Sean \(E,F\) Banach, sea \(T:E\rightarrow F\) lineal continua y sobreyectiva. Entonces \(T\) es abierto, esto es, \(T(O)\) es abierto en \(F\) para todo \(O\) abierto en \(E\).
\end{thm}
\begin{proof}
    Sea \(T:E\rightarrow F\) lineal, continua y sobreyectiva. Con \(E,F\) espacios de Banach. Basta probar que \(T(B_t(x))\) es abierto en F \(\forall x\in E,\forall t>0\). Ya que dado \(V\) abierto, \(V=\bigcup_{x\in V}B_{\delta(x)}(x)\) para \(\delta(x)>0\), y entonces:
    \begin{equation*}
        \bigcup_{x\in V}T(B_{\delta(x)}(x))=T(V)
    \end{equation*}
    Lo que sería un abierto. Ahora, se nota que \(T(B_t(x))=T(x)+T(B_t(0))\). Esto nos dice que basta ver que \(T(B_t(0))\) sea abierto \(\forall t>0\).\\
    Además, se ve que \(h_t(T(B_1(0)))=T(B_t(0))\), por lo que basta que \(T(B_t(0))\) sea abierto para algún \(t>0\).\\
    Para eso, se demostrará que es suficiente que \(T(B_t(0))\) tenga interior no vacío para que sea abierto. Si \(T(B_t(0))\) tiene interior no vacío, entonces \(\exists\delta>0\), \(y\in F\) tal que \(D_{2\delta}^F(y)\subseteq T(B_t^E(0))\). De aquí, \(\exists x\in B_t(0)\) tal que \(T(x)=y\) y \(\norm{x}<t\). Con esto:
    \begin{align*}
        B_{2\delta}^F(y)=T(x)+B_{2\delta}^F(0) \subseteq T(B_t^E(0)) \\
        \implies B_{2\delta}^F(0) \subseteq T(-x + B_t^E(0)) \subseteq T(B_{2t}^E(0))\footnotemark
    \end{align*}\footnotetext{Esta última contención viene dada porque \(\norm{-x+z}\leq\norm{x}+\norm{z}\leq2t\).}
    Es decir, \(B_{2\delta}^F(0)\subseteq T(B_{2t}^E(0))\), lo que gracias a la dilatación \(h_{\frac\varepsilon{2\delta}}\) es equivalente a decir que \(\forall\varepsilon>0 B_\varepsilon^F(0)\subseteq T(B_{\frac{t\varepsilon}\delta}(0))\).\\
    Se usará lo anterior para probar que \(T(B_t(0))\) es abierto. Sea \(v\in T(B_t(0))\). Digamos que \(T(u)=v\), con \(u\in B_t(0)\).
    \begin{obs}
        \(B_{t-\norm{u}}^E(u)\subseteq B_t^E(0)\), así que \(u+B_{t-\norm{u}}^E(0)\subseteq B_t^E(0)\). Gracias a esto concluimos que \(T(u)+T(B_{t-\norm{u}}^E(0))\subseteq T(B_t^E(0))\).
    \end{obs}
    Ahora, escogiendo \(\frac{t\varepsilon}\delta\leq t-\norm{u}\), concluimos que \(B_\varepsilon^E(T(u))=T(u)+B_\varepsilon^F(0)\subseteq T(B_t^E(0))\).
\end{proof}
\begin{cor}
    Si \(E,F\) Banach, y \(T:E\rightarrow F\) es lineal continua y biyectiva, entonces es bicontinua.
\end{cor}

\begin{thm}
    Sea \(H\) Hilbert
    \begin{enumerate}
        \item Si \(\{e_i\}_{i\in I},\{f_j\}_{j\in J}\) son ortonormales completos en \(H\). Entonces \(\exists h: I\rightarrow J\) biyectiva.
        \item \(H\) tiene al menos un conjunto ortonormal completo.
    \end{enumerate}
\end{thm}
\begin{proof}
    \
    \begin{enumerate}
        \item Sean \(\{e_i\}_{i\in I},\{f_j\}_{j\in J}\) completos en \(H\). Vamos a probar que \(\exists z: I\rightarrow J\) inyectiva. Luego, similarmente existiría \(\eta: J\rightarrow I\), por lo que por Cantor-Bernstein se tiene que \(\exists h: I\rightarrow J\) biyectiva.\\
              Para construir \(z\), observemos que \(\forall j\in J\), \(I_j\{i\in I:\angled{e_i,f_j}\neq0\}\) es numerable. Además, \(I=\bigcup_{j\in J}I_j\) pues si \(i\notin I_j\forall j\in J\), entonces \(\angled{e_i,f_j}=0\forall j\in J\implies e_i=0\).\\
              Definamos: para \(i\in I\) escojamos un \(j\in J\) tq \(i\in I_j\). Enuramos \(I_j=\{\alpha,\alpha_2,\alpha_3,\dots\}\) y definamos \(z: I\rightarrow I\times\set{N}\) donde \(i\mapsto (j,n)\) donde \(i\in I_j\) y \(i=\alpha_n\).\\
              Afirmación: \(z\) es inyectivo. Para ver esto, sean \(i_1,i_2\in I\) distintos:
              \begin{itemize}
                  \item[Caso 1]: \(j_1=j_2=j\). Entonces \(i_1,i_2\in I_j=\{\alpha,\alpha_1,\alpha_3,\dots\}\), luego \((j,n_1)\) \(j,n_2\) son distintos ya que \(i_1\neq i_2\). Así \(z(i_1)\neq z(i_2)\)
                  \item[Caso 2]: \(j_1\neq j_2\), entonces \(z(i_1)\neq z(i_2)\)
              \end{itemize}
              Entonces, \(z\) es inyectiva, y como \(\exists\eta:J\times\set{N}\rightarrow J\) entonces \(\eta\circ z:I\rightarrow J\) es inyectiva.
        \item Sea \(\mathcal{C}=\{\{e_i\}_{i\in I}\subseteq H: \{e_i\}_{i\in I}\text{ ortonormal}\}\). \(\mathcal{C}\) es no vacía, ya que si \(x\neq 0,e_x=\frac{x}{\norm{x}}\) satisface lo pedido. Se toma el orden parcial dado por la inclusión en \(\mathcal{C}\). Ahora, si \(\mathcal{B}_i\in\mathcal{C}\forall i\in \Lambda\) y si \(\{\mathcal{B_i}_{i\in\Lambda}\}\) es totalmente ordenado por \(\subseteq\). Se define \(\mathcal{B}=\bigcup_{i\in\Lambda}\mathcal{B}_i\). Claramente \(\mathcal{B}\) es cota superior en \(\{\mathcal{B}_i\}_{i\in\Lambda}\). Ahora por Zorn, se tiene un elemento maximal \(\mathcal{B}^*\) en \(\mathcal{C}\). Solo falta que \(\mathcal{B}^*\) sea completo, si \(\exists x\in H\) tq \(\angled{x,e}=0\forall e\in\mathcal{B}^*\), con \(x\neq 0\) entonces \(\mathcal{B}^*\cup\{\frac{x}{\norm{x}}\}\) es ortonormal, pero esto contradice la maximalidad de \(\mathcal{B}^*\).
    \end{enumerate}
\end{proof}

\begin{thm}
    Sea \(E\) Banach, \(E\) reflexiva ssi \(\overline{B}_1=\{x\in E:\norm{x}\leq1\}\) es compacto en \(\sigma(E,E^*)\)
\end{thm}
\begin{proof}
    \(\implies\): Hecho en clase anterior%Anotar
    \\
    \(\impliedby\): Se toma un desvio por Hahn-Banach Geométrico
\end{proof}

\begin{thm}[Hahn-Banach Geométrico]
    Sea \(E\) Banach
    \begin{enumerate}
        \item Si \(U,V\) son abiertos convexos, \(U\cap V=\emptyset\), entonces \(\exists \varphi\in E^*\) tq \(\Re(\varphi(x))<\Re(\varphi(y))\) \(\forall x\in U\forall y\in V\).
        \item Si \(C\) y \(K\) cerrados convexos en \(E\), \(K\) compacto y \(K\cap C=\emptyset\), entonces \(\exists \varphi\in E^*,\delta>0\) tq \(\Re(\varphi(x))+\delta\leq\Re(\varphi(y))-\delta\) \(\forall x\in C\forall y\in K\)
    \end{enumerate}
\end{thm}
Para demostrar lo anterior, se usarán los siguientes lemas:
\begin{lem}
    Sea \(V\subseteq E\), \(V\) abierto, convexo, \(0\in V\). Existe una función \(P_v:E\rightarrow[0,\infty)\) tq
    \begin{enumerate}
        \item \(P_V(\lambda x)=\lambda P_V(x)\) \(\forall x\in E\forall\lambda>0\)
        \item \(P_V(x+y)\leq P_V(x)+P_V(y)\) \(\forall x,y\in E\)
        \item \(V=\{x\in E:P_V(x)<1\}\)
    \end{enumerate}
\end{lem}
\begin{proof}
    Definamos \(P_V(x)=\inf\{t>0:\frac1tx\in V\}\), verifiquemos \(\forall x\in E\) el conjunto \(\{t>0:\frac1tx\in V\}\neq\emptyset\). Esto garantiza que el \(\inf\) existe.\\
    \underline{Para esto}: \(0\in V\implies\exists\delta>0\) tq \(B_\delta(0)\subseteq V\). Entonces \(\forall x\in E\) se tiene que \(\frac\delta{2\norm{x}}x\in B_\delta(0)\subseteq V\). En particular \(t=\frac{2\norm{x}}\delta\in\{t>0:\frac1tx\in V\}\)\\
    Sea ahora \(x\in E,\lambda>0\).
    \begin{align*}
        P_V(\lambda x)&=\inf\{t>0:\frac1t(\lambda x)\in V\}\\
        &=\lambda\cdot\inf\{(t/\lambda)>0:\frac1{t//\lambda}x\in V\}\\
        &=\lambda P_V(x)
    \end{align*}
    Para 2), es claro que \(\forall x\in E\forall\varepsilon>0\frac1{P_V(x)+\varepsilon}x\in V\). Sean \(x,y\in E,\varepsilon>0\). Definamos \(\alpha=\frac{P_V(x)+\varepsilon}{P_v(x)+P_V(y)+2\varepsilon}\in[0,1]\), \(1=\alpha=\frac{P_V(y)+\varepsilon}{P_v(x)+P_V(y)+2\varepsilon}\in[0,1]\).\\
    Entonces
    \begin{align*}
        \alpha\cdot\paren{\frac{x}{P_V(x)+\varepsilon}}+(1-\alpha)\paren{\frac{y}{P_V(y)+\varepsilon}}=\frac{x+y}{P_v(x)+P_V(y)+2\varepsilon}\in V
    \end{align*}
    Por lo que \(P_V(x+y)\leq P_V(x)+P_V(y)+2\varepsilon\) \(\forall \varepsilon>0\).\\
    Solo falta 3), llamemos \(V_1=\{x\in E:P_V(x)<1\}\). Para ver que \(V)1\subseteq V\), tomemos \(x\in V_1\), o sea, \(P_V(x)<1\). Escojamos \(t\) tq \(P_V(x)<t<1\). Entonces
    \begin{equation*}
        \frac1tx\in V\implies t\cdot\paren{\frac1tx}+(1-t)\cdot 0=x\in V
    \end{equation*}
    Conclusión \(V_1\subseteq V\). Para la otra parte, sea \(x\in V\). Si \(x=0\) entonces \(P_V(x)=0<1\). Si \(x\in V,x\neq 0\) \(\exists r>0\) tq \(B_r(x)\subseteq V\). Observemos entonces que \(\paren{1+\frac{r}{2\norm{x}}}x\in B_r(x)\). En otras palabras \(t=\frac1{1+\frac{r}{2\norm{x}}}<1\) cumple que \(\frac1tx\in V\implies P_V(x)<1\). Con lo que se tiene lo pedido.
\end{proof}

\begin{lem}
    Sea \(E\) Banach, \(V\subseteq E\), \(V\) abierto, convexo, \(V\neq\emptyset\), y escojamos \(x_0\notin V\). Entonces \(\exists \varphi\in E^*\) tq \(\Re(\varphi(x))<\Re(\varphi(x_0))\) \(\forall x\in V\)
\end{lem}

\begin{proof}
    Tomemos \(y_0\in V\), y definamos \(V_1=\{x-y_0:x\in V\}\), \(x_1=x_0-y_0\), con esto \(0\in V_1\), \(x_1\notin V_1\). Definamos \(F=\{\lambda x_1:\lambda\in\set{R}\}\). \(F\) es s.e.v. real de \(E\). Definamos \(\varphi_1:F\rightarrow\set{R}\) donde \(\varphi_1(\lambda x_1)=\lambda\). Observemos que \(\forall\lambda>0:\) \(\varphi_1(\lambda x_1)=\lambda\leq\lambda P_{V_1}(x_1)=P_{V_1}(\lambda x_1)\).\\
    Esto dice que \(\varphi_1\leq P_V(x)\) \(\forall x\in F\). Ahora, por HB, \(\varphi_1\) se extiende a \(\varphi_1:E\rightarrow\set{R}\) lineal real tq \(\varphi_1(x)\leq P_V(x)\) \(\forall x\in E\).
    \begin{obs}
        \(\varphi_1(x)\leq P_V(x)\leq \frac{2\norm{x}}\delta\implies\varphi_1\) es lineal (real) continua.
    \end{obs}
    Sea \(\varphi(x)=\varphi_1(x)-i\varphi_1(ix)\). Claramente \(\Re(\varphi)=\varphi_1\), \(\varphi\in E^*\) y además con \(x\in V_1\)
    \begin{align*}
        \Re(\varphi(x))\leq P_{V_1}(x)<1=\varphi_1(x_1)=\Re(\varphi(x_1))
    \end{align*}
    Trasladando \(V_1\) de vuelta se tiene lo pedido.
\end{proof}
\end{document}