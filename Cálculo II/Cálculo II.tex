\documentclass[11pt]{book}
    \usepackage[spanish]{babel}
    \usepackage[utf8]{inputenc}
    \usepackage[margin=1in]{geometry}          
    \usepackage{graphicx}
    \usepackage{amsthm, amsmath, amssymb}
    \usepackage{mathtools}
    \usepackage{setspace}\onehalfspacing
    \usepackage[loose,nice]{units} 
    \usepackage{enumitem}
    \usepackage{hyperref}
    \hypersetup{
        colorlinks,
        citecolor=black,
        filecolor=black,
        linkcolor=black,
        urlcolor=black
    }
    
    \renewcommand{\d}[1]{\ensuremath{\operatorname{d}\!{#1}}}
    \renewcommand{\vec}[1]{\mathbf{#1}}
    \newcommand{\set}[1]{\mathbb{#1}}
    \newcommand{\func}[5]{#1:#2\xrightarrow[#5]{#4}#3}
    \newcommand{\contr}{\rightarrow\leftarrow}


    \newtheorem{thm}{Teorema}[section]
    \newtheorem{lem}[thm]{Lema}
    \newtheorem{prop}[thm]{Proposición}
    \newtheorem*{cor}{Corolario}

    \theoremstyle{definition}
    \newtheorem{defn}{Definición}[section]
    \newtheorem{obs}{Observación}[section]
    \newtheorem{ejm}[thm]{Ejemplo:}

\title{Cálculo II}
\author{Nicholas Mc-Donnell}
\date{1er semestre 2018}

\pagenumbering{gobble}

\begin{document}
    \maketitle

    \newpage
    \hfill
    \newpage
    \section*{Programa}
    {\raggedleft Profesor: Godofredo Iommi}\\
    Email: \url{giommi@mat.uc.cl}
    \begin{enumerate}
        \item La Integral de Riemann

        \item Técnicas de integración

        \item Aplicaciones

        \item Integrales impropias

        \item Sucesiones y Series de funciones
    \end{enumerate}

    \section*{Bibliografía}
    \begin{itemize}
        \item Calculus, 4 edición, Kitchens

        \item \url{www.mat.uc.cl/~igiommi}

    \end{itemize}

    \section*{Adicional}
    \begin{itemize}
        \item Análise Real Vol. I, Lima

        \item Introduction to Calculus and Analysis, Courant y John
    \end{itemize}

    \section*{Evaluaciones}
    %\begin{justify}
        
    %\end{justify}
    {\raggedright I1: Jueves 5 Abril 7-8}\\
    I2: Jueves 3 Mayo 7-8\\
    I3: Miércoles 6 Junio 7-8\\
    Examen: Martes 26 Junio 3-4\\

    \[
        NF = 0.7\cdot\frac{(I1+I2+I3)}{3}+0.3\cdot EX
    \]
    No hay eximición 
    \tableofcontents
    \pagenumbering{arabic}

    \part{La Integral de Riemann}
    \chapter{Axioma del Supremo}
    {\raggedleft Estructura algebraica: $\set{R}$ es un cuerpo}\\
    Estructura de orden: $\set{R}$ esta ordenado\\
    $\set{R}$ es completo ($\set{Q}$ no es completo)\\
    \begin{defn}[Cota Superior]
        Sea $A\subseteq\set{R}$ un conjunto no vacío, diremos que $a\in\set{R}$ es \underline{cota superior} de $A$ si para todos $x\in A$ se tiene que $x\leq a$
    \end{defn}
    \begin{defn}[Cota Inferior]
        Análogamente se define cota inferior
    \end{defn}
    \begin{obs}
        Las cotas superiores e inferiores \underline{no} son unicas.
    \end{obs}
    \begin{ejm}
        \[A=\left\{\frac{1}{n}:n\in\set{N}\right\}\]
        es tal que $0$ es cota inferior y $2$ es cota superior.
    \end{ejm}
    \begin{ejm}
        \[A=[0,+\infty]\]
        Posee cota inferior, pero no superior
    \end{ejm}
    \begin{defn}
        Diremos que $A\subset\set{R},A\neq\emptyset$ es \underline{acotado superiormente} (resp. inferiormente) si posee cotas superiores (resp. inferiores).\\
        Diremos que $A$ es \underline{acotado} si lo es inferior y superiormente.
    \end{defn}
    \begin{defn}
        Sea $A\subset\set{R},A\neq\emptyset$ si $a\in A$ es cota superior (resp. inferior) de $A$ diremos que "$a$" es el \underline{maximo} (resp. minimo) de $A$
    \end{defn}
    \begin{defn}[Supremo]
        Sea $A\subset\set{R}$ un conjunto no vacio, diremos que $a\in\set{R}$ es el supremo de $A$ y anotaremos $a=\sup A$ si satisface:
        \begin{enumerate}
            \item El numero $a$ es cota superior de $A$

            \item Si $b\in\set{R}$ es cota superior de $A$ entonces $a\leq b$
        \end{enumerate}
    \end{defn}

    \begin{obs}
        El supremo es la menor de las cotas superiores de $A$
    \end{obs}

    \begin{obs}
        Es posible reformular la definición de supremo. En efecto, $a=\sup A$ si:
        \begin{enumerate}
            \item $a$ es cota superior de $A$

            \item $\forall\epsilon>0\exists x\in A:a-\epsilon<x\leq a$

        \end{enumerate}
    \end{obs}

    \begin{defn}[Ínfimo]
        Análogamente el ínfimo se define con cotas inferiores y con notación $\inf$
    \end{defn}
    \begin{defn}[Axioma del Supremo]
        Todo conjunto $A\subset\set{R}$ acotado superiormente posee supremo.
    \end{defn}
    \begin{obs}
        Todo conjunto $A\subset\set{R}$ acotado inferiormente posee ínfimo.
    \end{obs}

    \begin{ejm}
        Sea $A=(a,b)$, demuestre que $\inf A=a$
    \end{ejm}
    \begin{proof}
        De la defincion de intervalo tenemos que "$a$" es cota inferior.\\
        Si $\epsilon>b-a$ entonces para todo $x\in A$ se tiene que $x<a+\epsilon$\\
        Sea $0<\epsilon<b-a$ y consideremos el numero $c=a+\frac{\epsilon}{2}$\\
        Entonces:
        \begin{enumerate}
            \item $a<c$

            \item $a+\frac{\epsilon}{2}\leq a+\epsilon<a+b-a=b$
        \end{enumerate}
        \[\implies c\in A\]
        Luego $\inf A=a$
    \end{proof}
    \begin{prop}[Arquimediana]
        Dado un numero real $x$, existe $n\in\set{N}$ tal que $x<n$
    \end{prop}
    \begin{proof}
        La afirmación es equivalente a decir que el conjunto $\set{N}$ \underline{no} esta acotado superiormente.\\
        Supongamos por el contrario que $\set{N}$ es acotado superiormente. Por el axioma del supremo existe $c=\sup\set{N}$\\
        En particular $c-1$ no es cota superior de $\set{N}$. Es decir, existe $n\in\set{N}$ tal que $c-1<n$. Luego, $\sup\set{N}=c<n+1$ como $n+1\in\set{N}$, tenemos la contradicción que prueba el resultado.
    \end{proof}

    \begin{ejm}
        Pruebe que el infimo de $A=\{\frac{1}{n}:n\in\set{N}\}$ es igual a $0$.
    \end{ejm}
    \begin{proof}
        Cero es cota inferior ya que los elementos de $A$ son cuocientes de numeros positivos y por lo tanto son positivos.\\
        Supongamos que $a>0$ es tal que $\inf A=a$. Es decir, para todo $n\in\set{N}$ se tiene que
        \[a\leq\frac{1}{n}\]
        En particular, para todo $n\in\set{N}$ se tiene que $n\leq\frac{1}{a}$ que contradice la proposicion Arquimediana. Luego, $\inf A=0$
    \end{proof}

    \begin{ejm}
        Demuestre que $\inf\{\frac{|\sin(n)|}{n}:n\in\set{N}\}$ es igual a $0$.
    \end{ejm}

    \begin{proof}
        Notemos que $|\sin(n)|\geq 0$ y $n\geq0$. Luego $x=0$ es cota inferior de $\{\frac{|\sin(n)|}{n}:n\in\set{N}\}$.\\
        Notemos ademas que $|\sin(n)|\leq 1$. Luego, $\frac{|\sin(n)|}{n}\leq\frac{1}{n}$
    \end{proof}


\end{document}