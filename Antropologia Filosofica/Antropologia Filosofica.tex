\documentclass[11pt]{book}
    \usepackage{notetaking}

\title{Antropologia Filosofica}
\author{Nicholas Mc-Donnell}
\date{1er semestre 2018}

\pagenumbering{gobble}

\begin{document}
    \maketitle

    \section*{Objetivo General}
    La idea de estas discusiones es generar un debate en torno a los temas propuestos durante la primera sección de cada modulo. Esto permitirá generar un espacio donde cada alumno o grupo podrá desarrollar su punto de vista sobre las cuestiones propuestas y defenderlo por medio de argumentos y conceptos propios. Todo ello con el propósito de familiarizar al alumno con la actividad filosófica.

    \section*{Evaluaciones}
    {\raggedleft3 Controles de lectura después de cada unidad}\\
    Participación en Clases\\
    C1: Miércoles 28 de Marzo

    \begin{obs}
        Máx 10 min. de atraso\\
        \underline{No} celular
    \end{obs}

    \section*{Bibliografía}
    \begin{itemize}
        \item ¿Qué es el hombre?, Martin Buber

        \item La Filosofía, Karl Jaspers
        
        \item La critica a la Razón Pura, Immanuel Kant

        \item El mono desnudo, Desmond Morris
    \end{itemize}
    \newpage

    \tableofcontents

    \chapter{Clase 1}
    Objetivo: Justificar la enseñanza de Antropología Filosófica\\
    \section{Filosofía}
    Dos visiones, técnica y no técnica. La vision técnica es una vision académica, la otra es una vision del común de la gente.\\
    \section{¿Considera que la universidad es un espacio de formación profesional o un espacio de formación humana?}
    \section{¿Considera que en su disciplina hay obras de divulgación?¿Cual es la función que les atribuye?}

    \chapter{Clase 2}
    \section{Preguntas de Kant}
    \begin{enumerate}
        \item ¿Que puedo saber?

        \item ¿Que debo hacer?

        \item ¿Que puedo esperar?

        \item ¿Que es el hombre?
    \end{enumerate}

    \section{¿Puede establecer una relación entra las preguntas y la profesion que desea ejercer?}
    La primera pregunta tiene una relación directa con la disciplina en la que planeo desarrollarme, ya que las matemáticas, pueden verse de dos formas, el descubrimiento de verdades puras que existen por si mismas, o la creación de las mismas. En el primer caso, las matemáticas son una area de la cual se puede desprender una respuesta parcial a la pregunta, ya que seria ver que verdades puede el hombre llegar a conocer. En el segundo caso, las matemáticas son la búsqueda de encontrar la limitación de los conocimientos que el ser humano puede llegar a crear, o sea, ver las limitaciones mismas de la creatividad del ser humano en su búsqueda de conocimiento.\\
    La segunda pregunta y la tercera pregunta no tienen mucha relación, ya que estas tienen un carácter moral o ético, y religioso, respectivamente. Pero se puede argumentar que para la segunda pregunta una posible respuesta es ``extender los limites del conocimiento humano'', y usar esa respuesta como una maxima moral.\\
    La ultima pregunta se podría llegar a relacionar en el sentido que ``el hombre es quien crea nuevo conocimiento'' o ``el hombre es quien descubre el conocimiento''. Siendo la segunda una mirada mas Platónica, o hasta se podría decir judeo-cristiana. Mientras que la primera, podría decirse que es una mirada mas nihilista, o en otros términos, Nietzcheana.

    \section{¿Que concepcion del ser humano subyace a la disciplina que estamos estudiando?}
    La única característica que esta presente en toda concepcion del ser humano que subyace a las matemáticas, es la capacidad del uso de la razón. Cabe destacar, que esta capacidad no define al ser humano, y tomando un poco de la respuesta anterior, dependiendo la vision sobre si el conocimiento se ``crea'' o se ``descubre'' cual se seria su característica principal.

    \chapter{Clase 3}
    \section{El origen de la filosofía}
    El origen se puede referir a dos cosas, la causa de algo y como un comienzo. Aquí, nos referiremos a la primera. Hay ciertas experiencias limites que impulsan a filosofar.
    \section{¿Al interior de la historia de su disciplina hay casos de duda como motor de la investigacion?}
    \section{¿Podria dar ejemplos concretos de momentos en los que se experimentan situaciones limites?}
    \section{¿Como nos enfrentamos normalmente a las situaciones limite?}

    \chapter{Clase 4}
    \section{Comienzo historico de la filosofía}
    Condiciones sociales donde aparece la Filosofía. La democracia.
    \section{¿Considera que las condiciones historicas y sociales que dieron origen a la filosofía aun existen? Justifique su respuesta}
    \section{¿Podria nombar otras formas de acuerdo social a parte del dialogo?}
    \section{¿Considera actual y relavante un debate en torno a los fines que deseamos? Justificar su respuesta}
\end{document}