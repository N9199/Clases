\input{../Templates/Notetaking}

\title{Antropologia Filosofica}
\author{Nicholas Mc-Donnell}
\date{1er semestre 2018}

\pagenumbering{gobble}

\begin{document}
    \maketitle

    \section*{Objetivo General}
    La idea de estas discusiones es generar un debate en torno a los temas propuestos durante la primera sección de cada modulo. Esto permitirá generar un espacio donde cada alumno o grupo podrá desarrollar su punto de vista sobre las cuestiones propuestas y defenderlo por medio de argumentos y conceptos propios. Todo ello con el propósito de familiarizar al alumno con la actividad filosófica.

    \section*{Evaluaciones}
    {\raggedleft3 Controles de lectura después de cada unidad}\\
    Participación en Clases\\
    C1: Miércoles 28 de Marzo

    \begin{obs}
        Máx 10 min. de atraso\\
        \underline{No} celular
    \end{obs}

    \section*{Bibliografía}
    \begin{itemize}
        \item ¿Qué es el hombre?, Martin Buber

        \item El mono desnudo, Desmond Morris

        \item La critica a la Razón Pura, Immanuel Kant

        \item La Filosofía, Karl Jaspers
    \end{itemize}
    \newpage

    \tableofcontents

    \chapter{Clase 1}
    Objetivo: Justificar la enseñanza de Antropología Filosófica\\
    \section{Filosofía}
    Dos visiones, técnica y no técnica. La vision técnica es una vision académica, la otra es una vision del común de la gente.\\
    \section{¿Considera que la universidad es un espacio de formación profesional o un espacio de formación humana?}
    \section{¿Considera que en su disciplina hay obras de divulgación?¿Cual es la función que les atribuye?}
\end{document}