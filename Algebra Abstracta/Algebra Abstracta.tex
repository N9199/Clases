\documentclass[11pt]{book}
\usepackage[spanish]{babel}
\usepackage[utf8]{inputenc}
\usepackage[margin=1in]{geometry}          
\usepackage{graphicx}
\usepackage{amsthm, amsmath, amssymb}
\usepackage{mathtools}
\usepackage{manfnt}
\usepackage{setspace}\onehalfspacing
\usepackage[loose,nice]{units} 
\usepackage{enumitem}
\usepackage{hyperref}
\hypersetup{
    colorlinks,
    citecolor=black,
    filecolor=black,
    linkcolor=black,
    urlcolor=black
}

\title{Algebra Abstracta I}
\author{Nicholas Mc-Donnell}
\date{2do semestre 2017}

\renewcommand{\d}[1]{\ensuremath{\operatorname{d}\!{#1}}}
\renewcommand{\vec}[1]{\mathbf{#1}}
\newcommand{\set}[1]{\mathbb{#1}}
\newcommand{\func}[5]{#1:#2\xrightarrow[#5]{#4}#3}
\newcommand{\contr}{\rightarrow\leftarrow}
\newcommand*\cube{\mbox{\mancube}}

\DeclareMathOperator{\Ima}{Im}
\DeclareMathOperator{\mcd}{mcd}

\newtheorem{thm}{Teorema}[section]
\newtheorem{lem}[thm]{Lema}
\newtheorem{prop}[thm]{Proposición}
\newtheorem*{cor}{Corolario}

\theoremstyle{definition}
\newtheorem{defn}{Definición}[section]
\newtheorem{obs}{Observación}[section]

\pagenumbering{gobble}

\begin{document}
\maketitle

\newpage
\tableofcontents

\pagenumbering{arabic}
\chapter{Grupos}
\section{Grupos}
Una operación en un conjunto $S$ es una función.
\[
S\times S\rightarrow S
\]
\[
(a,b)\mapsto ab
\]
Ejemplo: $S= \textrm{Matrices de } n\times n$\\
La multiplicación y la suma son operaciones en este conjunto.\\
La operación puede ser asociativa: $(ab)c=a(bc)$\\
Esto implica que se le puede dar un y solo un sentido a $a_1\cdot a_2\cdot...\cdot a_n$\\
La operación es conmutativa: $ab=ba$\\
Ejemplo: Dado un conjunto $T$
\[S=\{ \textrm{funciones de } T\textrm{ en } T\}\]
En este conjunto la operación de composición es asociativa.\\
La operación tiene identidad (o neutro) para la operación en $S$ es el clásico neutro y es único:
\[
\exists e: \forall a, ae=ea=a
\]
Lo operación tiene inverso, con identidad $e$: $\forall a \exists a^{-1}: aa^{-1}=a^{-1}a=e$
\begin{lem}
Si la operación es asociativa, esto implica que los inversos son únicos.
\end{lem}
\begin{proof}

\[
b\cdot a =a\cdot b=e=a\cdot b'\quad / b\cdot
\]
\[
(b\cdot a)\cdot b=(b\cdot a)\cdot b' \quad /\textrm{Propiedad asociativa}
\]
\[
e\cdot b= e\cdot b'
\]
\[
b=b'
\]
Además, si $a$ y $b$ tienen inverso y la operación es asociativa, esto implica que $ab$ tiene inverso:
\[
(ab)^{-1}=b^{-1}a^{-1} \qedhere
\]
\end{proof}
\subsection{Grupos}
\begin{defn}
Un grupo es un conjunto $G$ con operación asociativa e identidad, tal que todo elemento tiene inverso.
\end{defn}
Ejemplo:  $GL_n\begin{pmatrix}\mathbb{Q} \\ \mathbb{R} \\ \mathbb{C} \end{pmatrix}=\{M\in\textrm{Matrices}\begin{pmatrix}\mathbb{Q} \\ \mathbb{R} \\ \mathbb{C} \end{pmatrix}:det(M)\neq 0\}$ (Grupo general lineal)\\
Este es un grupo no abeliano con el producto.

\begin{defn}[Orden]
El orden de un grupo $G$ es su cardinalidad $|G|$
\end{defn}
Sea $T$ un conjunto (no vací­o).
\[
S_{|T|}=\{\textrm{Las biyecciones de } T \textrm{ en si mismo}\}
\]
Entonces, $(S_{|T|},\circ)$ es un grupo.\\
Explicación:
\begin{itemize}
	\item Asociatividad, esta aparece como propiedad de la composición de las funciones
	
	\item Identidad, $e=1|_T$, la función identidad (biyectiva)
	
	\item Dada $f:T\rightarrow T$ biyección $\implies \exists f^{-1}:T\rightarrow T$ tal que $f\circ f^{-1}=f^{-1}\circ f=1|_T$
\end{itemize}

Si $|T|=n$ finito, $S_{|T|}=$ grupo de simetrí­a de $n$ elementos. Y $|S_n|=n!$.

\subsection{Motivación para estudiar esto:}

Resolver: $p(x)=a_nx^n+a_{n-1}x^{n-1}+...+a_1x+a_0=0$ por radicales, donde $a_i\in\mathbb{Q}$\\
Existe cierta manera de asociar un grupo a $p(x)$.\\
$Gal(p)=Gal(K|\mathbb{Q}$ donde $K=\mathbb{Q}(\alpha_1,\alpha_2,...,\alpha_n)$ (cuerpo).
\begin{thm}
Sea $f(x)\in \mathbb{Q}[x]$ irreducible:
\[
f(x)\textrm{ se resuelve por radicales}\iff Gal(f)\textrm{ es soluble}
\]
Ser soluble es la siguiente propiedad:
\[
\exists 1\triangleleft G_1\triangleleft G_2\triangleleft ...\triangleleft Gal(f):G_{i+1}/G_i\textrm{ es grupo abeliano y } G_i\textrm{ subgrupo de } G_i+1
\]
\end{thm}
Ejemplo: $f(x)=2x^5-10x+5$\\
$Gal(f)$ es isomorfo a $S_5$, pero $S_5$ no es soluble, lo que implica que $f(x)=0$ no se resuelve por radicales (fórmula).

\section{Subgrupos}
Def: Sea $G$ un grupo, $H\subset G$ es subgrupo $\iff$ $H$ es grupo (con la misma operación).
\subsection{Subgrupos triviales}
Sea $(G,\cdot)$ grupo y $e$ su neutro. $(G,\cdot)<(G,\cdot)$ y $(\{e\},\cdot)<(G,\cdot)$ (Notación de subgrupo).
\subsubsection{Subgrupos de $(\mathbb{Z},+)$}
Los pares son subgrupo de $(\mathbb{Z},+)$, además los múltiplos de $n$ son subgrupo de $(\mathbb{Z},+)$\\
$b\mathbb{Z}:=\{bk,k\in\mathbb{Z}\}$
\begin{prop}
	Todo subgrupo de $\mathbb{Z}$ es de la forma $b\mathbb{Z}$ ($H<\mathbb{Z}\iff H=b\mathbb{Z}$)
	\begin{proof}
		\begin{enumerate}
			\item Caso: $H$ sin números positivos $\iff H=\{0\}=0\mathbb{Z}$, esto es por los inversos (sin positivos no hay negativos)
			
			\item Caso: $H$ tiene números positivos $\implies\exists m\in\mathbb{Z}^+:\forall a>0\in H\implies m\leq a$
		\end{enumerate}
		Demostrar que $H=m\mathbb{Z}$.\\
		$\supseteq$\\
		Clausura e inducción $(\forall x\in H\implies x\in m\mathbb{Z})$ y también tirar inversos.\\
		$\subseteq$\\
		Todo elemento de $H$ es divisible por $m$.\\
		Sea $a\in H$ no divisible por $m\implies a=mc+r$ con $0<r<m$\\
		Como $a\in H,m\in H\implies mc\in H\implies -mc\in H$\\
		Por clausura $a-mc=r\in H$. Pero por buen orden es el más pequeño.
		\[
		\rightarrow\leftarrow
		\]
	\end{proof}
\end{prop} 
\subsubsection{Subgrupos Generados}

Sea $(G,\cdot)$ grupo y $S\subseteq G$, con $S\neq\emptyset$.\\
$<S>=\bigcap _{S\subseteq H<G}=$ es el subgrupo más pequeño que contiene a $S$.

\subsubsection{Subgrupos generados por un elemento}
Subgrupo de $G$ generado por $x$
\[
<x>:=\{e,x^1,x^{-1},x^2,x^{-2},...\}
\]

\begin{lem}
	\[x\in G, (G,\cdot)\]
	$\{\textrm{Los $K$ tal que $x^k=e$}\}$ es subgrupo de $\mathbb{Z}$
\end{lem}

\begin{defn}[Centro]
	Si $G$ es grupo, el centro de $G$ es:
\end{defn}
\[
Z(G)=\{z\in G: zg=gz\,\forall g\in G\}\trianglelefteq G
\]
Si $g\in G,z\in Z\implies gzg^{-1}=z$
\section{Relación de equivalencia y particiones}
\begin{defn}[Partición]
	Sea $S\neq\emptyset$ conjunto. Una partición $P$ de $S$ es una subdivisión $S$ en un subconjuntos disjuntos.
\end{defn}
Ejemplos:\\
$\{1,3\},\{2,5\},\{4\}$ es partición de $\{1,2,3,4,5\}$\\
Pares e impares en $\mathbb{Z}$
\subsection{Relación de equivalencia}
\begin{defn}[Relaciones de equivalencia]
	Una \underline{relación de equivalencia} en $S$ es una forma de relacionar elementos de $S$, $a\sim b$, tal que:
	\begin{enumerate}[label=(\arabic*)]
		\item Si $a\sim b,b\sim c\implies a\sim c$ (transitivo)
		
		\item Si $a\sim b\implies b\sim a$ (simétrico)
		
		\item $a\sim a$ (reflexivo)
	\end{enumerate}
\end{defn}

Ejemplo:\\
Los isomorfismos particionan el conjunto de objetos. Luego tenemos un conjunto que clasifica los objetos.\\
En Matemáticas \underline{clasificamos}. Cómo?\\
Buscamos \underline{isomorfismos} entre objetos que se presentan en formas distintas, pero estructuralmente son lo mismo.\\
Dado $S\neq\emptyset$\\
Partición de $S\equiv$ una relación de equivalencia\\
Despues de particionar se crea un nuevo conjunto, $\overline{S}=S/\sim=$ Conjunto de las particiones.\\
Ejemplo:\\
$\mathbb{Z}\rightarrow\overline{\mathbb{Z}}=\{\textrm{pares, impares}\},\textrm{impares}=\overline{1},\overline{-1},\overline{3},\textrm{pares}=\overline{0},\overline{-2},\overline{2}$\\
Queremos: $\overline{a}+\overline{b}=\overline{a+b}$\\
$\therefore$ Siempre hay un función sobreyectiva:
\[
S\rightarrow \overline{S}
\]
\[
a\mapsto \overline{a}
\]
Cualquier función entre conjuntos $S\xrightarrow{\varphi}T$ define una partición en $S:a\sim b$ si $\varphi(a)=\varphi(b)$
\[
\therefore \overline{S}=\{\varphi^{-1}(t):t\in T\}
\]
Asá­ tenemos morfismo biyectivo (isomorfismo)
\[
\overline{S}\xrightarrow{\overline{\varphi}}\Im(\varphi)
\]
Volviendo a grupos: Sea $\varphi:G\rightarrow G'$ morfismo entre grupos.\\
Ejemplo:
\[
\mathbb{C}^\times\xrightarrow{\varphi}\mathbb{R}_{>0}^\times \quad \varphi(a)=|a|
\]
\[
\varphi(a\cdot b)=|a\cdot b|=|a|\cdot |b|=\varphi(a)\cdot\varphi(b)
\]
Esta partición es $\{z\in\mathbb{C}^\times:|z|=r,r\in\mathbb{R}_{>0}\}$\\
Notar que: $\ker(\varphi)=\{z\in\mathbb{C}^\times:|z|=1\}$\\
Proposición: $G\xrightarrow{\varphi}G'$ morfismo de grupo con kernel $N$. Sean $a,b\in G\implies$
\[
\varphi(a)=\varphi(b)\iff a=b\cdot n\quad\textrm{para algún }n\in N
\]
\[
\iff a\cdot b^{-1}\in N
\]
Notación: $aN=\{an: n\in N\}$\\
$|aN|=|N|$\\
Dem:
\[
N\rightarrow aN
\]
\[n\mapsto an
\]
\[
\textrm{(Inyectiva)}\quad an=an'\implies n=n'
\]
\[
\textrm{(Sobreyectiva)}\quad an'\in aN \implies \varphi{n'}=an'
\]
Dem:
\[
\varphi(a)=\varphi(b)\iff \varphi(a)\varphi(b)^{-1}=e
\]
\[
\iff \varphi(ab^{-1})=e
\]
\[
\iff ab^{-1}\in \ker(\varphi)=N
\]
Def: Dado $H\leq G,a \in G$.
\[
aH=\{a\cdot h:h\in H\}
\]
se llama clase lateral izquierda. (clases laterales derechas $Ha$)\\
Proposición: Dado $H\leq G$, las clases laterales izquierdas particionan $G$.\\
Ejemplos:
\begin{itemize}
	\item Si $G$ es abeliano $\implies aH=Ha\quad \forall a\in G\implies$ la misma partición.
	
	\item $S_3=$ permutaciones de $3$ elementos $\iff$ simetrá­as del triángulo equilátero\\
	Si $H=\{1|,\sigma_1\}=<\sigma_1>$
\end{itemize}
Tarea: verificar que clases laterales coinciden o no coinciden.\\
Notación: la cardinalidad de las clases laterales se denota por $\left[G:H\right]$ (indice de $H$ en $G$)
\subsubsection{Corolario: Teorema de Lagrange}
Si $G$ es finito y $H\leq G\implies |H|\cdot [G:H]=|G|$\\
En particular: $|H|\, |\,|G|$\\
Más particular, $|a| \, |\,|G|\quad \forall a \in G$\\
Corolario: Si $G$ tiene orden $p$ primo y $a\in G\setminus\{e\}\implies G=<a>$\\
En efecto, $G$ es isomorfo a $\mathbb{Z}/p\mathbb{Z}$.\\
Dem:\\
Si $a\neq e\implies |a|\neq 1$\\
Pero $|a|\, |\, |G|=p\textrm{ primo}\implies G=<a>\implies |a|=p$
\[
\therefore \{a,a^2,a^3,...,a^{p-1},e\}=G=<a>
\]
Isomorfismo:
\[
\mathbb{Z}/p\mathbb{Z}\rightarrow G
\]
\[
\overline{i}\mapsto a^i
\]

\subsubsection{$\mathbb{Z}/p\mathbb{Z}$}
Sea $n\in\mathbb{Z}_{>0}$.\\
En $\mathbb{Z}$ definir la relacion de equivalencia:
\[
a\sim b\iff a-b\textrm{ es divisible por } n
\]
\[
\therefore \mathbb{Z}/n\mathbb{Z}=\{i\mathbb{Z}:i\in\mathbb{Z}\}=\{\overline{0},\overline{1},\overline{2},...,\overline{n-1}\}
\]
Tarea: la suma designada por $\overline{a}+\overline{b}=\overline{a+b}$ no depende de $a,b$ sino de su clase.
\[
\therefore (\mathbb{Z}/n\mathbb{Z},+)\textrm{ es un grupo de $n$ elementos, y $\mathbb{Z}/n\mathbb{Z}=<\overline{1}>$}
\]
Tarea: $G=\bigcup_{g\in G}gH=\bigcup_{g\in G}Hg$ y para $g,g'\in G$ $gH\cap f'H=\emptyset$ o $gH=g'H$, por relación de equivalencia ($a\sim b\iff a=bh$ para algún $h\in H$)\\
Notación: $[G:H]=$ \# de clases lat. izq.$=$\# de clases lat. der.
\[
\therefore |G|=[G:H]\cdot |H|
\]
\subsection{Meta}
Dado $n>0$ entero. Cuántos grupos $G$ existen $|G|=n$?\\
Si $n$ es primo $\implies$ hay sólo 1\\
Si $n=4\implies$ hay sólo 2
\[
\mathbb{Z}/4\mathbb{Z},\mathbb{Z}/2\mathbb{Z}\times\mathbb{Z}/2\mathbb{Z}
\]
Si $n=6\implies$ hay sólo 2
\[
\mathbb{Z}/6\mathbb{Z}, S_3
\]
Si $n=8\implies$ hay 5.

\subsubsection{Corolario}
Sea $\varphi:G\rightarrow G'$ morfismo entre grupos finitos
\[
\implies \ker(\varphi) \unlhd G,\varphi (G) \unlhd G'
\]
\[
|G|=|\ker\varphi|\cdot[G:\ker\varphi]=|\ker\varphi|\cdot |\varphi(G)|
\]
Prop: $H\unlhd G\iff$Toda clase lateral izquierda es derecha $\iff gH=Hg\forall g\in G$\\
Dem: Tenemos siempre
\[
gh=(ghg^{-1})g\forall g\in G
\]
Suponer $H\lhd G\implies ghg^{-1}\in H\implies gH\subseteq Hg$. También
\[
hg=g(g^{-1}hg)\forall g\in G
\]
\[
\implies Hg\subseteq gH\implies gH=Hg
\]
Si $H\ntriangleleft G\implies \exists ghg^{-1}\notin H$
\[
\implies gh\in Hg
\]
\[
\therefore Hg\neq gH
\]
$Hg=g'H$?
No, ya que las clases laterales izquierda y derecha particionan. Luego, si $Hg=g'H$
\[
\implies g\in g'H\textrm{ y }g\in gH
\]
\[
\implies g'H\cap gH\neq\emptyset\implies g'H=gH
\]
\[
\rightarrow\leftarrow
\]
\section{Restricción de morfismos a subgrupos}
Obs: $K,H\leq G\implies K\cap H\leq H$
\[
K\lhd G\implies K\cap H\lhd H
\]
Obs: $\varphi:G\rightarrow G'$ morfismo
\[
\implies \varphi|_H:H\rightarrow G'\quad\textrm{es morfismo}
\]
Prop: $\varphi:G\rightarrow G'$ morfismo, $h'\leq G$. Sea $\varphi^{-1}(H')=\tilde{H}$
\begin{enumerate}[label=(\alph*)]
	\item $\tilde{H}\leq G$
	
	\item $H'\lhd G'\implies \tilde{H}\lhd G$
	
	\item $\tilde{H}$ contiene a $\ker\varphi$
	
	\item $\varphi|_H:\tilde{H}\rightarrow H'$ tiene kernel $\ker\varphi$
\end{enumerate}
Dem: p.d. $\tilde{H}\leq G$
\begin{enumerate}
	\item $e\in\tilde{H}$ ya que $\varphi(e)=e'\in H'$
	
	\item $x,y\in\tilde{H}\implies\varphi(xy)=\varphi(x)\varphi(y)\in H'$
	
	\item $x\in\tilde{H}\implies x^{-1}\in\tilde{H}$
\end{enumerate}
\section{Producto de grupos}
Def: Dados $g,G'$ grupos, podemos formar un \underline{nuevo grupo}:
\[
G\times G'=\{(g,g'):g\in G,g'\in G'\}
\]
Con la operación:
\[
(a,b)\cdot_{G\times G'}(c,d)=(a\cdot_G b,c\cdot_{G'} d)
\]
\subsection{Porqué es grupo?}
\begin{enumerate}[label=(\alph*)]
	\item Identidad: $(e,e')$
	
	\item Invertibilidad: Para $(a,b),(a,b)^{-1}=(a^{-1},b^{-1})$
\end{enumerate}
Ejemplos:
\begin{itemize}
	\item $S_3\times\mathbb{Z}/2\mathbb{Z}$ es un grupo de orden 12 y no es abeliano
	
	\item $\mathbb{Z}/2\mathbb{Z}\times\mathbb{Z}/2\mathbb{Z}$ es grupo de orden 4 y \underline{no} es $\mathbb{Z}/4\mathbb{Z}$
	
	\item $\mathbb{Z}/2\mathbb{Z}\times\mathbb{Z}/3\mathbb{Z}$ es grupo abeliano de orden 6.\\
	Notar que es generado por el $(1,1)$, por lo que es isomorfo a $\mathbb{Z}/6\mathbb{Z}$
\end{itemize}
Sean $n,m$ coprimos enteros
\[
\implies \mathbb{Z}/n\mathbb{Z}\times\mathbb{Z}/m\mathbb{Z}\simeq\mathbb{Z}/m,\mathbb{Z}
\]

\chapter{Simetrías}
\section{Simetrías en figuras planas}
\begin{defn}[Isometría]
Una función $m:\mathbb{R}^2\rightarrow\mathbb{R}^2$ es una \underline{isometría} si preserva distancia, es decir $\forall p,q\in\mathbb{R}^2$
\[ dist(P,Q)=dist(m(P),m(Q))\]
\end{defn}
\begin{prop}
$m:\mathbb{R}^2\rightarrow\mathbb{R}^2$ isometría
\[\implies m\left(\begin{bmatrix} x\\ y\end{bmatrix}\right)=\begin{bmatrix} x_0\\y_0\end{bmatrix}+M \begin{bmatrix} x\\ y\end{bmatrix}\]
\[\textrm{Con }M^tM=\begin{bmatrix} 1 && 0\\ 0 && 1\end{bmatrix}\]
Notar que $\begin{bmatrix} x_0 \\ y_0\end{bmatrix}=m\left(\begin{bmatrix} 0 \\ 0\end{bmatrix}\right)$
\end{prop}
\begin{proof}
\[\mathbb{R}^2\xrightarrow{m}\mathbb{R}^2\xrightarrow{-m\begin{bmatrix} 0\\ 0\end{bmatrix}}\mathbb{R}^2\]
\[T=\textrm{ isometría con }T\begin{bmatrix} 0 \\ 0\end{bmatrix}\]
\[\therefore \textrm{ Sea }v\in \mathbb{R}^2\setminus\vec{0}, T(v)\]
\[T(\lambda v)=\lambda T(v)\quad\textrm(ejercicio)\]
\[T(u+v)=T(u)+T(v)\]
Luego $T$ es transformación lineal: $\exists M\in M_{2\times 2}$
\[ T\begin{bmatrix} x\\ y\end{bmatrix}=M\begin{bmatrix} x\\ y\end{bmatrix}, M=\begin{bmatrix} A && C\\ B && D\end{bmatrix}\]
\[ M\begin{bmatrix} 1\\ 0\end{bmatrix}=\begin{bmatrix} A\\ B\end{bmatrix}\implies A^2+B^2=1\]
\[ M\begin{bmatrix} 0\\ 1\end{bmatrix}=\begin{bmatrix} C\\ D\end{bmatrix}\implies C^2+D^2=1\]
\[\begin{bmatrix} A\\ B\end{bmatrix}\cdot\begin{bmatrix} C\\ D\end{bmatrix}=0\implies AC+BD=0\]
\[\therefore \begin{bmatrix} A && B\\ C && D\end{bmatrix}\begin{bmatrix} A && C\\ B && D\end{bmatrix}=\begin{bmatrix} 1 && 0\\ 0 && 1\end{bmatrix}\qedhere\]
\end{proof}
\begin{cor}
Isometrías son biyecciones
\end{cor}
\begin{cor}
Isometrías forman un grupo con la composición.
\end{cor}
\begin{defn}[Simetría]
Sea $F\subseteq\mathbb{R}^2$ una figura. Una \underline{simetría} es una isometría tal que
\[m(F)=F\]
\[Sim(F)\leq Sim(\mathbb{R}^2\]
\end{defn}
\section{Acciones de grupo}
\begin{defn}
	$G=$ Grupo, $S\neq\emptyset$ conjunto. Sea $G\times S\rightarrow S,(g,s)\mapsto g\cdot s$ tal que:
	\begin{enumerate}[label=(\alph*)]
		\item $e\cdot s= s,\forall s\in S$

		\item $(gg')\cdot s=g\cdot(g'\cdot s)\forall g,g'\in G,\forall s\in S$
	\end{enumerate}
	Si tenemos esto decimos que $G$ actua en $S$
	\[(G S)\]
\end{defn}
Ejemplos:
\begin{itemize}
	\item $F\subseteq\mathbb{R}^2$ figura.
	\[Sim(F)=G, S=F\]
	\[\therefore G S\]
	\[G\times S\rightarrow S\]
	\[(g,p)\mapsto g\cdot p=g(p)\]

	\item $G=\mathbb{Z}/2\mathbb{Z}, S=\mathbb{C}$
	$G S$ por conjugación
	\[\{0,1\}\times\mathbb{C}\rightarrow\mathbb{C}\]
	\[0\cdot z= z\]
	\[1\cdot z=\bar{z}\]
	
\end{itemize}
\begin{obs}
	Si $G S$ y $g\in G$
	\[\implies m_g:S\rightarrow S, m_g(s)=g\cdot s\quad\textrm{ y es biyección:}\]
	\begin{itemize}
		\item Inyectiva: 
		\[g\cdot s=g\cdot s'\quad /g^{-1}\cdot\]
		\[g^{-1}\cdot (g\cdot s)=g^{-1}\cdot (g\cdot s')\implies e\cdot s=e\cdot s'\]
		\[\implies s=s'\]

		\item Sobreyectiva: Dado $s\in S$, $g\cdot ?=s\therefore ?=g^{-1}\cdot s$
	\end{itemize}
	
	Lo principal de $G S$ es que particiona a $S$ en órbitas.
	\[O_s=\{s'\in S:g\cdot s=s'\textrm{ para algún }g\in G\}\]
\end{obs}
Ejemplo: $G=D_4, S=\square$, $D_4$ verlo como $Sim(\square)$\\
Las órbitas de $G\circlearrowleft S$ definen una relación de equivalencia:
\[s\sim s'\iff s'=g\cdot s\exists g\in G\]
\[\therefore S\textrm{ es unión de órbitas disjuntas}\]
\begin{defn}
	Si $S$ es una órbita $\implies$ decimos que $G$ actua \underline{transitivamente}. $\iff$ Dados $s,s'\in S\exists g\in G$ tal que $s=g\cdot s'$
\end{defn}
Ejemplo: $Sim(\mathbb{R}^2$ actua en $\mathbb{R}^2$ transitivamente.
\begin{defn}
	El \underline{estabilizador de $s\in S$} es $G_s=\{g\in G: g\cdot s=s\}$
\end{defn}
Ejemplo: $G_{(0,0)}$ para $G=Sim(\mathbb{R}^2)\circlearrowleft\mathbb{R}^2$
\[\therefore G_{(0,0)}=\{M\in Mat_{2\times 2}(\mathbb{R}):M^tM=Id\}=\textrm{ grupo ortogonal }= O(2,\mathbb{R})\]
\begin{obs}
	$G_s\leq G$
\end{obs}
Ejemplo: $G=Sim(\mathbb{R}^2), S=\{\triangle\in\mathbb{R}^2\}$
$\implies$ las órbitas son los $\triangle_s$ congruentes.
\[G_\triangle=\{e\}\quad G_{\triangle\textrm{ (equilátero)}}\simeq S_3\]
\[G_{\triangle\textrm{ (isosceles)}}\simeq \mathbb{Z}/2\mathbb{Z}\]
\subsection{Acción en clases laterales}
\begin{defn}
	$H\leq G\implies$ clases laterales izquierdas particionan $G$\\
	\underline{Notación:} part.$=G/H$\\
	$G$ actua en $G/H$!
	\[G\times G/H\rightarrow G/H\]
	\[(g, aH)\mapsto gaH=g\cdot aH\]
	es acción y transitiva.
\end{defn}
\begin{prop}
	$G\circlearrowleft S, s\in S, H=G_s, O_s$ la órbita de $s$. Luego $G/H\xrightarrow{\varphi}O_s,\varphi(aH)=a\cdot s$ es biyección.
	\begin{proof}
		\begin{itemize}
			\item Bien definido: Sean $aH=bH\iff \exists h\in H: b=ah$
			\[\implies b\cdot s= ah\cdot s= a\cdot(h\cdot s)= a\cdot s\]
			
			\item Inyectiva:
			\[a\cdot s=b\cdot s\implies s=a^{-1}b\cdot s\implies a^{-1}b\in G_s=H\]
			\[\iff aH=bH\]

			\item  Sobreyectiva: Si $g\cdot s\in O_s\implies\varphi(gH)=gs$
		\end{itemize}
	\end{proof}
\end{prop}
\begin{prop}
	\[G\circlearrowleft S, s\in S,\exists a\in G: s'=a\cdot s\]
	\begin{enumerate}[label=(\alph*)]
		\item $aG_s=\{g\in G:g\cdot s=s'\}$

		\item $G_{s'}=aG_sa^{-1}$
	\end{enumerate}
	\begin{proof}
		\begin{enumerate}[label=(\alph*)]
			\item Si $b\in aG_s\implies b=ah$ para algún $h\in G_s$
			\[\implies b\cdot s=ah\cdot s= a\cdot(h\cdot s)=a\cdot s= s'\implies b\in\textrm{Derecha}\]
			\[ b\in\textrm{Derecha}, b\cdot s=s'=a\cdot s\implies a^{-1}bs=s\]
			\[a^{-1}b\in G_s\implies b\in aG_s\]

			\item Si $h\in G_s$
			\[\implies h\cdot s'=s'\implies h\cdot(a\cdot s)=a\cdot s\]
			\[\implies a^{-1}ha\cdot s=s\implies a^{-1}ha\in G_s\]
			\[\implies h\in aG_sa^{-1}\]
			\[h\in aG_sa^{-1}\]
			\[\implies h=ah'a^{-1}\implies h\cdot s'=ah'a^{-1}\cdot s'=ah'a^{-1}\cdot as=s'\qedhere\]
		\end{enumerate}
	\end{proof}
\end{prop}
Ejemplo: Sea $(a,b)\in\mathbb{R}^2$
\[\therefore G_{(a,b)}=t_{(a,b)}G_{(0,0)}t^{-1}_{(a,b)}\]
\[G_{(a,b)}=\{f\in Sim(\mathbb{R}^2):f(x,y)=t^{-1}(M(t_{(a,b)}(x,y)))\}\]
\section{Simetrías en $\set{R}^3$}
\begin{thm}
	Todo grupo finito $G$ de $SO_3$ es uno de los siguientes:
	\begin{itemize}
		\item $C_k:$ grupo ciclico de orden $k$

		\item $D_k:$ Diedral de orden $2k$ (isometrías de un poligono regular de $k$ lados)

		\item $T:$ Tetraedral; 12 rotaciones de llevar un tetraedro en si mismo.

		\item $O:$ Octaedral; 24 rotaciones que llevan un cubo o un octaedro en si mismo.

		\item $I:$ Icosaedral; 60 rotaciones que llevan dodecaedros o icoseaedros en si mismo.
	\end{itemize}
	\begin{proof}
		Sea $G\leq SO_3$ finito: $|G|=N$\\
		Si $g\in G$ y $g\neq Id\implies g$ fija 2 puntos en la esfera.
		\[P=\{Pg,P'gLg\in G\}=\textrm{Polos de $G$}\]
		$G\circlearrowright P:G$ envia polos en polos.
		\begin{proof}
			$p\in P, g\in G$. Necesitamos $gp\in P$ fijo por algún $g'\in G$. Asumir $x\neq Id, x\in G$ tal que $xp=p\implies gxg^{-1}(gp)=gp$
		\end{proof}
		\underline{La Idea} es contar polos. 
		Creemos que hay $2n-2$ polos, pero no ya que el estabilizador de $p\in P$ es ciclico de orden $r_p$.
		\[G_p\textrm{ es cíclico}\]
		\[\therefore |O_p|=\frac{|G|}{|G_p|}\]
		Digamos que $|O_p| = n_p$
		\[r_p\cdot n_p=N\]
		\# elementos en $G$ con $p$ polo $=r_p-1$
		\[\implies\sum_{p\in P}(r_p-1)=2(N-1)\]
		Dividir $P$ en órbitas:
		\[O_1,...,O_s\]
		disjuntas: $|O_i|=n_i$
		\[\therefore\sum^s_{i=1}n_i(r_i-1)=2N-2\]
		Como $r_in_i=N$, entonces
		\[\sum^s_{i=1}\frac{N}{r_i}(r_i-1)=2N-2\]
		\[\therefore 2-\frac{2}{N}=\sum^s_{i=1}\left(1-\frac{1}{r_i}\right)\]
		Notar que:
		\[2-\frac{2}{N}<2\textrm{ y }1-\frac{1}{r_i}\geq \frac{1}{2}\]
		\[\therefore 2>\frac{1}{2}s\implies 4>s\implies s\leq 3\]
		\begin{enumerate}[label=(\arabic* órbitas):]
			\item $2-2/N=1-1/r_1$, $2-2/N\geq 1$ y $1-1/r_1<1$
			\[\contr\]
			Por lo que no existe este caso.
			
			\item $s=2$
			\[2-\frac{2}{N}=1-\frac{1}{r_1}+1-\frac{1}{r_2}\]
			\[\implies\frac{2}{N}=\frac{1}{r_1}+\frac{1}{r_2}\]
			Pero $r_i\leq N$
			\[\therefore r_1=r_2=N\]
			\[\therefore n_1=n_2=1\]
			\[\therefore G_p=G=G_{p'}\]
			Rotaciones $2\pi/N$

			\item $s=3$
			\[\frac{2}{N}=\frac{1}{r_1}+\frac{1}{r_2}+\frac{1}{r_3}-1\]
			Asumir que $r_1\leq r_2\leq r_3$
			\[\therefore r_1=2\]
			\begin{enumerate}[label=(\roman*)]
				\item Asumimos $r_1=r_2=2,r_3=r$
				\[\therefore N=2r\implies n_3=2.\]
				\[O_3=\{p,p'\}\]
				\[\therefore G\simeq D_r\]

				\item Asumimos $r_1=2,r_i\geq 3$, pero $r_2\geq 4, r_3\geq 4\implies \contr$ y $r_2=3, r_3\geq 6\implies \contr$
				\[\therefore r_2=3\]
				\begin{center}
					\begin{tabular}{ c | c | c | c | c | c }
						a & a & a & a & a & a \\ \hline
						a & a & a & a & a & a \\
						a & a & a & a & a & a \\
						a & a & a & a & a & a \\
						a & a & a & a & a & a
					\end{tabular}
				\end{center}
			\end{enumerate}
		\end{enumerate}
	\end{proof}
\end{thm}
\subsection{Acciones de grupos en si mismos}
\subsection{Acciones de grupos en subconjuntos}
Ej: $O=$ grupo de octaedro de 24 rotaciones del cubo
\[O\circlearrowright\cube\]
\[S=\textrm{Conjunto de vértices del cubo}\implies O\circlearrowright S\]
\[S'=\textrm{Pares no ordenados de vértices de }\cube\quad(\textrm{es decir,}\binom{8}{2}=28\textrm{ pares.})\]
\[\implies O\circlearrowright S'\]
\begin{enumerate}[label=(\roman*)]
	\item \{Pares de vértices en un lado\}$=12$

	\item \{Pares de vértices opuestos en una cara\}$=12$

	\item \{Pares de vértices opuestos en el cubo\}$=4$
\end{enumerate}
\[\therefore 28=12+12+4\]
Si $G\circlearrowright S$
\[\implies G\circlearrowright\textrm{ Subconjuntos de }S\]
\[U\subset S, g\cdot U=\{g\cdot u:u\in U\}\]
\[G_u=\{g\in G:gU=U\}\]
\begin{prop}
	$H\circlearrowright S$ y $U\subset S$\\
	Entonces:
	\[H\textrm{ estabiliza a }U\iff U=\bigcup\textrm{ algunas órbitas}\]
	\begin{proof}
		Dibujo
	\end{proof}
\end{prop}
\begin{prop}
	\[U\subset G, |G|<\infty\]
	$G\circlearrowright G$ por multiplicación por la izquierda así $G\circlearrowright$ subconjuntos de $G$.
	\[\implies |G| | |U|\]
	\begin{proof}
		$H=G_u$ y así $H$ estabiliza a $U\iff U=\bigcup$ algunas órbitas $=\bigcup_{\textrm{algunos }g\in G}g\cdot H$, pero $|gH|=|H|\implies |U|=\lambda |H|$
	\end{proof}
\end{prop}
$G\circlearrowright$ subconjuntos de $G$ por conjugación.\\
Un subgrupo $H\triangleleft G\iff$ su órbita contiene sólo a $H$.\\
Qué sucede si $H\ntriangleleft G$?\\
\underline{Respuesta:} Definir Normalizador de $H$ en $G$:
\begin{defn}[Normalizador]
	\[H\subseteq N(H)=\{g\in G:gHg^{-1}=H\}\]
\end{defn}
\section{Teoremas de Sylow}
"Describir subgrupos de orden primo de un grupo finito"\\
Sea $G$ un grupo de orden $n=p^e\cdot m$ $p$ primo, $e\geq 1$, $p$ no divide a $m$.\\
\underline{Ejm:} $n=100, p=2 e=2\implies 100=2^2\cdot 25$
\begin{thm}[Sylow 1]
	$\exists$ un subgrupo de orden $p^e$.
	\begin{lem}
		Si $S$ es un conjunto con $n=p^e\cdot m$ elementos, con $p$ primo y $p$ no divide a $m$, y $p^e<n\implies$ El número de subconjuntos de cardinalidad $p^e$ es
		\[N=\binom{n}{p^e}\quad p\textrm{no divide a} N\]
		\begin{proof}
			$N=\binom{n}{p^e}$ por introducción al álgebra.
			\[\binom{n}{p^e}=\frac{n\cdot(n-1)\cdot...\cdot (n-(p^e-1))}{p^e\cdot (p^e-1)\cdot...\cdot 1}\]
			Notar que $n-k$ y $p^e-k$ tienen exactamente la misma cantidad de $p^i$ como factor. Simplemente escribir $k=p^i\cdot l$, donde $p$ no divide a $l$, $i\geq 0$ y $e>i$.
			\[n-k=p^e\cdot m-p^i\cdot l=p^i(p^{e-i}\cdot m-l)\]
			\[p^e-k=p^e-p^i\cdot l=p^i(p^{e-i}-l)\qedhere\]
		\end{proof}
	\end{lem}
	\begin{proof}
		Sea $S=\{$ subconjuntos de $G$ de cardinalidad $p^e\}$\\
		$G\circlearrowright S$ por multiplicación por la izquierda
		\[N=\sum_{\textrm{órbitas disjuntas}}|O|\]
		por el lema anterior, $p$ no divide a $N$
		\[\implies \exists O:\mcd(|O|,p)=1\]
		Sea $U\in O$
		\[\therefore |G_u||O|=p^e\cdot m\]
		Como $G_U$ estabiliza a $U$
		\[\implies U=\bigcup\textrm{órbitas}\]
		órbitas$=g\cdot G_U$
		\[\therefore |G_U|\mid|U|\]
		\[\implies|U|=p^e\qedhere\]
	\end{proof}
\end{thm}
\begin{cor}
	Si $p$ primo divide a $|G|\implies G$ tiene un elemento de orden $p$.
	\begin{proof}
		Sea $H<G$ con $|H|=p^e$, por Sylow 1\\
		Sea $x\in H,x\neq e$.\\
		Entonces ord$(x)=p^\alpha,1\leq 1 \alpha\leq e$\\
		Si $\alpha=1$, listo.\\
		Si $\alpha>1\implies$
		\[1|=x^{p^{\alpha}}=x^{p^{\alpha-1}\cdot p}=(x^{p^{\alpha-1}})^p\]
		\[\therefore x^{p^{\alpha-1}}\textrm{ tiene orden }p\qedhere\]
	\end{proof}  
\end{cor}
\begin{cor}
	$|G|=6$
	\[\implies G\simeq\set{Z}_6\vee G\simeq D_3\]
	\begin{proof}
		Sea $x$ de orden 3 e $y$ de orden 2 en $G$.
		\[\implies x^iy^i\quad 0\leq i\leq 2, 0\leq j\leq 1\]
		Son todos distintos.
		\[x^iy^j=x^{i'}y^{j'}\]
		\[\therefore x^{i-i'}=y^{j'-j}=e\]
		\[\therefore G=\{e,x,x^2,y,xy,x^2y\}\]
		Notar que $yx\neq e,x,x^2,y$
		\[\implies yx=xy\vee yx=x^2y\qedhere\]
	\end{proof}
\end{cor}
\begin{defn}[Sylow p]
	$G$ finito de orden $p^em$, $p$ primo, $p\nmid m$. Un $H< G$ de orden $p^e$ se llama \underline{Sylow p}.
\end{defn}
\begin{thm}[Sylow 2]
	Sea $K<G$,$p\mid|K|$ y $H$ es un Sylow p.
	\[\implies\exists g\in G: K\cap gHg^{-1}\subset K\]
	es un Sylow p de $K$
	\begin{proof}
		$S=\{$ clases laterales izquierdas de $H\}=G/H$\\
		$G\circlearrowright S$ por multiplicación por izquierda es acción transitiva, $H$ estabiliza a $H$\\
		\underline{Clave:} hacer actuar $K\circlearrowright S$ por multiplicación izquierda. Descomponer $S$ en $K$-órbitas.\\
		$H$ Sylow-p
		\[\implies |S|=[G:H]=m\implies p\nmid m\]
		\[\therefore \exists \textrm{K-órbita }O:\mcd(|O|,p)=1\]
		Digamos
		\[O=O(aH)\]
		Sea $H'=aHa^{-1}$. [Es Sylow-p para $G$]\\
		Notar que $H'$ es estabilizador de $aH$ por la acción de $G$.\\
		$\implies$ El estabilizador de $aH$ por $K$ es $H'\cap K$ y $[K:H'\cap K]=|O(aH)|$. Y $[K:K\cap H']$ es coprimo a $p$.\\
		Notar que $H'$ es p-grupo de $K$ y así $K\cap H'$ es p-grupo.\\
		$\implies H'\cap K$ es Sylow-p para $K$
	\end{proof}
	\begin{cor}
		\
		\begin{enumerate}[label=(\alph*)]
			\item $K<G$ p-grupo\\
			$\implies K\subset$ algún Sylow p
			
			\item Sylow p de $G$ aon conjugados entre si.
		\end{enumerate}
		\begin{enumerate}[label=(\alph*)]
			\item \begin{proof}
				$gHg^{-1}$ es Sylow p si $H$ es Sylow p,$\forall g\in G$. Sea $K$ un p-grupo en $G$.\\
				Por teo Sylow 2, $gHg^{-1}\cap K=K$ es p-grupo de $K$
				\[\implies gHg^{-1}\supset K\qedhere\]
			\end{proof}

			\item \begin{proof}
				$K$ y $H$ Sylow p. Por Sylow 2 $\exists g:gHg^{-1}\supset K$ es Sylow p para $K$.
				\[\therefore gHg^{-1}\cap K=K\]
				\[\therefore gHg^{-1}\supset K\]
				\[\implies gHg^{-1}=K\]
			\end{proof}
		\end{enumerate}
	\end{cor}
\end{thm}
\begin{thm}[Sylow 3]
	$|G|=p^e\cdot m$
	\[s=\#\{\textrm{Sylow p en }G\}\]
	\[\implies s ||G|\wedge s\equiv 1\mod p\]
	\begin{proof}
		Por Corolario $(b)$: Todos los Sylow-p son conjugados a uno: $H$\\
		$\#\{$Sylow-p$\}=s[G:N(H)]$ $N(H)=$normalizador de $H$ en $G$
		\[\therefore s\mid m\]
		$S=\{H_1=H,H_2,...,H_s\}=$Sylow-p\\
		$H\circlearrowright S$ conjugar con elementos de $H$.
		\[\textrm{Una órbita consiste de 1 elemento}\iff H\subset N_i=\textrm{normalizador de }H_i\]
		\[\implies H=H_i\]
		$\therefore$ la única órbita por 1 elemento es la de $H$
		\[s=\sum_{O\textrm{ disjuntas}}|O|=1+\sum_{\textrm{resto}}|O|\]
		\[\implies s\equiv 1\mod p\qedhere\]
	\end{proof}
	\begin{cor}
		$H$ Sylow-p de $G$.
		\[H\triangleleft G\iff s=1\]
		\begin{proof}
			Si $H\triangleleft G\implies gHg^{-1}=H\quad\forall g\in G$. Teo Sylow 2, tenemos $s=1$.\\
			Si $s=1\implies gHg^{-1}\in$ Sylow p$=\{H\}$
			\[\implies gHg^{-1}=H\forall g\in G\qedhere\]
		\end{proof}
	\end{cor}
\end{thm}
\end{document}