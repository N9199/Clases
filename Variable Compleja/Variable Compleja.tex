\documentclass{notetaking}

\title{Variable Compleja - MAT2705}
\author{Nicholas Mc-Donnell}
\date{}
\gdate{Segundo Semestre de 2019}

\setlist[enumerate]{label=\arabic*)}

\begin{document}
\maketitle
\tableofcontents
\newpage

\section*{Preliminares}

\subsection*{Evaluaciones}
\begin{itemize}
    \item 3 Interrogaciones (27 Sept, 22 Oct, 20 Nov)
    \item 4 Tareas
    \item 1 Examen
\end{itemize}
Eximición con promedio 5.5
\[
    NF:=\frac{I_1+I_2+I_3+PT+E}5
\]

\subsection*{Textos}
Libro: Gamelin: Complex Analysis

\part{}
\section{Números Complejos}
Se identifica \(z=x+iy\in\set{C}\) ssi \((x,y)\in\set{R}^2\) (\(\set{C}\iff\set{R}^2\)). En este contexto se define \(i^2=-1\), se definen las operaciones decentes dado \(\set{C}\simeq\set{R}[x]/\paren{x^2+1}\) como cuerpo.\\
También se puede identificar \(z=r\exp(i\theta)\in\set{C}\), donde \(r=\abs{z}=\sqrt{x^2+y^2}\) y \(\tan\theta=\frac{y}{x}\)

\begin{defn}[Conjugado]
    El conjugado de \(z\) se anota \(\bar{z}\) y se define \(\bar{z}=a-ib\) si \(z=a+ib\), \(\bar{z}=r\exp{-i\theta}\) si \(z=r\exp(i\theta)\).
\end{defn}

\subsection{Proyección Esterográfica}
\begin{defn}[Proyección Estereográfica]
    Cada \(z\in\set{C}\) identificarlo con un punto en \(S^2\) (la esfera)
\end{defn}

\begin{thm}
    Bajo proyección estereográfica círculos y rectas en el plano corresponden a intersecciones de planos en \(S^2\) (círculos en la esfera)
\end{thm}

\subsection{Funciones en \(\set{C}\)}
La topología en \(\set{C}\) es la heredada de \(\set{R}^2\) y bajo esta tenemos nociones de convergencia y continuidad.

\begin{ejm}\
    \begin{enumerate}
        \item \(f(z)=a_nz^n+\cdots+a_0\) es continua
        \item \(f(z)=\bar{z}\) continua
        \item \(f(z)=\exp(z)\) continua
        \item Definiendo \(\ln(z)=\ln(r\exp(i\theta))=\ln(r)+i\theta)\) con \(\theta\in(-\pi,\pi]\) (la rama principal), la función no es continua en todo el eje \(\{y\leq 0\}\)
        \item Definiendo \(\sqrt{z}=\exp(\frac12\ln(z))\) no es continua.
    \end{enumerate}
\end{ejm}

\subsection{Funciones Analíticas}
\begin{obs}
    Dado \(f:\set{C}\rightarrow\set{C}\), tal que \(((x,y)\mapsto(u(x,y),v(x,y)))\), \(x+iy\mapsto u(z)+iv(z)\). Esta función es diferenciable en \(\Omega\subseteq\set{R}^2\) en el sentido de CVV ssi \(\frac{\partial{u}}{\partial{x}},\frac{\partial{u}}{\partial{y}},\frac{\partial{v}}{\partial{x}},\frac{\partial{v}}{\partial{y}}\) existen y son continuas en \(\Omega\). El diferencial es:
    \[
        \begin{bmatrix}
            \frac{\partial{u}}{\partial{x}}       & \frac{\partial{u}}{\partial{y}} \\
            \frac{\partial{v}}{\partial{x}} & \frac{\partial{v}}{\partial{y}}
        \end{bmatrix}
    \]
\end{obs}

\begin{defn}
    Decimos que \(f:\Omega\subseteq\set{C}\rightarrow\set{C}\) es diferenciable en \(z_0\) ssi \(\lim_{z\rightarrow z_0}\frac{f(z)-f(z_0)}{z-z_0}=f'(z_0)\) existe. \(f\) se dice analítica en \(Omega\) ssi \(f\) es diferenciable en el sentido anterior para todo \(z_0\in\Omega\) y \(f'(z)\) es continua en \(\Omega\).
\end{defn}

\begin{ejm}
    \(f(z)=z^m\), dado \(\Delta z=z-z_0\), \(\frac{f(z)-f(z_0)}{z-z_0}=\frac{f(z_0+\Delta z)-f(z_0)}{\Delta z}\). Por lo que
    \begin{align*}
        \frac{(z_0+\Delta z)^m-z_0^m}{\Delta z} & =\frac{\sum_{j=0}^m\binom{m}{j}z_0^{m-j}(\Delta z)^j-z_0^m}{\Delta z}\\
        &=\sum_{j=1}^m\binom{m}{j}(\Delta z)^{j-1}z_0^{m-j}\\
        &\xrightarrow[\Delta z\rightarrow 0]{}mz_0^{m-1}
    \end{align*}
    Entonces \(f\) es diferenciable en \(\set{C}\) y la derivada es la usual.
\end{ejm}

\begin{ejm}
    \(f(z)=\bar{z}\)
    \begin{align*}
        \frac{f(z_0+\Delta z)-f(z_0)}{\Delta z} &= \frac{\overline{x_0+iy_0+\Delta x+i\Delta y}-\overline{x_0+iy_0}}{\Delta x+i\Delta y}\\
        &= \frac{\Delta x-i\Delta y}{\Delta x+i\Delta y}
    \end{align*}
    Si se toma \(\Delta z=\varepsilon\), entonces el límite es \(1\), por otro lado si \(\Delta z=i\varepsilon\) el límite es \(-1\).\\
    Por lo que no es diferenciable en este contexto, pero si es diferenciable como función de \(\set{R}^2\) a \(\set{R}^2\).
\end{ejm}

Veamos ahora, como relacionar ambos conceptos. Si existe el límite, este tiene que ser igual independiente de la dirección. Por lo que, notamos que si \(\Delta z=\varepsilon\), el límite es \(\frac{\partial u}{\partial x}+i\frac{\partial v}{\partial x}\), en cambio si \(\Delta z=i\varepsilon\), el límite es \(-i\frac{\partial u}{\partial y}+\frac{\partial v}{\partial y}\). Por lo que una condición necesaria para la diferenciabilidad en sentido complejo es\footnote{Condiciones de Cauchy-Riemann}:
\begin{align*}
    \frac{\partial u}{\partial x}&=\frac{\partial v}{\partial y}\\
    \frac{\partial v}{\partial x}&=-\frac{\partial u}{\partial y}
\end{align*}

\begin{thm}
    \(f\) es analítica en \(\Omega\) ssi las derivadas parciales existen, son continuas y se tiene CR.
\end{thm}

\begin{proof}
    \(\implies\): Se hizo anteriormente\\
    \(\impliedby\): Se reescribe de la siguiente forma:
    \begin{align*}
        \frac{f(z_0+\Delta z)-f(z_0)}{\Delta z} &=\frac{u(x_0+\Delta x,y_0+\Delta y)-u(x_0,y_0)}{\paren{\Delta x}^2+\paren{\Delta y}^2}\paren{\Delta x-i\Delta y}\\&+i\frac{v(x_0+\Delta x,y_0+\Delta y)-v(x_0,y_0)}{\paren{\Delta x}^2+\paren{\Delta y}^2}\paren{\Delta x-i\Delta y}
    \end{align*}
    Por Taylor, se puede reescribir el límite usando harta matraca, y usando CR se tiene lo pedido.
\end{proof}

\begin{thm}
    Si \(f,g\) analíticas
    \begin{enumerate}
        \item \((f+g)'=f'+g'\)
        \item \((f\cdot g)'=f'g+fg'\)
        \item \((cf)'=cf'\) con \(c\in\set{C}\)
        \item \((f/g)'=\frac{f'g-g'f}{g^2}\) si \(g\neq0\)
    \end{enumerate}
\end{thm}
\begin{thm}
    Si \(f\) es analítica y \(f'(z)=0\) entonces \(f\) es constante
\end{thm}

\begin{proof}
    Si \(f'(z)=0\) las derivadas parciales son cero, por lo que \(u=c_1,v=c_2\implies f=c_1+ic_2\)
\end{proof}

\begin{thm}
    Si \(f\) analítica y \(f(z_0)\neq 0\), \(f\) es localmente invertible en una vecindad de \(z_0\) y \(f^{-1}\) es analítica
    \[
        (f^{-1})'(z_0)=\frac1{f'(f^{-1}(z_0))}
    \]
\end{thm}

\subsection{Mapeos Conformes}
\begin{defn}
    Sean \(v_1,v_2\) vectores de \(\set{R}^2\), entonces \(\cos(\theta)=\frac{\angled{v_1,v_2}}{\norm{v_1}\norm{v_2}}\), donde \(\theta\) es el ángulo entre \(v_1,v_2\). Luego, sean \(\gamma_1,\gamma_2\) curvas diferenciables en \(\set{R}^2\) que se intersectan en el punto \(P\) en \(t_1,t_2\) correspondientemente, entonces el ángulo entre \(\gamma_1,\gamma_2\) en \(P\) esta dado por:
    \[
        \cos\theta=\frac{\angled{\gamma_1'(t_1),\gamma_2'(t_2)}}{\norm{\gamma_1'}\norm{\gamma_2'}}
    \]
\end{defn}

\begin{defn}
    Decimos que \(f:\set{R}^2\rightarrow\set{R}^2\) es conforme si \(\gamma_1,\gamma_2\) son curvas que se intersectan en \(P\) y forman un ángulo \(\theta\), entonces \(f\circ\gamma_1,f\circ\gamma_2\) forman el mismo ángulo \(\theta\) en \(f(P)\). Es decir, \(f\) preserva ángulos. (No necesariamente el largo)
\end{defn}
\end{document}