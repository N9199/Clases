\input{../Templates/Notetaking}

\title{Algebra Abstracta II}
\author{Nicholas Mc-Donnell}
\date{1er semestre 2018}

\pagenumbering{gobble}

\begin{document}
    \maketitle

    \section*{Programa}
    {\raggedleft Profesor: Ricardo Menares}\\
    Email: 
    \begin{enumerate}
        \item Teoría de Cuerpos

        \item Teoría de Galois
    \end{enumerate}

    \section*{Evaluaciones}

    \section*{Bibliografía}
    Algebra, Artin

    \newpage

    \tableofcontents

    \part{Cuerpos}
    \chapter{Anillos}
    \begin{defn}[Anillo]
        $(A,+,\cdot)$ es un \underline{anillo} si:
        \begin{enumerate}
            \item $(A,+)$ es un grupo abeliano

            \item $\cdot$ es conmutativa, asociativa y $1$ es identidad

            \item Propiedad distributiva: $\forall a,b,c\in A: (a+b)\cdot c=a\cdot c+ b\cdot c$
        \end{enumerate}
    \end{defn}

\end{document}