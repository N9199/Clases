\documentclass[11pt]{book}
    \usepackage[spanish]{babel}
    \usepackage[utf8]{inputenc}
    \usepackage[margin=1in]{geometry}          
    \usepackage{graphicx}
    \usepackage{amsthm, amsmath, amssymb}
    \usepackage{mathtools}
    \usepackage{setspace}\onehalfspacing
    \usepackage[loose,nice]{units} 
    \usepackage{enumitem}
    \usepackage{hyperref}
    \hypersetup{
        colorlinks,
        citecolor=black,
        filecolor=black,
        linkcolor=black,
        urlcolor=black
    }
    
    \renewcommand{\d}[1]{\ensuremath{\operatorname{d}\!{#1}}}
    \renewcommand{\vec}[1]{\mathbf{#1}}
    \newcommand{\set}[1]{\mathbb{#1}}
    \newcommand{\func}[5]{#1:#2\xrightarrow[#5]{#4}#3}
    \newcommand{\contr}{\rightarrow\leftarrow}


    \newtheorem{thm}{Teorema}[section]
    \newtheorem{lem}[thm]{Lema}
    \newtheorem{prop}[thm]{Proposición}
    \newtheorem*{cor}{Corolario}

    \theoremstyle{definition}
    \newtheorem{defn}{Definición}[section]
    \newtheorem{obs}{Observación}[section]
    \newtheorem{ejm}[thm]{Ejemplo:}

\title{Algebra Abstracta II}
\author{Nicholas Mc-Donnell}
\date{1er semestre 2018}

\pagenumbering{gobble}

\begin{document}
    \maketitle

    \section*{Programa}
    {\raggedleft Profesor: Ricardo Menares}\\
    Email: 
    \begin{enumerate}
        \item Teoría de Cuerpos

        \item Teoría de Galois
    \end{enumerate}

    \section*{Evaluaciones}

    \section*{Bibliografía}
    Algebra, Artin

    \newpage

    \tableofcontents

    \part{Cuerpos}
    \chapter{Anillos}
    \begin{defn}[Anillo]
        $(A,+,\cdot)$ es un \underline{anillo} si:
        \begin{enumerate}
            \item $(A,+)$ es un grupo abeliano

            \item $\cdot$ es conmutativa, asociativa y $1$ es identidad

            \item Propiedad distributiva: $\forall a,b,c\in A: (a+b)\cdot c=a\cdot c+ b\cdot c$
        \end{enumerate}
    \end{defn}

    \begin{ejm}
        \[\set{Z},\set{Q},\set{R}\]
        \[\set{Z}/m\set{Z},m\in\set{Z}\]
        $(\set{N},+,\cdot)$ no es anillo
    \end{ejm}

    \begin{obs}
        $A=\{0\}$ es anillo trivial ($0\equiv 1$)
    \end{obs}

    \begin{defn}[Divisor de cero]
        $a\in A$ es \underline{divisor de cero} si
        \begin{enumerate}
            \item $a\neq 0$

            \item $\exists b\in A: b\neq0 \wedge a\cdot b=0$
        \end{enumerate}
    \end{defn}

    \begin{ejm}
        En $\set{Z}/4\set{Z}$, $\bar{2}\neq\bar{0}$ y $\bar{2}\cdot\bar{2}=\bar{4}=\bar{0}$
    \end{ejm}

    \begin{defn}[Unidad]
        $u\in A$ \underline{unidad} si $\exists v\in A: u\cdot v=1$
        \[A^\star=\{\text{unidades de }A\}\]
        $(A^\star,\cdot)$ es grupo abeliano
    \end{defn}

    \begin{ejm}
        \hfill
        \begin{itemize}
            \item $\set{Z}^\star=\{-1,1\}$

            \item $\set{R}^\star=\set{R}\setminus\{0\}$

            \item $\set{Z}[i]^\star=\{-1,1,-i,i\}$
        \end{itemize}
    \end{ejm}

    \begin{eje}
        $m\in\set{Z},m\neq 0$\\
        $(\set{Z}/m\set{Z})^\star=\{\bar{b}:(b,m)=1\}$
    \end{eje}

    \chapter{Cuerpos}
    \begin{defn}[Cuerpo]
        Un \underline{cuerpo} es un anillo tal que $A^\star=A\setminus\{0\}$.
    \end{defn}

    \begin{ejm}
        $\set{R},\set{Q}$ son cuerpos\\
        $\set{Z},\set{Z}[i]$ \underline{no} son cuerpos\\
        $\set{Q}[i]$ es cuerpo
    \end{ejm}

    \begin{defn}
        Sea $A$ anillo, $p(x)=a_0+a_1x+...+a_nx^n\in A[x]$ es un polinomio con coeficientes en $A$
    \end{defn}

    \begin{prop}
        $A[x]=\{\text{polinomios con coeficientes en }A\}$ es un anillo con las operaciones usuales.
    \end{prop}

    \begin{obs}
        Si $A\neq\{0\}$, entonces $A[x]$ no es cuerpo.\\
        En efecto, $x\neq 0$ y no tiene inverso pues si tuviera $p(x)=\sum^n_{i=0}a_ix^i$ tal que
        \[x\cdot(p(x))=1\]
        \[x(a_0+a_1x+...+a_nx^n)=1\]
        \[0+a_0x+a_1x^2+...+a_nx^{n+1}=1\]
        Al ser igualdad de polinomios $0=1,\forall i:a_i=0\contr$
    \end{obs}

    \begin{eje}
        \begin{enumerate}
            \item Si $A$ no tiene divisores de cero $(A[x])^\star=A^\star$

            \item Encontrar un anillo $A$ tal que $(A[x])^\star\neq A^\star$
        \end{enumerate}
    \end{eje}

    \section{Morfismos}
    $A,B$ anillos $\func{f}{A}{B}{}{}$ se dice morfismo de anillos si:
    \begin{enumerate}
        \item $f(0_A)=0_B, f(1_A)=f(1_B)$

        \item $\forall a,b\in A$
    \end{enumerate}
    %Continuacion

    %Ideales maximos y principales

    \chapter{Anillos Euclideanos}
    Sea $A$ un anillo. Una funcion Euclideana $\func{f}{A\setminus\{0\}}{\set{Z}}{}{}$ es tal que $\forall a,b\in A\setminus\{0\}$, se cumple $f(ab)\geq f(a)$.\\
    $A$ se dice Euclideano si la funcion $f$ ademas satisface:
    \[\forall a,b \in A:b\neq 0\]
    \[\exists q,r\in A:q=bq+r\]
    Donde, o bien $r=0$, o bien $f(r)<f(b)$\\
    (Algoritmo de la division)
    \begin{ejm}
        \hfill
        \begin{enumerate}
            \item $A=\set{Z}; f(x)=|x|$

            \item $A=F[x]$, $F$ cuerpo\\
            $f(p(x))=$ grado de $p(x)$\\
            (Asumiendo $p(x)\neq 0$)
        \end{enumerate}
    \end{ejm}

    \begin{obs}
        Todo anillo Euclideano es un DIP
        \begin{proof}
            Dada $I\subset A$, con $A$ Euclideano, elegimos $a\in I$ tal que $f(a)$ cumple $f(a)=\min\Ima(f)$\\
            Dado $b\in I$, escribimos $b=a\cdot q+r$\\
            Notar que $r=b-a\cdot q\in I$ y si $r\neq 0$, se tiene $f(b)<f(a)$
            \[\contr\]
        \end{proof}
    \end{obs}

\end{document}