\documentclass[11pt]{book}
    \usepackage[spanish]{babel}
    \usepackage[utf8]{inputenc}
    \usepackage[margin=1in]{geometry}          
    \usepackage{graphicx}
    \usepackage{amsthm, amsmath, amssymb}
    \usepackage{mathtools}
    \usepackage{setspace}\onehalfspacing
    \usepackage[loose,nice]{units} 
    \usepackage{enumitem}
    \usepackage{hyperref}
    \hypersetup{
        colorlinks,
        citecolor=black,
        filecolor=black,
        linkcolor=black,
        urlcolor=black
    }
    
    \renewcommand{\d}[1]{\ensuremath{\operatorname{d}\!{#1}}}
    \renewcommand{\vec}[1]{\mathbf{#1}}
    \newcommand{\set}[1]{\mathbb{#1}}
    \newcommand{\func}[5]{#1:#2\xrightarrow[#5]{#4}#3}
    \newcommand{\contr}{\rightarrow\leftarrow}


    \newtheorem{thm}{Teorema}[section]
    \newtheorem{lem}[thm]{Lema}
    \newtheorem{prop}[thm]{Proposición}
    \newtheorem*{cor}{Corolario}

    \theoremstyle{definition}
    \newtheorem{defn}{Definición}[section]
    \newtheorem{obs}{Observación}[section]
    \newtheorem{ejm}[thm]{Ejemplo:}

\title{Algebra Abstracta II}
\author{Nicholas Mc-Donnell}
\date{1er semestre 2018}

\pagenumbering{gobble}

\begin{document}
    \maketitle

    \section*{Programa}
    {\raggedleft Profesor: Ricardo Menares}\\
    Email: 
    \begin{enumerate}
        \item Teoría de Cuerpos

        \item Teoría de Galois
    \end{enumerate}

    \section*{Evaluaciones}

    \section*{Bibliografía}
    Algebra, Artin

    \newpage

    \tableofcontents

    \part{Cuerpos}
    \chapter{Anillos}
    \begin{defn}[Anillo]
        $(A,+,\cdot)$ es un \underline{anillo} si:
        \begin{enumerate}
            \item $(A,+)$ es un grupo abeliano

            \item $\cdot$ es conmutativa, asociativa y $1$ es identidad

            \item Propiedad distributiva: $\forall a,b,c\in A: (a+b)\cdot c=a\cdot c+ b\cdot c$
        \end{enumerate}
    \end{defn}

\end{document}