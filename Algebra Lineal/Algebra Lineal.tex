\documentclass[11pt]{book}
\usepackage[spanish]{babel}
\usepackage[utf8]{inputenc}
\usepackage[margin=1in]{geometry}          
\usepackage{graphicx}
\usepackage{amsthm, amsmath, amssymb}
\usepackage{mathtools}
\usepackage{setspace}\onehalfspacing
\usepackage[loose,nice]{units} 
\usepackage{enumitem}
\usepackage{hyperref}
\hypersetup{
    colorlinks,
    citecolor=black,
    filecolor=black,
    linkcolor=black,
    urlcolor=black
}

\title{Algebra Lineal}
\author{Nicholas Mc-Donnell}
\date{2do semestre 2017}

\renewcommand{\d}[1]{\ensuremath{\operatorname{d}\!{#1}}}
\renewcommand{\vec}[1]{\mathbf{#1}}
\newcommand{\set}[1]{\mathbb{#1}}
\newcommand{\func}[5]{#1:#2\xrightarrow[#5]{#4}#3}
\newcommand{\contr}{\rightarrow\leftarrow}

\DeclareMathOperator{\Ima}{Im}

\newtheorem{thm}{Teorema}[section]
\newtheorem{lem}[thm]{Lema}
\newtheorem{prop}[thm]{Proposición}
\newtheorem*{cor}{Corolario}
\newtheorem*{conj}{Conjetura}

\theoremstyle{definition}
\newtheorem{defn}{Definición}[section]
\newtheorem{obs}{Observación}[section]

\begin{document}
\maketitle
\pagenumbering{gobble}

\tableofcontents
\pagenumbering{arabic}
\chapter{Espacios Vectoriales}
\section{Cuerpos}
Si $\set{F}=\set{R}$ o $\set{C}$, recordemos que se tienen las siguientes propiedades:
\begin{enumerate}
	\item Conmutatividad de la suma
	\[a+y=y+x\quad\forall x,y\in\set{F}\]

	\item Asociatividad de la suma
	\[(x+y)+z=x+(y+z)\quad\forall x,y,z\in\set{F}\]

	\item Existe un único elemento neutro para la suma, tal que:
	\[x+0=x\quad\forall x\in\set{F}\]

	\item Para todo $x\in\set{F}$ existe un único inverso para la suma, tal que:
	\[x+y=0\quad (-x=y)\]

	\item Conmutatividad del producto
	\[xy=yx\quad\forall x,y\in\set{F}\]

	\item Asociatividad del producto
	\[(xy)z=x(yz)\quad\forall x,y\in\set{F}\]

	\item Existe único neutro para el producto, el 1, tal que:
	\[x\cdot 1=x\quad\forall x\in\set{F}\]

	\item Todo elemento $x\neq 0$ posee inverso multiplicativo único, tal que:
	\[x\cdot y=1\quad (x^{-1}=y)\]

	\item Distributividad de $\cdot$ con respecto a $+$
	\[x\cdot (y+z)=xy+xz\quad\forall x,y,z\in\set{F}\]
\end{enumerate}
\begin{defn}[Cuerpo]
	Un cuerpo es un conjunto dotado de dos operaciones $(\set{F},+,\cdot)$, que cumplen lo siguiente:
	\[\func{+}{\set{F}\times\set{F}}{\set{F}}{}{}\]
	\[\func{\cdot}{\set{F}\times\set{F}}{\set{F}}{}{}\]
	\begin{itemize}
		\item $(\set{F},+)$ es grupo abeliano

		\item $(\set{F}\setminus\{0\},\cdot)$ es un grupo abeliano ($0$ es el neutro aditivo)

		\item $a,b,c\in\set{F}\implies a(b+c)=ab+ac$
	\end{itemize}
\end{defn}
Ejemplos:
\begin{enumerate}[label=(\alph*)]
	\item $\set{N}$ con $+$ y $\cdot$ usuales. No, no hay inverso aditivo.
	
	\item $\set{Z}$ con $+$ y $\cdot$ usuales. No, no hay inverso multiplicativo.

	\item $\set{Q}$ con $+$ y $\cdot$ usuales. Si.

	\item $\set{R}$ con $+$ y $\cdot$ usuales. Si.

	\item $\set{C}$ con $+$ y $\cdot$ usuales. Si.

	\item $\set{F}=\{0,1\}$ con $+$ y $\cdot$ definida de la siguiente forma:\\
	\begin{center}
		\begin{tabular}{ c | c  c }
			+ & 0 & 1 \\ \hline
			0 & 0 & 1 \\
			1 & 1 & 0
		\end{tabular}
		\ \ \ \ \ 
		\begin{tabular}{ c | c  c }
			$\cdot$ & 0 & 1\\ \hline
			0 & 0 & 0 \\
			1 & 0 & 1
		\end{tabular}
	\end{center}
\end{enumerate}
\begin{defn}[Subcuerpo]
	$\set{L}\subseteq\set{F}$ es un subcuerpo si $(\set{L},+,\cdot)$ es un cuerpo (donde $+$ y $\cdot$ vienen de $(\set{F},+,\cdot)$)
\end{defn}
\begin{prop}
	Sea $(\set{F},+,\cdot)$ un cuerpo $\set{L}$ es subcuerpo $\iff$
	\begin{enumerate}[label = (\alph*)]
		\item $\{0,1\}\subseteq\set{L}$

		\item $\set{L}$ es cerrado para $+$, y $\forall x\in\set{L}\implies \exists -x\in\set{L}$

		\item $\set{L}$ es cerrado para $\cdot$, y $\forall x\in\set{L}\setminus\{0\}\implies \exists x^{-1}\in\set{L}$
	\end{enumerate}
	\begin{proof}
		$\implies$ trivial\\
		$\impliedby$
		\begin{enumerate}
			\item La propiedad se hereda de $(\set{F},+,\cdot)$

			\item Idem

			\item Por (a)

			\item Por (b)

			\item La propiedad se hereda de $(\set{F},+,\cdot)$

			\item Idem

			\item Por (a)

			\item Por (c)

			\item La propiedad se hereda de $(\set{F},+,\cdot)$
		\end{enumerate}
	\end{proof}
\end{prop}
Ejemplos:
\begin{enumerate}
	\item $\set{F}$ es un subcuerpo de $\set{F}$

	\item $\set{R}$ es un subcuerpo de $\set{C}$

	\item $\set{Q}$ es un subcuerpo de $\set{R}$

	\item $\{a+b\sqrt{2}:a,b\in\set{Q}\}$ es subcuerpo de $\set{R}$
\end{enumerate}
\begin{lem}
	Todo subcuerpo de $\set{R}$ contiene a $\set{Q}$
	\begin{proof}
		Sea $\set{L}$ un subcuerpo de $\set{R}$
		\[(a)\implies \{0,1\}\subseteq\set{L}\]
		\[1\in\set{L}\implies\set{N}\subseteq\set{L}\quad (b)\]
		\[\implies \set{Z}\subseteq\set{L}\quad (b)\]
		\[\implies \set{Q}\subseteq\set{L}\quad (c)\]
		\underline{Observación:} El único subcuerpo de $\set{Q}$ es $\set{Q}$ (ver dem. anterior)
	\end{proof}
\end{lem}
\underline{Ejercicio:} Existen infinitos subcuerpos de $\set{R}$
Recursivamente: sean $p_0=2,p_1=3,p_2=5,...$ los números primos. Definamos:
\[\set{L}_0\]
\section{Espacios Vectoriales}
\begin{defn}[Espacio Vectorial]
	
\end{defn}
\section{Subespacios generados}
\subsection{Combinaciones lineales}
\begin{thm}[*]
	Sea $V$ espacio vectorial generado por un conjunto finito m de vectores. Entonces cualquier

	\begin{proof}
		Sean $u_1,u_2,...,u_n\in V$ con $n>m$. Por contradicción, suponiendo que son linealmente independientes.\\
		Sean $<v_1,...,v_m>=V$ (por hipótesis)
		\[
		\implies u_1=\sum^m_{i=1}\lambda_i^{(1)} v_i
		\]
		Claramente $(\lambda_1^{(1)},...,\lambda_m^{(1)})\neq 0$ (de lo contrario $u_1=0$)\\
		Sin perder generalidad, $\lambda_1^{(1)}\neq 0$ y por el Lema:
		\[
		<u_1,v_2,...,v_m>=V
		\]
		Ahora, existan $(\lambda_1^{(2)},...,\lambda_m^{(2)}\neq(0,...,0)$ tales que:
		\[
		u_2=\lambda_1^{(2)} u_1+\sum^m_{i=2}\lambda_i^{(2)}v_i
		\]
		Más. aun, $(\lambda_2^{(2)},...,\lambda_m^{(2)}\neq(0,...,0)$\\
		De lo contrario:
		\[
		u_2=\lambda_1^{(2)}u_1\\
		0=\lambda_1^{(2)}u_1-u_2\\
		\rightarrow\leftarrow (u_1,...,u_n\textrm{ son linealmente independientes})
		\]
		En conclusión, sin perdida de generalidad  $\lambda_2^{(2)}\neq 0$\\
		Por el Lema:
		\[
		<u_1,u_2,v_3...,v_m>=V
		\]
		Iterando el argumento, se tiene que:
		\[
		<u_1,...,u_m>=V
		\]
		Notemos que $u_{m+1}\in<u_1,...,u_m>$
		\[
		0\neq u_{m+1}=\sum^m_{i=1}\mu_i u_i
		\rightarrow\leftarrow
		\]		
	\end{proof}
\end{thm}

\begin{defn}[Base,Dimensión finita]
Una base B de un espacio vectorial es un conjunto $B\subseteq V$ tal que:
\begin{enumerate}
	\item $B$ es linealmente independiente
	\item $<B>=V$
\end{enumerate}
Un espacio vectorial $V$ se dice finito-dimensional si existe un conjunto $S\subseteq V,\|S\|<\infty$ tal que $<S>=V$.
\end{defn}
\begin{cor}
Si $V$ es finito-dimensional, todas las bases de $V$ son finitas y tienen la misma cardinalidad.
\begin{proof}
Sean $B_1$ y $B_2$ bases de V.\\
Por Teo (*), $\|B_1\|,\|B_2\|\leq m$, donde $m$ es el tamaño de $S$ tal que $<S>=V$ y $\|S\|<\infty$.\\
Como $B_1$ es base, $<B_1>=V$, y como $B_2$ es linealmente independiente:
\[
Teo(*)\implies \|B_2\|\leq\|B_1\|
\]
Como $B_2$ es base.
\end{proof}
\end{cor}

\begin{defn}[Dimensión]
Si V es un espacio vectorial finito-dimensional definimos su dimensión, $\dim V$, como el cardinal de una base cualquiera de $V$. Si $V$ no es finito-dimensional $dimV=+\infty$
\end{defn}
Ejemplos:
\begin{enumerate}
	\item $V=Sim^2(\mathbb{R})=\{A\in\mathbb{R}^{2\times 2}:A=A^T\}$
	\[
	abc
	\]
	
	\item $\dim(\mathbb{R}^{n\times n})=n^2$
	
	\item $\dim(Antisim^m(\mathbb{R}))=\frac{n\cdot(n-1)}{2}$ y $\dim(Sim^m(\mathbb{R}))=\frac{n\cdot (n+1)}{2}$
	
	\item$P_n(\mathbb{C}$\\
	$\{1,x,x^2,...,x^n\}$ es base
	\begin{enumerate}
		\item $<\{1,x,x^2,...,x^n\}>=P_n(\mathbb{C})$\\
		$p\in P_n(\mathbb{C})$\\
		$\implies p(x)=a_0\cdot 1+a_1\cdot x+,,,+a_nx^n\in<\{1,x,x^2,...,x^n\}>$
		
		\item $\{1,x,...,x^n\}$ es linealmente independiente.\\
		Por contradicción supongamos $(a_0,a_1,...,a_n)\neq (0,0,...,0)$\\
		$0=a_0\cdot 1+a_1\cdot x+...+a_n\cdot x^n$ Con igualdad de funciones.\\
		$\iff (\forall x\in \mathbb{C}) 0=a_0\cdot 1+a_1\cdot x+...+a_n\cdot x^n$\\
		Recuerdo (TFA): Todo polinomio complejo de grado $\geq 1$ posee una raíz compleja.\\
		$\implies p(x)=a_0\cdot 1+a_1\cdot x+...+a_n\cdot x^n=(x-z_1)\cdot a_0'\cdot 1+a'_1\cdot x+...+a'_{n-1}\cdot x^{n-1}$\\
		$\implies p(x)=(x-z_1)\cdot (x-z_2)\cdot...(x-z_k)\cdot A$ donde $k=gr(p)$, y $A\neq 0$.\\
		Tomando $z'\neq z_1,z_2,...,z_k$, tenemos: $0=(z'-z_1)\cdot (z'-z_2)\cdot...(z'-z_k)\cdot A$ Pero multiplicar cosas distintas de $0$ no da $0$.
		\[
		\rightarrow\leftarrow
		\]
	\end{enumerate}
	$\dim(P_n(\mathbb{C}))=n+1$
\end{enumerate}

\begin{obs}
$\{0\}, \dim\{0\}=0$, notando que base es $\emptyset$, tenemos que $<\emptyset>=\{0\}$
\end{obs}
\begin{lem}
Sea $S\subseteq , S=\{v_1,...,v_n\}$ conjunto linealmente independiente.\\
$v\notin<v_1,...,v_n>\implies\{v,v_1,...,v_n\}$ es linealmente independiente.
\begin{proof}
Ejercicio
\end{proof}
\end{lem}

\begin{thm}
Sea $V$ espacio vectorial finito dimensional, entonces:
\begin{enumerate}
	\item Todo conjunto linealmente extiende a una base
	
	\item Todo conjunto generado contiene una base
\end{enumerate}

\end{thm}
Todo espacio vectorial $V\neq\{0\}$ finito dimensional posee una base.
\begin{proof}
\begin{enumerate}[label=\alph*)]
	\item Sea $v\in V\setminus\{0\}$. Entonces $\{V\}$ es linealmente independiente.\\
	Por Teo, estamos listos.
	
	Dem(Teo):\\
	$V$ es finito dimensional $\implies\exists\{v_1,...,v_n\}$ que genera $V$.\\
	Sea $S=\{v_1,...,v_n\}$ conjunto linealmente independente.\\
	Dos casos:
	\begin{enumerate}
		\item Si $<S>=V$, entonces $S$ es base
		
		\item Si $<S>\subset V$, entonces existe $v\notin<S>$, y opr el Lema, $S\cup\{V\}$ es linealmente independiente.
	\end{enumerate}
	Inductivamente, o bien eventualmente $1$, o iteramos $2$.
	
	Sin embargo, $2$ no puede ocurrir infinitas veces.\\
	Teo(*)$\implies$ todo conjunto linealmente independiente posee cardinalidad $\leq n$.
	
	\item Sea $S$, tal que $<S>=V$.\\
	Consideremos, $d=\max\{|S'|:S'\subseteq S, S'\textrm{ es linealmente independiente}\}$\\
	Notemos que $d\leq n<+\infty$, por el Teo(*).\\
	Como el máximo se alcanza, existe $S'$ tal que $|S'|=d$ y $S'$ es linealmente independiente.\\
	Por contradicción, supongamos que $S'$ no es base.
	\[
	<S'>\subset V=<S>
	\]
	Luego, $S\setminus<S'>\neq\emptyset$. De lo contrario $<S>\setminus<S'>\neq\emptyset$\\
	$\implies S\subseteq <S'>\implies <S>\subset<S'>$
	$\implies <S'>=V$\\
	Finalmente, existe $v\in S\setminus<S'>$, y por el Lema, $S'\cup\{v\}$ es linealmente independiente y $|S'\cup\{v\}|=d+1$
	\[
	\rightarrow\leftarrow
	\]
\end{enumerate}
\end{proof}
\begin{cor}
	Sea $W\subset V$ subespacio propio ($W\neq V$) con $V$ finito dimensional.
	Entonces, $\dim W< \dim V$
	\begin{proof}
	\begin{itemize}
		\item Si $\dim W=0\implies \dim W=0<\dim V$.
		
		\item Si $\dim W\geq 1$ entonces, por el corolario, $W$ tiene una base.
		\[
		B_W=\{w_1,...,w_m\}\quad\textrm{es base de } W
		\]
		Como $W$ es subespacio propio, existe $v\in V\setminus<B_W>$\\
		$\implies B_W\cup\{v\}$ es linealmente independiente y $\subset V$.\\
		Por Teo. a), $B_W\cup\{v\}$ se extiende cin una base $B_V,|B_V|\geq|B_W|+1$.
		\[
		\dim W=|B_W|<|B_W|+1\leq |B_V|=\dim V
		\]
	\end{itemize}
	\end{proof}	
\end{cor}


\begin{prop}
Sean $U,W$ subespacios vectoriales de un espacio vectorial finito dimensional $V$. Entonces:
\[
\dim (U+W)=\dim U+\dim W-\dim (U\cap W)
\]
\begin{proof}
Si $U\cap W=\{0\}$, trivial\\
Si $U\cap W\neq\{0\}$, entonces sea $B_{U\cap W}$ base de $U\cap W$.
\[
B_{U\cap W}=\{v_1,...v_m\}
\]
Base de U: Sea $B_U=\{v_1,...,v_m,u_1,...,u_p\}$\\
Base de W: Sea $B_W=\{v_1,...v_m,w_1,...,w_r\}$\\
Ambos existen por Teo.a).\\
Afirmación: $B_U\cap B_W=\{v_1,...,v_m,u_1,...,u_p,w_1,...,w_r\}$ es base de $U+W$.
\begin{enumerate}
	\item $<B_U\cap B_W>=U+W$\\
	$v\in u+w,u\in U,w\in W$
	\[
	\implies v=\sum^{p}_{i=1}\lambda_i u_i+\sum^{m}_{j=1}\alpha_j v_j+\sum^{r}_{k=1}\delta_k w_k+\sum^{m}_{j=1}\beta_j v_j\in<B_U\cup B_W>
	\]
	
	\item $B_U\cup B_W$ es linealmente independiente:
	\[
	0=\sum^{m}_{i=1}\lambda_i v_i+\sum^{r}_{j=1}\mu_j w_j+\sum^{p}_{k=1}\nu_k u_k
	\]
	\[
	-\sum^{p}_{k=1}\nu_k u_k=\sum^{m}_{i=1}\lambda_i v_i+\sum^{r}_{j=1}\mu_j w_j \in U\cap W
	\]
	\[
	\implies 0=\sum^{m}_{i=1}\lambda_i v_i+\sum^{r}_{j=1}\mu_j w_j\implies \nu_k,\lambda_i=0\,\forall k,i
	\]
\end{enumerate}
\[\dim(U+W)=\dim U+\dim W-\dim(U\cap W)\]
\[\dim(U\cap W)=|b_{U\cap W}|\]
\[\dim U=|B_U|\]	
\[\dim W=|B_W|\]	
\[\dim(U+W)=|B_U\cup B_W|\qedhere\]
\end{proof}
\end{prop}
\section{Transformaciones lineales}
\begin{defn}[Transformación lineal]
Sean $V$ y $W$ espacios vectoriales sobre un cuerpo común $\mathbb{F}$.
Una función $T:V\rightarrow W$ es transformación lineal si 
\[T(\lambda u+v)=\lambda T(u)+T(v)\quad \forall u,v\in V \forall\lambda\in\mathbb{F}
\]
\end{defn}
Ejemplos:
\begin{enumerate}
	\item $V=\mathbb{F}^n, W=\mathbb{F}^m, T:\mathbb{F}^n\rightarrow\mathbb{F}^m$ y $A\in\mathbb{F}^{m\times n}\quad v\mapsto Av$\\
	$T$ es una transformación lineal
	
	\item $V=W=C^\infty(\mathbb{R}=\{f:\mathbb{R}\rightarrow\mathbb{R}: f\textrm{ es infinitamente diferenciable}\}$\\
	$V=C^1(\mathbb{R}), W=C^0(\mathbb{R})$\\
	$T:V\rightarrow W$\\
	$f\mapsto\frac{\d f}{\d x}$\\
	Por álgebra de funciones diferenciables si $f,g\in V$ y $\lambda\in\mathbb{R}$
	\[
	\frac{\d (\lambda f+g)}{\d x}=\lambda\frac{\d f}{\d x}+\frac{\d g}{\d x}
	\]
	\[
	T(\lambda u+v)=\lambda T(u)+T(v)
	\]
	
	\item $\mathbb{R}[x]$ Es un espacio vectorial\\
	$T:\mathbb{R}[x]\rightarrow \mathbb{R}[x]$\\
	$p\mapsto\frac{\d p}{\d x}$
	
	\item $V=C^0([0,1]), W=\mathbb{R}$\\
	$T:V\rightarrow W$\\
	$f\rightarrow\mapsto\int^1_0f(x)\d x$
\end{enumerate}
\begin{thm}
Sea $V$ un espacio finito dimensional y $\{v_1,...,v_n\}$ es base de $V$.\\
Sea $W$ un espacio vectorial y consideramos vectores $w_1,...w_n\in W$\\
Entonces $\exists !T:V\rightarrow W: T(v_i)=w_i\forall i=1,...,n$
\begin{proof}
\begin{itemize}
	\item Existencia: Si $v\in V$ entonces:
	\[(*)\, v=\sum^n_{i=1}\lambda_i v_i\quad \exists \lambda_i\]
	Definimos:
	\[T(v)=\sum^n_{i=1}\lambda_i T(v_i)\]
	$T:V\rightarrow W$ es una función porque $\forall v\in V$ la descomposición (*) es única.\\
	Además, es lineal. Si $v,u\in V$ y $\lambda\in\mathbb{F}$
	\[v=\sum_i \lambda_i v_i\]
	\[u=\sum_i\mu_i v_i\]
	\[T(\lambda v+u)=T(\sum_i(\lambda\lambda_i+\mu_i)v_i)\]
	\[T(\lambda v+u)=\sum_i(\lambda\lambda_i+\mu_i)T(v_i)\]
	\[T(\lambda v+u)=\lambda\sum_i\lambda_i T(v_i)+\sum_i\mu_i T(v_i)\]
	\[T(\lambda v+u)=\lambda T(v)+T(u)\]
	\item Unicidad: Sean $T,T'$ transformaciones lineales tales que:
	\[T(v_i)=T'(v_i)=w_i\]
	Sea $v\in V$, entonces:
	\[v=\sum^n_{i=1}\lambda_i v_i\quad \exists \lambda_i\]
	\[T(v)=\sum^n_{i=1}\lambda_i T(v_i)=\sum^n_{i=1}\lambda T'(v_i)\]
	\[T(v)=T'(v)\implies T=T'\]
\end{itemize}
\end{proof}
\end{thm}
Propiedades:\\
Si $T:V\rightarrow W$ transformaciones lineales, entonces:
\begin{enumerate}
	\item $T(0)=0$
	
	\item $T(\sum^n_{i=1}\lambda_iT(v_i)$
\end{enumerate}
\begin{proof}

\end{proof}

La importancia de este teorema se relaciona con el hecho de poder definir transformaciones lineales sólo a través de como actuan sobre una base. Esto se relaciona con la noción de \underline{matriz representante}.
\begin{defn}
Si $T:V\rightarrow W$ transformación lineal, definimos:
\begin{itemize}
	\item Núcleo(Kernel): $\ker(T)=\{v\in V:T(v)=0\}$, subespacio vectorial de $V$
	
	\item Imagen(Rango): $T(V)=\{w\in W:\exists v\in V\, T(v)=w\}$, subespacio vectorial de $W$
\end{itemize}
\end{defn}
\begin{thm}[Núcleo-Imagen]
Sean $V,W$ espacios vectoriales sobre $\mathbb{F}$ y $T:V\rightarrow W$ transformación lineal. Entonces, si $V$ es finito dimensional:
\[
\dim(V)=\dim(\ker(T))+\dim(T(V))
\]
\begin{proof}
Como $\ker T$ es subespacio vectorial de $V$ entonces es finito dimensional y por ende posee una base:
\[
\{v_1,...,v_k\} \quad\textrm{ es base de} \ker T
\]
Por un teorema podemos extender a una base de $V$\\
$\{v_1,...,v_k,v_{k+1},...,v_u\}$ es base de $V$\\
Sea ahora $w\in\Im(T)\implies w=T(v)$\\
Entonces,
\[
w=T(v)=T(\sum^n_{i=1}\quad\exists\lambda_i\in\mathbb{F}
\]
\[
w=\sum^n+{i=k+1}\lambda_iT(v_i)
\]
\[
\in<\{T(v_{k+1}),...,T(v_n)\}>
\]
Queremos probar ahora que $(T(v_i))^n_{i=k+1}$ son linealmente independiente. Supongamos $\exists\lambda_{k+1},...,\lambda_{n}$ tal que:
\[
\sum^n_{i=k+1}\lambda_iT(v_I)=0
\]
\[
T(\sum^n_{i=k+1}\lambda_iv_i)=0
\]
\[
\implies \in Ker(T)
\]
$\exists \lambda_1,...,\lambda_k$ tal que
\[
\sum^n_{i=k+1}(-\lambda_i)v_i+\sum^k_{j=1}\lambda_jv_j=0
\]
Como $\{v_1,...,v_n\}$ son linealmente independientes tenemos que:
\[
\lambda_1=\lambda_2=...=\lambda_n=0
\]
Finalmente:
\[
\dim V=n,\dim \ker T=k
\]
\[
\dim\Im T=n-k\qedhere
\]
\end{proof}
\end{thm}
\subsection{Algebra de Transformaciones Lineales}
Def[Espacio de transformaciones lineales]:\\
Sean $V$ y $W$ espacios vectoriales sobre $\mathbb{F}$. Definimos
\[
\mathcal{L}(V,W)=\{T:V\rightarrow W:T\textrm{ es transformación lineal}\}
\]
Dadas $T_1,T_2\in\mathcal{L}(V,W)$ y $\lambda\in\mathbb{F}$, definimos
\[
T_1+T_2:V\rightarrow W
\]
\[
v\mapsto T_1v+T_2v
\]
\[
\lambda T_1:V\rightarrow W
\]
\[
v\mapsto \lambda T_1v
\]
\subsubsection{Teorema}
$\mathcal{L}(V,W)$ dotado de $+$ y $\cdot$ es un espacio vectorial sobre $\mathbb{F}$\\
Dem: Primero probar que $T_1+T_2\in\mathcal{L}(V,W)$ y $\lambda T_1\in\mathcal{L}(V,W)$\\
$T_1+T_2$ es transformación lineal.\\
$(T_1+T_2)(\lambda v+w)=\lambda(T_1+T_2)(v)+(T_1+T_2)(w)$
$(T_1+T_2)(\lambda v+w)=T_1(\lambda v+w)+T_2(\lambda v+w)$
$(T_1+T_2)(\lambda v+w)=\lambda(T_1+T_2)(v)+(T_1+T_2)(w)$
Similarmente:
\[
(\lambda T_1)(\mu v+w)=\mu(\lambda T_1)(v)+(\lambda T_1)(w)
\]
Y el resto de las propiedades se dejan propuestas como ejercicio.
\subsubsection{Teorema}
Sean $V,W$ espacios vectoriales sobre $\mathbb{F}$, $\dim V=n$ y $\dim W=m$. Luego, $\mathcal{L}(V,W)$ es un espacio finito dimensional y $\dim\mathcal{L}(V,W)=m\cdot n$\\
Dem: Sean $V=<\{v_1,...,0_i,...,v_n\}>$, $W=<\{w_1,...,0_j,...,w_n\}>$, bases respectivamente.\\
Dados $1\leq p\leq m, 1\leq q\leq n$, definimos $E^{p,q}\in\mathcal{L}(V,W)$ como única transformación lineal que satisface:
\[
E^{p,q}(v_j)=
\begin{dcases}
	w_p\quad\textrm{si }j=q\\
	0\quad\textrm{si } j\neq q
\end{dcases}
\]
\[
E^{p,q}\in\mathcal{L}(V,W)
\]
$E^{p,q}$ esta bien definida por el primer teorema de la sección
\[
\mathcal{B}_L=\{E^{p,q}:1\leq p\leq n,1\leq q\leq m\}
\]
Afirmación: $\mathcal{B}_L$ es base de $\mathcal{L}(V,W)$
\[
<\mathcal{B}_L>=\mathcal{L}:\textrm{Sea }T\in\mathcal{L}(V,W)
\]
Dado $1\leq j\leq n$, existen $\lambda_1,...,\lambda_m\in\mathbb{F}$
\[
T(v_j)=\sum^m_{i=1}\lambda_{i,j}w_i
\]
Dado $v\in V$, existen $\mu_1,...\mu_n\in\mathbb{F}$
\[
v=\sum^n_{j=1}\mu_jv_j
\]
\[
T(v)=\sum^n_{j=1}\mu_jT(v_j)=\sum^n_{j=1}\mu_j(\sum^m_{i=1}\lambda_iw_i)
\]
\[
T(v)=\sum^n_{j=1}\sum^m_{i=1}\mu_j\lambda_{i,j}E^{i,j}(v_j)
\]
\[
T(v)=\sum_i\sum_j\lambda_{i,j}\mu_jE^{i,j}(v_j)
\]
\[
T(v)=\sum_i\left(\sum_j\lambda_{i,j}E^{i,j}\left(\sum_k\mu_kv_k\right)\right)
\]
\[
T(v)=\sum_i\sum_j\lambda_{i,j}E^{i,j}(v)
\]
$\mathcal{B}_L$ es linealmente independiente: Supongamos que:
\[
\sum_i\sum_j\lambda_{i,j}E^{i,j}=0
\]
Dado $1\leq k\leq n$
\[
\sum_i\sum_j\lambda_{i,j}E^{i,j}(v_k)=\vec{0}
\]
\[
\iff \sum_i\lambda_{i,k}w_i=\vec{0}
\]
\[
\iff \lambda_{1,k}=\lambda_{2,k}=...=\lambda_{m,k}=0\quad \forall k=1,...,n
\]
\[
\dim\mathcal{L}(V,W)=|\mathcal{B}|=m\cdot n
\]
\subsection{Matriz representante}

El Teo anterior permite identificar transformaciones lineales entre espacios finito-dimensionales y matrices $m\times n$ (donde $m=\dim W,n=\dim V$).

Sea $T\in\mathcal{L}(V,W),\mathcal{B}_V=\{v_1,...,v_n\}$ base de $V,\mathcal{B}_W=\{w_1,...,w_m\}$ base de $W$.

Definimos la matriz representante de $T$ con respecto a las bases $\mathcal{B}_V$ y $\mathcal{B}_W$ como $\mathcal{M}(T)=(a_{ij})_{i=1...m,j=1...n}$, tal que
\[
T(v_j)=\sum^m_{i=1}a_{ij}w_i
\]
Si las bases no están claras por contexto usamos la notación $\mathcal{M}(T,\mathcal{B}_V,\mathcal{B}_W)$\\
Es facil ver que
\[
\mathcal{M}(T_1+\lambda T_2)=\mathcal{M}(T_1)+\lambda\mathcal{M}(T_2)
\]
(Ejercicio)\\
De esta forma, $\mathcal{M}$ "respeta" la estructura lineal de $\mathcal{L}(V,W)$\\
Ejemplos:\\
$\mathbb{F}^\infty=V=W$
\begin{enumerate}
	\item $L:\mathbb{F}^\infty\rightarrow\mathbb{F}^\infty$
	\[
	x=(x_1,x_2,...)
	\]
	\[
	Lx=(x_2,x_3,...)
	\]
	
	\item $R:\mathbb{F}^\infty\rightarrow\mathbb{F}^\infty$
	\[
	x=(x_1,x_2,...)
	\]
	\[
	Rx=(0,x_1,x_2,...)
	\]
\end{enumerate}
Pregunta: $\mathcal{M}(L),\mathcal{M}(R)$?
\subsection{Composición y Productos}
\subsubsection{Teo}
Sean $V,W,Z$ espacios vectoriales sobre $\mathbb{F}$. Sea $T\in\mathcal{L}(V,W)$ y $S\in\mathcal{W,Z}$. Entonces por composición: $S\circ T:V\rightarrow Z$ dada por $(S\circ T)(v)=S(T(v))$ es una transformación lineal.\\
Dem: Tenemos que probar que:
\[
(S\circ T)(\lambda u+v)=\lambda(S\circ T)(u)+(S\circ T)(v)
\]
En efecto:
\[
(S\circ T)(\lambda u+v)=S(T(\lambda u+v))=S(\lambda T(u)+T(v))
\]
\[
S\circ T)(\lambda u+v)=\lambda S(T(u))+S(T(v))=lambda(S\circ T)(u)+(S\circ T)(v)
\]
\begin{defn}{Endomorfismo}

Una transformación lineal $T\in\mathcal{L}(V,V)$ de dice endomorfismo u operador lineal. Denotamos $End (V)=\mathcal{L}(V,V)$. La composición funciona como producto sobre $End(V)$\\
Propiedades:
\begin{enumerate}[label=\alph*)]
	\item $I\circ S=S\circ I=S$ ($I$ es neutro para $\circ$)
	
	\item \begin{itemize} \item $S\circ (T_1+T_2)=S\circ T_1+S\circ T_2$
					\item $(T_1+T_2)\circ S=T_1\circ S+T_2\circ S$
		\end{itemize}
	
	\item $\lambda(S\circ T)=(\lambda S)\circ T=S\circ(\lambda S)$
\end{enumerate}
\end{defn}
\begin{proof}
Ejercicio
\end{proof}
Importante notar que $\circ$ no es conmutativo.\\
También es importante observar que \underline{NO} todo operador posee elemento inverso para $\circ$. Dado $T\in End(V)\setminus\{0\}$, decimos que $S\in End(V)\setminus\{0\}$ es su inversa si:
\[
S\circ T=T\circ S=I
\]
\subsubsection{Corolario}
Sea $V$ espacio vectorial finito-dimensional y $\mathbb{B}_V=\{v_1,...,v_n\}$ base de $V$. Entonces $\{E^{p,1}:p,q=1,...,n\}$ es base de $End(v)$.
\begin{proof}
Directo por Teo (+).
\end{proof}
Lema: Sean $S,T\in End(V)$ con $V$ finito dimensional. Entonces:
\[
\mathcal{M}(S\circ T)=\mathcal{M}(S)\cdot\mathcal{M}(T)
\]
\begin{proof}
Recordar que
\[\{E^{p,q}:p,q=1,...,n\}\]
son base de $End(V)$.
\[\mathcal{M}(S)=\left(b_{p,q}\right)_{p=1,...,n\\q=1,...n}\]
\end{proof}

\subsection{Invertibilidad}
\begin{defn}
$T\in\mathcal{L}(V,W)$ se dice invertible si existe $S:W\rightarrow V$ tal que
\begin{equation}
S\circ T=I_V, T\circ S=I_W
\end{equation}
Cuando $T$ es invertible, denotamos $T^{-1}$ com su inversa
\[T\textrm{ invertible}\iff T\textrm{ es inyectiva y es sobreyectiva}\]
\end{defn}
\begin{obs}
\
\begin{enumerate}
	\item No todo $T\in End(V)\setminus\{0\}$ es invertible
	
	\item En el caso $V=W$, $I_V$ es neutro para $\circ$
	
	\item Puede ser que $S\circ T=I$ y $T\circ S\neq I$
\end{enumerate}
\end{obs}

\begin{thm}
Sean $V,W$ espacios vectoriales sobre $\mathbb{F}$ y $T\in\mathcal{L}(V,W)$. Si $T$ es invertible entonces $T^{-1}\in\mathcal{L}(W,V)$
\end{thm}
\begin{proof}
Queremos probar $T^{-1}(w_1+\lambda w_2)=T^{-1}(w_1)+\lambda T^{-1}(w_2)$\\
Sean $v_1=T^{-1}(w_1),v_2=T^{-1}(w_2)$
\[
T(v_1+\lambda v_2)=T(v_1)+\lambda T(v_2)\quad /T^{-1}(\,)
\]
\[
v_1+\lambda v_2=T^{-1}(T(v_1)+\lambda T(v_2))
\]
\begin{equation}
T^{-1}(w_1)+\lambda T^{-1}(w_2)=T^{-1}(w_1+\lambda w_2)\qedhere
\end{equation}
\end{proof}
\begin{prop}
Sean $T:V\rightarrow W,S:W\rightarrow Z$ lineales e invertibles. Entonces $S\circ T$ es lineal e invertible; mas aun
\begin{equation}
(S\circ T)^{-1}=T^{-1}\circ S^{-1}
\end{equation}
\end{prop}
\begin{proof}
$S\circ T$ es lineal (visto lunes).\\
Invertible: basta probar:
\[
(S\circ T)\circ(T^{-1}\circ S^{-1})=S\circ (T\circ T^{-1})\circ S^{-1}=S\circ S^{-1}=I_Z
\]
\[
(T^{-1}\circ S^{-1})\circ (S\circ T)=T^{-1}\circ (S^{-1}\circ S)\circ T=T^{-1}\circ T=I_V
\]
\end{proof}
\begin{defn}
Decimos que $T\in\mathcal{L}(V,W)$ es no singular si $\ker T=[0]$. Notar ademas que $T$ no singular si solo si inyectiva
\begin{equation}
Tv_1=Tv_2\iff T(v_1-v_2)=0
\end{equation}
\end{defn}
\begin{thm}
Sea $T\in\mathcal{L}(V,W)$. Entonces:\\
$T$ no-singular $\iff T$ transforma conjuntos linealmente independientes en conjuntos linealmente independientes, es decir, $\{v_1,...,v_k\}\subseteq V\textrm{ l.i.}\implies\{Tv_1,...,Tv_k\}\subseteq W\textrm{ l.i.}$
\end{thm}
\begin{thm}{$\square$ }
Sean $V,W$ finito-dimensionales tal que $\dim V=\dim W$. Si $T\in\mathcal{L}(V,W)$ entonces las siguientes son equivalentes:
\begin{enumerate}[label=\roman*)]
	\item $T$ es invertible
	
	\item $T$ es no-singular
	
	\item $T$ es sobreyectiva
	
	\item Para toda base $\{v_1,...,v_n\}$ de $V$, $\{Tv_1,...,Tv_n\}$ es base de $W$.
	
	\item Existe $\{v_1,...,v_n\}$ base de $V$ tal que $\{Tv_1,...,Tv_n\}$ es base de $W$
\end{enumerate}
\end{thm}
\begin{obs}
\
\begin{enumerate}
	\item Si $\dim V\neq\dim W$\\
	$T:\mathbb{R}^n\rightarrow\mathbb{R}^m n<m$\\
	$(x_1,..,x_n)\mapsto (x_1,...,x_n,0,...,0)$\\
	es inyectiva, pero no sobreyectiva
	
	\item $V=W,\dim V=+\infty$\\
	$R:\mathbb{F}^\infty\rightarrow\mathbb{F}^\infty$\\
	$(x_1,...)\mapsto (0,x_1,...)$\\
	es inyectiva, pero no sobreyectiva
\end{enumerate}
\end{obs}
\begin{proof}
$(i)\implies (ii)$\\
$(ii) \ker T=\{0\}\iff T(V)=W$\\
$(ii)\implies (iii)$\\
$(iii) T$ es sobre. Por TNI
\[
\dim T=n\implies\dim\ker T=0\implies T\textrm{ no singular}
\]
Si $\{v_1,...,v_n\}$ es base de $V$, por el Teo. anterior $\{Tv_1,...,Tv_n\}$ se base de $W$
\end{proof}
\section{Isomorfismos}
\begin{defn}{Isomorfismos}
Sean $V,W$ espacio vectorial sobre $\mathbb{F}$. Decimos que $T\in\mathcal{L}(V,W)$ es un isomorfismo si es invertible. En tal caso, diremos que $V$ y $W$ son isomorfos
\end{defn}
Ejemplos:
\begin{enumerate}
	\item $\mathbb{F}^{n+1}$ y $\mathcal{P}_n(\mathbb{F})$ son isomorfos
	\[T:\mathbb{F}^{n+1}\rightarrow \mathcal{P}_n(\mathbb{F})\]
	\[(a_0,a_1,....,a_n)\mapsto p(x)=\sum^n_{I=1}a_ix^i+a_o\]
	
	\item (Coordenadas baricéntricas): Si $B=\{v_1,...,v_n\}$ es base de $V$
	\[(\forall v\in V)v=\sum^n_{I=1}\lambda_i v_i\]
	\[ [\cdot ]_B:V\rightarrow \mathbb{F}^n\]
	\[ v\mapsto(\lambda_1,...,\lambda_n)=[v]_B\]
	
	\item $V,W$ finito dimensional sobre $\mathbb{F}$, con
	\[B_V=\{v_1,...,v_n\}\]
	\[B_W=\{w_1,..,w_m\}\]
	\[\mathcal{M}(\cdot,B_V,B_W):\mathcal{L}(V,W)\rightarrow \mathbb{F}^{n\times m}\]
	\[T\mapsto\mathcal{M}(T,B_V,B_W)\]
	es un isomorfismo (ejercicio)
\end{enumerate}
\begin{thm}
Dos espacios finito dimensionales $V,W$ (sobre $\mathbb{F}$ son isomorfos si solo si $\dim V= \dim W$
\begin{proof}
Sean $B_V=\{v_1,...,v_n\}$ base de $V$, $B_W=\{w_1,...,w_m\}$ base de $W$.\\
$\implies$ Sea $T\in\mathcal{L}(V,W)$ isomorfismo
\[\{Tv_1,...,Tv_n\}\subseteq W\]
Es un conjunto linealmente independiente (un Teorema)
\[\implies n\leq \dim W=m\]
Tomando $T^{-1}$, (que también es isomorfismo) tomemos
\[\{T^{-1}w_1,...,T^{-1}w_m\}\subseteq V\]
Es linealmente independiente
\[\implies m\leq\dim V=n\]
\[\implies m=n\]
$\impliedby$ Suponemos $n=m$, sea $T$ la única transformación lineal tal que
\[Tv_i=w_i\quad\forall I=1,...,n\]
Por el teorema $\square$, parte $(v)\implies (i)$ tenemos que $T$ es isomorfismo.
\end{proof}
\end{thm}
\subsection{Matrices representantes}
Sean $V,W$ espacios vectoriales sobre $\mathbb{F}$, y sean $T\in\mathcal{L}(V,W),B_V=\{v_1,...,v_n\}$ base de $V$, $B_W=\{w_1,...,w_m\}$ base de $W$.
\[ [Tv_j]_{B_W}=(a_{1j},...,a_{mj})\iff T(v_j)=\sum^m_{I=1}a_{ij}w_i\quad (\forall v\in V \exists a_{ij})\]
Luego, par todo $v\in V$
\[v=\sum_j\lambda_jv_j\quad(\exists \lambda_j)\iff [Tv]_{B_V}=(\lambda_{1},...,\lambda_{m})\]
\[T(v)=T\left(\sum^n_{j=1}\lambda_j v_j\right)=\sum^n_{j=1}\lambda_jT(v_j)\]
De esta forma,
\[ [T(v)]_{B_W}=A\cdot [v]_{B_V}\]
Notemos que $A$ coincide con la matriz representante de $T$ con respecto a las bases $B_V$ y $B_W$.
\[\mathcal{M}(T,B_V,B_W)=M\]
Evaluar (*) para $v=v_j$
\[[T(v_j)]_{B_W}=A\cdot[v_j]_{B_V}=A\cdot e_j\]
\[\implies T(v_j)=\sum^m_{I=1}a_{i,j}\cdot w_i\]
\[\therefore T(v)=\sum^n_{j=1}\lambda_jT(v_j)=\sum_j\sum_i\lambda_ja_{i,j}w_i\]
\[T(v)=\sum_j\sum_ia_{i,j}E^{i,j}(\sum_q\lambda_qv_q)=\sum_j\sum_ia_{i,j}E^{i,j}(v)\]
\[\therefore T=\sum_i\sum_ja_{i,j}E^{I,j}\]
\subsection{Cambios de Base:}
Sean $B=(v_1,...,v_n)$ y $B'=(v_1',...,v_n')$ 2 bases (ordenadas).\\
Como se relacionan $[\cdot ]_{'}B$ y $[\cdot ]_{B}$?
\[ T:V\rightarrow\mathbb{F}^n,U:V\rightarrow\mathbb{F}^n\]
\[v\mapsto [v]_B,v\mapsto [v]_{B'}\]
Notemos que $T\circ U^{-1}:\mathbb{F}^n\rightarrow\mathbb{F}^n,[v]_B\mapsto [v]_{B'}$ es un isomorfismo.\\
Sea $P$ la matriz representante de $T\circ U^{-1}$ con respecto a la base canónica en $\mathbb{F}^n$\\
Usando (*):
\[ [v]_B=P\cdot [v]_{B'}\quad \forall v\]
Sea ahora $T\in\mathcal{L}(V,V)$. Tomemos:
\[[T(v)]_B=\mathcal{M}_{B,B}(T)\cdot [v]_B\]
\[ P\cdot [T(v)]_{B'}=\mathcal{M}_{B,B}(T)\cdot P\cdot [v]_{B'}\]
\[[T(v)]_{B'}=\left(P^{-1}\cdot\mathcal{M}_{B,B}(T)\cdot P\right)\cdot [v]_{B'}\]
\[\implies \mathcal{M}_{B',B}(T)=\left(P^{-1}\cdot\mathcal{M}_{B,B}(T)\cdot P\right)\]
\underline{Pregunta}: Como calcular $P$?
\[ [v_j']_B=P\cdot [v_j']_{B'}=P\cdot j\]
\[ P=\left[ [v_1']_B| [v_2']_B|...|[v_n']_B\right]\oplus\]
\begin{thm}
Sea $V$ espacio vectorial finito-dimensional con bases ordenadas $B=(v_1,...,v_n)$ y $B'=(v_1',...,'v_n')$. Sea $T\in End(V)$. Luego
\[\mathcal{M}_{B',B}(T)=P^{-1}\mathcal{M}_{B,B}(T)P\]
\end{thm}

\begin{defn}
$A,B\in\mathbb{F}^{n\times n}$ son similares si $\exists P$ invertible tal que:
\[A=P^{-1}BP\]
\end{defn}
\section{Productos de Espacios Vectoriales}
\begin{defn}
Sean $V_1,...,V_m$ espacios vectoriales sobre $\mathbb{F}$. Definimos el espacio producto como
\[V=V_1\times...\times V_m=\{(v_1,...,v_m):v_i\in V_i(\forall i=1,...,m)\}\]
\begin{enumerate}
	\item Suma: $(v_1,...,v_m)+(v_1',...v_m')=(v_1+v_1',...,v_m+v_m')$
	
	\item Producto por escalar: $\lambda\cdot (v_1,...,v_m)=(\lambda v_1,...,\lambda v_m)$
\end{enumerate}
\end{defn}
\begin{prop}
$V\times...\times V_m$ con suma y producto escalar es un espacio vectorial.
\end{prop}
\begin{proof}
Ejercicio: similar a $\mathbb{F}^n$
\end{proof}
\begin{prop}
Sean $V_1,...,V_m$ espacios vectoriales sobre $\mathbb{F}$. Entonces $V_1\times...\times V_m$ es finito-dimensional y 
\[\dim(V_1\times...\times V_m)=\sum^m_{i=1}\dim V_i\]
\end{prop}
\begin{proof}
Dado $I=1,...,m$ sea
\[B_{V_i}=\{V_{1,1},...,V_{i,n_i}\}\]
Base de $V_i$. Ahora sea
\[B=\bigcup^m_{i=1}\{(0,...,0,v_{ij},...,0):j=1,...,n_i\}\]
Probaremos que $B$ es base de $V_1\times...\times V_m$. Notemos que esto basta:
\[\dim V_1\times...\times V_m=|B|=\sum^m_{i=1}n_i=\sum^m_{i=1}\dim V_i\]
\begin{itemize}
	\item \underline{$B$ genera}: Sea $v\in V\times...\times V_m$, entonces
	\[v=(v_1,...,v_m)\quad\textrm{donde }v_i\in V_i\]
	Como $B_{V_i}$ es base de $V_i$:
	\[v_i=\sum^{n_i}_{j=1}\lambda_{i,j}\cdot v_{i,j}\]
	\[v=\left(\sum^{n_1}_{j=1}\lambda_{1,j}v_{1,j},...,\sum^{n_m}_{j=1}\lambda_{m,j}v_{m,j}\right)\]
	\[v=\sum^m_{i=1}\sum^{n_i}_{j=1}\lambda_{i,j}(0,...,0,v_{i,j},0,...,0)\in <B>\]
	
	\item $B$ es linealmente independiente (ejercicio)
\end{itemize}
\end{proof}
\subsection{Productos y Sumas directas}
\begin{thm}
Sean $U_1,...U_m$ subespacios vectoriales de $V$. Definimos la transformación lineal
\[\Gamma:U_1\times...\times U_m\rightarrow U_1+...+U_m\]
\[\Gamma(u_1,...,u_m)=u_1+...+u_m\]
Entonces, $U_1+...+U_m$ es suma directa si solo si $\Gamma$ inyectiva
\end{thm}
\begin{obs}
Notar que $\Gamma$ siempre es sobreyectiva
\end{obs}
\begin{proof}
\[\Gamma\textrm{ inyectiva} \iff\ker\Gamma=\{0\}\]
\[\iff [u_1+...+u_m=0\iff(u_1,...,u_m)=(0,...,0)]\]
\[\iff U_1+...+U_m\textrm{ es directa}\qedhere\]
\end{proof}

\chapter{Parte II}
\section{Polinomios}
\begin{defn}[Grado]
Si $p(z)=a_0+....+a_nz^n$ con $a_n\neq 0$, entonces $gr(p)=n$. Si $p(z)=0$ entonces $gr(p)=-\infty$
\end{defn}
\begin{prop}

\end{prop}
\subsection{Algoritmo de la división}
Recordemos que si $p,s$ son enteros, no negativos, con $s\neq 0$, existen únicos $q,r$ enteros no negativos tal que:
\[p=s\cdot q+r\]
Donde $r<s$.\\
De ahora en adelante, $\mathbb{F}=\mathbb{R}\vee\mathbb{C}$ a menos que se diga lo contrario.
\begin{thm}[División de polinomios]
Sean $p,s\in\mathbb{F}[z]$ con $s\neq 0$. Entonces existen únicos polinomios $q,r\in\mathbb{F}[z]$ tales que:
\[p=q\cdots+r\implies n=gr(q)+m\]
Con $gr(r)<gr(s)$.
\end{thm}
\begin{proof}
Sean $n=gr(p), m=gr(s)$. Definamos
\[T:\mathcal{P}_{n-m}(\mathbb{F})\times\mathcal{P}_{m-1}(\mathbb{F})\rightarrow\mathcal{P}_{n}(\mathbb{F})\]
\[(q,r)\mapsto q\cdot s+r\]
Probaremos que $T$ es una transformación lineal biyectiva. Notar que esto es suficiente para concluir el Teorema.
\begin{itemize}
	\item \underline{T es lineal}
	\[T((q_1,r_1)+\lambda (q_2,r_1))=T(q_1,r_1)+\lambda T(q_2,r_2)\]
	
	\item \underline{T inyectivo}: Basta probar que $\ker T=\{0\}$. En efecto, si $(q,r)$ son tales que:
	\[q\cdot s+r=0\]
	\[q\cdot s=-r\]
	Comparando grados:
	\[gr(LI)=gr(q)\cdot m,gr(LD)=\leq m-1\]
	\[\implies q=0\implies r=0\]
	
	\item \underline{T sobreyectiva}: Por el TNI, y como $\ker T=\{0\}$
	\[\dim(\mathcal{P}_{n-m}\times\mathcal{P}_{m-1})=\dim T(\mathcal{P}_{n-m}\times\mathcal{P}_{m-1})\]
	\[\dim(\mathcal{P}_{n-m})+\dim(\mathcal{P}_{m-1})=n+1\]
	\[\therefore \dim T(\mathcal{P}_{n-m}\times\mathcal{P}_{m-1})=n+1=\dim\mathcal{P}_{n-}\]
	Lo que implica que $T$ es sobreyectiva.
\end{itemize}
\end{proof}
\begin{obs}
La demostración del Teo. Anterior entrega un "algoritmo" para dividir polinomios. Resolver la ecuación lineal no-homogénea
\[T(q,r)=p\]
Asignando bases a la partida y la llegada, se obtendrá un sistema lineal
\[Ax=b\]
\end{obs}
\subsection{Raíces de Polinomios}
El estudio de la ecuación $p(z)=0$ es sumamente útil para analizar un polinomio $p\in\mathbb{F}$
\begin{defn}[Raíz]
$\lambda\in\mathbb{F}$ se dice raiz de un polinomio si
\[p(\lambda)=0\]
\end{defn}
\begin{defn}[Factor]
$s\in\mathbb{F}[z]$ se dice factor de $p\in\mathbb{F}[z]$ si existe un polinomio $q\in\mathbb{F}[z]$ tal que:
\[p=q\cdot s\]
\end{defn}
\begin{thm}[Raíces definen factores de grado 1]
Sea $p\in\mathbb{F}[z]$ y $\lambda\in\mathbb{F}$. Entonces $p(\lambda)=0\iff(z-\lambda)$ es factor de $p$
\begin{proof}
$\impliedby$ $(z-\lambda)$ factor de $p$, entonces $\exists q\in\mathbb{F}[z]$ tal que
\[p(z)=(z-\lambda)\cdot q(z)\forall z\in\mathbb{F}\]
\[\implies p(\lambda)=0\cdot q(\lambda)=0\]
$implies$. Por el algoritmo de la división para $p$ y $s(z)=z-\lambda$, tenemos que $\exists !r\in\mathbb{F}[z]$ con $gr(r)\leq=0\implies r\in\mathbb{F}$ tal que 
\[p(z)=q(z)\cdot (z-\lambda)+r\]
\[\implies 0=p(\lambda)=q(\lambda)\cdot 0+r\]
\[\therefore r=0\qedhere\]
\end{proof}
\begin{cor}[Un polinomio tiene tantas raíces como su grado]
Sea $p\in\mathbb{F}[z]$ un polinomio de grado $m\geq 0$. Entonces $p$ tiene a lo más $m$ raíces distintas sobre $\mathbb{F}$
\begin{proof}
Por inducción en $m$.\\
$m=0$: $p(z)=a_0\neq 0$. Entonces $p$ tiene 0 raíces.\\
$m-1\implies m$: Sea $p(z)=a_0+a_1z+...+a_mz^m$ con $a_m\neq 0$
\begin{enumerate}[label= \underline{Caso \arabic*}:]
	\item $p$ no posee raíces
	
	\item $p$ sí posee raíces. Sea $\lambda\in\mathbb{F}$ raíz de $p$
	\[\therefore \exists q\in\mathbb{F}[z]: p(z)=(z-\lambda)\cdot q(z)\]
	Claramente $gr(q)=m-1$. Por inducción, $q$ posee a lo más $m-1$ raíces en $\mathbb{F}$. Por lo tanto:
	\[\#\textrm{ raíces de }p\leq \#\textrm{ raíces de }(z-\lambda)+\#\textrm{ raíces de }q=m\]
\end{enumerate}
\end{proof}
\end{cor}
\end{thm}
\section{Subespacios invariantes, y valores y vectores propios}
Queremos entender la estructura de $T\in End(V)=\mathcal{L}(V)$ supongamos que
\[V=U_1\oplus...\oplus U_m\]
y
\[T|_{u_1},T|_{u_2},...,T|_{u_m}\]

\begin{defn}[Subespacio invariante]
	Sea $T\in\mathcal{L}(V)$ un subespacio $U$ de $V$ se dice invariante para $T$ si $TU\subseteq U$
	\[\implies TU=\{w\in V:w=Tv\}\]
\end{defn}
Propiedad: Sea $p,q\in\set{F}[z]$ y $T\in\mathcal{L}(V)$. Entonces:
\begin{enumerate}
	\item $(p\cdot q)(T)=p(T)\cdot q(T)$
	
	\item $p(T)\cdot q(T)=q(T)\cdot p(T)$
\end{enumerate}
\begin{proof}
	Sea $p(z)=\sum^n_{j=0}a_jz^j$ y $q(z)=\sum^m_{k=0}b_kz^k$
	\[(p\cdot q)(z)=\left(\sum^n_{j=0}a_jz^j\right)\left(\sum^m_{k=0}b_kz^k\right)\]
	\[p\cdot q)(z)=\sum^n_{j=0}\sum^m_{k=0}a_jb_kz^{j+k}\]
	\[\implies (p\cdot q)(T)=\sum^n_{j=0}\sum^m_{k=0}a_jb_kT^{j+k}\]
	\[(p\cdot q)(T)=(\sum_ja_jT^j)(\sum^m_kb_kT^k)\]
\end{proof}
\begin{thm}[Existencia de vps]
	Todo operador en un espacio vectorial complejo de dimensión finita y $>0$, posee un valor propio.
	\begin{proof}
		Sea $n=\dim V>0$ y sea $T\in\mathcal{L}(V)$. Sea ahora $v\neq\vec{0}$. Entonces
		\[\{v,Tv,T^2v,...,T^nv\}\]
		es linealmente dependiente. Luego existen $a_0,..,a_n$ tal que $a_0v+a_1Tv+...+a_nT^nv=0$
		\[\therefore p(T)v=0\quad\textrm{con gr$(p)=n$}\]
		Por el teorema fundamental del algebra, $p$ puede ser factorizado
		\[p(z)=C(z-\lambda_1)\cdot...\cdot(z-\lambda_n)\]
		Por ende:
		\[p(T)v=0=C(T-\lambda_1 I)\cdot...\cdot(T-\lambda_n I)v\]
		$\implies$ Existe $j=1,...,n$ tal que Im$(T-\lambda I)\neq V$
		$\therefore\lambda_j$ es valor propio de $T$
	\end{proof}
\end{thm}
\section{Matrices triangulares superiores}
Recordemos que daa $T\in\mathcal{L}(V)$ y una base $B$ la matriz representante de $T$ con respecto a $B$ es
\[\mathcal{M}_B(T)=(a_{i,j})_{i,j\in[n]}\]
tal que 
\[Tv_j=\sum^n_{i=1}a_{i,k}v_k\]
Notemos, por ejemplo, que en el caso complejo siempre existe una base $B$ tal que la matriz representante posee la siguiente estructura
\[\mathcal{M}_B(T)=\begin{bmatrix}\lambda && 0 && \ldots\\ 0 && \ddots\\ \vdots \end{bmatrix}\]
En efecto, sea $\lambda\in\set{F}$ valor propio de $T$ con vector propio asociado $\vec{v}\neq\vec{0}$.\\
Sea $B$ una base que contiene a $v$. Si hacemos esto
\[Tv=\lambda v\]
\[\begin{pmatrix}
	a_{1,1}\\
	\vdots\\
	\vdots\\
	a_{n,1}
\end{pmatrix}=\begin{pmatrix}
	\lambda\\
	0\\
	\vdots\\
	0
\end{pmatrix}\]
\begin{defn}[Matriz triangular superior]
	Una matriz $A\in\set{F}^{n\times m}$ se dice triangular superior si
	\[a_{i,j}=0\forall i>j\]
	\[\begin{bmatrix}
		\lambda_1 && * && \cdots && *\\
		0 && \lambda_2 && \cdots && *\\
		\vdots && \vdots && \ddots && \vdots\\
		0 && 0 && \cdots && \lambda_n
	\end{bmatrix}\]
\end{defn}
\begin{prop}
	Sea $T\in\mathcal{L}(V)$ y $v_1,...,v_n$ base (ordenada) de $V$. Las siguientes propiedades son equivalentes:
	\begin{enumerate}[label=\alph*)]
		\item La matriz representante $\mathcal{M}_B(T)$ es $\triangle$ superior

		\item $Tv_j\in<v_1,...,v_j>\forall j=1,...,n$

		\item $<v_1,...,v_j>$ es invariante bajo $T, \forall j=1,...,n$
	\end{enumerate}
	\begin{proof}
		
	\end{proof}
\end{prop}
\begin{prop}[Invertibilidad a través de representaciones matriciales]
	Sea $T\in\mathcal{L}(V)$ tal que posee una representación $\triangle$ superior con respecto a cierta base. Entonces $T$ invertible $\iff$ las entradas diagonales de la matriz son no nulas.
	\begin{proof}
		Sea $B$ base tal que
		\[\mathcal{M}_B(T)=\begin{bmatrix}
			\lambda_1 && \cdots && *\\
			\vdots && \ddots && \vdots\\
			0 && \cdots && \lambda_n
		\end{bmatrix}(*)\]
		$\impliedby$ Por $(*)$
		\[Tv_1=\lambda_1v_1\]
		\[\therefore v_1=T\left(\frac{v_1}{\lambda_1}\right)\in \Ima(T)\]
		Nuevamente, por $(*)$
		\[Tv_2=av_1+\lambda_2v_2\]
		\[\lambda\neq 0\implies v_2=T\left(\frac{v_2}{\lambda_2}\right)+\frac{a}{\lambda_2}v_1\in\Ima(T)\]
		Por inducción:
		\[Tv_j=a_1v_1+a_2v_2+...+a_{j-1}v_{j-1}+\lambda_jv_j\]
		\[v_j=T\left(\frac{v_j}{\lambda_j}\right)+\frac{a_1}{\lambda_j}v_1+...+\frac{a_{j-1}}{\lambda_j}v_{j-1}\in\Ima(T)\]
		$\implies$ Sabemos que $\forall j=1,...,n$
		\[Tv_j=a_1v_1+...+a_{j-1}v_{j-1+\lambda_jv_j}\]
		\[\implies Tv_j\in<v-1,...,v_j>\]
		Si algún $\lambda_j=0$:
		\[Tv_j\in<v_1,...,v_{j-1}>\]
		\[T(<v_1,...,v_j>)\subseteq <v_1,...,v_{j-1}>\]
		En efecto
	\end{proof}
	\begin{cor}[Valores propios de operador a través de representación $\triangle$ superior]
		Sea $T\in\mathcal{L}(V)$ con representación $\triangle$ superior
		\[\mathcal{M}_B(T)=\begin{bmatrix}
			\lambda_1 && \cdots && *\\
			\vdots && \ddots && \vdots\\
			0 && \cdots && \lambda_n
		\end{bmatrix}(*)\]
		con $B$ base de $V$. Entonces los valores propios de $T$ son $\lambda_1,...,\lambda_n$
		\begin{proof}
			\[T_\lambda I\quad\textrm{ no es biyectiva}\]
			\[\mathcal{M}_B(T-\lambda I)=\begin{bmatrix}
				\lambda_1-\lambda && \cdots && *\\
				\vdots && \ddots && \vdots\\
				0 && \cdots && \lambda_n-\lambda
			\end{bmatrix}(*)\]
			\[\mathcal{M}_B(T-\lambda I)=\mathcal{M}_B(T)-\lambda\mathcal{M}_B(I)\]
			Por el Teo ($\therefore$)\\
			$T-\lambda I$ invertible si solo si $\lambda_1-\lambda,...,\lambda_n-\lambda\neq 0$\\
			$\lambda$ es valor propio si solo si $\lambda_1=\lambda$ o $\lambda_2=\lambda$ ... $\lambda_n=\lambda$
		\end{proof}
	\end{cor}
\end{prop}
\section{Subespacios propios y Matrices Diagonales}
\begin{defn}[Diagonal de una matriz diagonal]
	Sea $A\in\set{F}^{n\times m}$. Definimos $Diag(A)\in\set{F}^{N\times m}$
	\[(Diag(A))_{i,j}=\]
	Y decimos que $A$ es diagonal si
	\[A=Diag(A)\]
	Equivalentemente, $A$ es diagonal si 
	\[a_{i,j}=0\forall i\neq j\]
	Claramente, toda matriz diagonal es $\triangle$ superior, entonces los valores propios de $A$ son los valores en la diagonal.	
\end{defn}
\begin{defn}[Subespacio Propio]
	Sea $T\in\mathcal{L}(V)$ y $\lambda\in\set{F}$. El subespacio propio de $T$ correspondiente a $\lambda, E(\lambda,T)$, se define como
	\[E(\lambda,T)=\ker (T-\lambda I)\]
	es decir, el conjunto de vectores propios asociados a $\lambda$, más el cero.
	\begin{obs}
		$\lambda$ es valor propio de $T\iff E(\lambda, T)\neq \{0\}$
	\end{obs}
\end{defn}
\begin{prop}[Suma de subespacios propios es directa]
	
\end{prop}
\chapter{Espacios de Producto Interno}
\begin{defn}[Producto Interno]
	Un producto interno sobre un espacio vectorial $V$ es una función
	\[\func{<\cdot,\cdot>}{V\times V}{\set{F}}{}{}\]
	\[(u,v)\mapsto <u,v>\]
	\begin{enumerate}[label=\roman*)]
		\item Positividad: $<v,v>\geq 0\quad\forall v\in V$

		\item Definitividad: $<v,v>=0\iff v=\vec{0}$

		\item Aditividad por la izquierda: $\forall u,v,w\in V$
		\[<u+v,w>=<u,w>+<v,w>\]

		\item Homogeneidad: $<\lambda u,v>=\lambda<u,v>\quad\forall\lambda\in\set{F}\forall u,v\in V$

		\item Simetría conjugada: $<u,v>=\overline{<v,u>}$
	\end{enumerate}
\end{defn}
\begin{obs}
	En el caso $\set{F}=\set{R}, (V)\iff <u,v>=<v,u>$
\end{obs}
\underline{Ejemplos:}
\begin{enumerate}[label=\alph*)]
	\item El producto interno Euclideano sobre $\set{F}^n$
	\[<(z_1,...,z_n),(w_1,...,w_n)>=z_1\cdot \overline{w_1}+...+z_n\cdot \overline{w_n}\]

	\item Si $c_1,...,c_n>0$, entonces
	\[<(z_1,...,z_n),(w_1,...,w_n)>=\sum^n_{i=1}c_i\cdot z_i\cdot \overline{w_i}\]
	
	\item Si $V=\mathcal{C}[-1,1]$
	\[<f,g>=\int^1_{-1}f(x)g(x)\d x\]

	\item Si $V=\set{R}[x]$ entonces
	\[<p,q>=\int_0^\infty e^{-x} p(x)q(x)\d{x}\]
\end{enumerate}
\subsubsection{Integrales}
Si $f\in\mathcal{C}[a,b]$ y $f\geq 0$ definimos la integral de $f$ como el "área bajo la curva".
\[\int^b_a f(x)\d{x}=\lim_{N\rightarrow\infty}\frac{1}{N}\sum^{N-1}_{n=0}f(x_n)\]
\[x_i=a+\frac{(b-a)}{N}\cdot i\]
\[i=0,...,N\]
\begin{enumerate}
	\item \begin{thm}[Teorema Fundamental del Cálculo]
		Sea $\func{f}{[a,b]}{\set{R}}{}{}$ continua tal que para $\func{F}{[a,b]}{\set{R}}{}{}$ se cumple $F'(x)=f(x)$ (primitiva).\\
		Entonces:
		\[\int^b_a f(x)\d{x}=\sum^J_{j=1}\int_{A_j=[a_j,b_j]}f(x)\d{x}\]
	\end{thm}

	\item Linealidad:
	\[\func{\mathcal{I}}{\mathcal{C}[a,b]}{\mathcal{C}[a,b]}{}{}\]
	\[f(x)\mapsto\int^x_af(t)\d{t}\]
	es una tra lineal.
\end{enumerate}
Ejemplos:
\begin{enumerate}
	\item Monomio:
	\[p(t)=t^k\quad P(t)=\frac{t^{k+1}}{k+1}\]
	$P$ es primitiva de $p$:
	\[P'(t)=t^k=p(t)\]
	\[\therefore TFC\]
	\[\int^t_0p(s)\d{s}=P(t)-P(0)\]

	\item Polinomios:
	\[p(t)=a_0+a_1t+...+a_nt\]
	\[\int^t_0p(s)\d{t}=\sum^n_{k=0}a_k\int^t_0s^k\d{s}\]

	\item Exponenciales:
	\[f(x)=e^{\lambda x}\quad(\lambda\neq 0)\]
	\[F(x)=\frac{e^{\lambda x}}{\lambda}\]
	\[\therefore \int^x_0e^{\lambda t}\d{t}=F(x)-F(0)=\frac{e^{\lambda x}-1}{\lambda}\]
	
	\item Seno-Coseno:
	\[\func{f}{\set{R}}{\set{C}}{}{}\]
	\[f(x)=e^{i\lambda x}=\cos\lambda x+i\sin\lambda x\]
	\[F(x)=\frac{e^{\lambda x}}{\lambda}\]
	\[\therefore\int^x_0f(t)\d{t}=\frac{e^{i\lambda x}-1}{i\lambda}\]
	\[\int^x_0f(t)=\frac{\sin \lambda x}{\lambda}+i\frac{1-\cos \lambda x}{\lambda}\]
	\[\implies \int^x_0\sin\lambda t\d{t}=\frac{1-\cos \lambda x}{\lambda}\]
	\[\int^x_0\cos\lambda t\d{t}=\frac{\sin \lambda x}{\lambda}\]
\end{enumerate}
\section{Espacio de producto interno}
\begin{defn}[Espacio de producto interno]
	$(V,<\cdot,\cdot>)$ es un espacio de producto interno (e.p.i.) si $V$ es una espacio vectorial y $<\cdot,\cdot>$ es producto interno sobre $V$.
\end{defn}
\begin{prop}[Propiedades básicas]
	\
	\begin{enumerate}[label=(\alph*)]
		\item \[\forall u\in V\quad v\mapsto <v,u> \]
		Es una transformación lineal de $V$ hacia $\set{F}$

		\item \[<0,u>=0\forall u\in V\]

		\item \[<u,0>=0\forall u\in V\]

		\item \[<u,v+w>=<u,v>+<u,w>\forall u,v,w\in V\]

		\item \[<u,\lambda v>=bar{\lambda}<u,v>\forall u,v\in V \forall \lambda\in\set{F}\]
	\end{enumerate}
\end{prop}
\section{Norma}
\begin{defn}[Norma]
	Sea $(V,<\cdot,\cdot>)$ un espacio de producto interno, definimos la norma asociada como:
	\[\|x\|=\sqrt{<x,>}\]
\end{defn}
\underline{Propiedades:} Sea $v\in V$
\begin{enumerate}[label=(\alph*)]
	\item \[\|v\|=0\iff v=0\]

	\item \[\|\lambda v\|=|\lambda|\|v\|\forall \lambda\in \set{F}\]
\end{enumerate}
\subsubsection{Intuición geométrica:}
Sean $u,v\in\set{R}^2$\\
Por teo del coseno
\[\|v-u\|^2=\|u\|^2+\|v\|^2-2\|u\|\|v\|\cos\theta\]
\[\|v-u\|^2=<v-u,v-u>=\|v\|^2-<v,u>-<u,v>+\|u\|^2\]
\[\|v-u\|^2=\|v\|^2+\|u\|^2-2<u,v>\]
Reemplazando:
\[<u,v>=\|u\|\|v\|\cos\theta\]
\section{Conjuntos Ortonormales}

\begin{prop}[Norma de una combinación ortonormal]
	Sea $(V,<\cdot,\cdot>)$ espacio con producto interno y $e_1,...,e_m$ conjunto ortonormal. Entonces
	\[\|a_1e_1+...a_me_m\|^2=|a_1|^2+...+|a_m|^2\]
	\begin{proof}
		Por Pitágoras:
		\[\|a_1e_1+...a_me_m\|^2=\|a_1e_1\|^2+\|a_2e_2+...+a_me_m\|^2\]
		\[\|a_1e_1+...a_me_m\|^2=|a_1|^2\|e_1\|^2+\|a_2e_2+...+a_me_m\|^2\]
		\[\|a_1e_1+...a_me_m\|^2=|a_1|^2+\|a_2e_2+...+a_me_m\|^2\]
		\[\vdots\]
		\[\|a_1e_1+...a_me_m\|^2=|a_1|^2+...+|a_m|^2\]
	\end{proof}
\end{prop}
\begin{thm}
	Todo familia ortonormal es linealmente independente.
	\begin{proof}
		Sea $(e_\lambda)_{\lambda\in\Lambda}$ conjunto ortonormal. Si fueran linealmente dependientes, existen $a_1,...,a_m\in\set{F}$ no todos nulos
		\[\lambda_1,...,\lambda_m\in\Lambda\]
		\[a_1e_{\lambda_1}+...+a_me_{\lambda_m}=0\quad /\|\cdot\|^2\]
		\[0=\|a_1e_{\lambda_1}+...+a_me_{\lambda_m}\|^2\]
		\[0=|a_1|^2+...+|a_m|^2\]
		\[\iff a_1=...=a_m=0\]
		\[\contr\qedhere\]
	\end{proof}
\end{thm}
\begin{defn}[Base ortonormal]
	Una base ortonormal de un espacio con producto interno $(V,<\cdot,\cdot>)$ es un conjunto ortonormal que también es base.
\end{defn}
\begin{lem}
	Sea $(V,<\cdot,\cdot>)$ un espacio producto interno finito dimensional. Entonces un conjunto ortonormal de cardinalidad $\dim V$ es una base ortonormal
	\begin{proof}
		Sea $n=\dim V$ y sea $e_1,...,e_m$ conjunto ortonormal en $V$.
		\[\implies e_1,..,e_m\textrm{ son linealmente independiente}\]
		\[\implies\textrm{ son base}\qedhere\]
	\end{proof}
\end{lem}
\underline{Ejemplos:}
\begin{enumerate}
	\item $\set{F}^n$

	\item $ser{F}^4$, el conjunto
	\[(0.5,0.5,0.5,0.5),(0.5,0.5,-0.5,-0.5)\]
	\[(0.5,-0.5,-0.5,0.5),(-0.5,0.5,-0.5,0.5)\]
	es base ortonormal.
	Claramente: $\|\cdot\|=\sqrt{(0.5)^2+(0.5)^2+(0.5)^2+(0.5)^2}=1$\\
	Tambien los productos internos cruzados dan 0
\end{enumerate}
\underline{Pregunta:} Cuándo existen bases ortonormales uniformes en $\set{R}^n$
\begin{lem}
	Sea $e_1,..e_n$ vectores de $\set{F}^n$ y sea 
	\[U=\left[e_1|e_2|...|e_n\right]\]
	Entonces, $e_1,..,e_n$ es base ortonormal $\iff$
	\[U^T\cdot U=I_{n\times n}\]
	\begin{proof}
		\[\left[\frac{e_1^T}{\cfrac{e_2^T}{\cfrac{\vdots}{e_n^T}}}\right]\]
	\end{proof}
\end{lem}
\begin{defn}[Matriz unitaria]
	$U\in\set{F}^{n\times n}$ se dice unitaria si
	\[U^TU=I_{n\times n}\]
	Notar que $U$ es unitaria $\iff$ las columnas de $U$ son base ortonormal.
\end{defn}
Para responder la pregunta hacemos lo siguiente:
\begin{proof}
	\underline{$n=1$:} $\{1\}$ es dase ortonormal de $\set{F}$
	\[H_1=[1]\]
	\underline{$n\implies 2n$:} Sea $H_n$ una matriz unitaria tal que
	\[|(H_n)_{i,j}|=\frac{1}{\sqrt{n}}=c\]
	\[H_{2n}=\frac{1}{\sqrt{2}}\begin{bmatrix}
		H_n & H_n\\
		H_n & -H_n
	\end{bmatrix}\]
	\begin{enumerate}
		\item Esta matriz también es uniforme

		\item Esta matriz es unitaria
		\[H_{2n}^T\cdot H_{2n}=\frac{1}{2}\begin{bmatrix}
			H_n^T\cdot H_n+H_n^T\cdot H_n & H_n^T\cdot H_n-H_n^T\cdot H_n^T\cdot H_n\\
			H_n^T\cdot H_n-H_n^T\cdot H_n & H_n^T\cdot H_n+H_n^T\cdot H_n
		\end{bmatrix}=\begin{bmatrix}
			I_{n\times n} & 0\\
			0 & I_{n\times n}
		\end{bmatrix}\qedhere\]
		Las columnas son las base ortonormal uniforme buscada.\\
		Las matrices $H_1,H_2,H_4,...,H_{2^k}$ se conocen como matrices de Hadanard
	\end{enumerate}
\end{proof}
\begin{conj}[Hadanard]
	Estas matrices solo existen para
	\[n=4k,k\in\set{Z}\]
\end{conj}
Dada una base $e_1,...,e_n$ de $V$ tenemos
\[v=\sum^n_{i=1}a_ie_i\quad \forall v\in V\]
Cómo calcular $a_1,..,a_m$?
\begin{thm}
	Sea $e_1,..,e_n$ base ortonormal de $(V,<\cdot,\cdot>)$, entonces
	\[v=\sum^n_{i=1}<v,e_i>e_i\]
	además
	\[\|v\|^2=\sum^n_{i=1}|<v,e_i>|^2\]
	\begin{proof}
		Como $e_1,...,e_n$ es base, existen $a_1,...,a_n$ tal que\[v=\sum^n_{i=1}a_ie_i\quad /<\cdot,e_j>\]
		\[<v,e_j>=\sum^n_{i=1}a_i<e_i,e_j>=a_j\cdot 1\]
		Probamos la primera parte. La segunda es consecuencia de Pitágoras.
	\end{proof}
\end{thm}
\section{Algoritmo de Gram-Schmidt}
\begin{thm}
	Sean $v_1,...,v_m$ conjunto linealmente independente de vectores en $(V,<\cdot,\cdot>)$. Sea
	\[e_1=\frac{v_1}{\|v_1\|}\]
	\[e_j=\frac{v_j-\sum_{i<j}<v_j,e_i>e_i}{\|v_j-\sum_{i<j}<v_j,e_i>e_i\|}\]
	Entonces $e_1,...,e_n$ es conjunto ortonormal y $<e_1,...,e_n>=<v_1,...,v_n>$
	\begin{proof}
		Por inducción en $j$\\
		\underline{$j=1$:} $e_1$ es conjunto ortonormal:
		\[\|e_1\|=\left\|\frac{v_1}{\|v_1\|}\right\|=1\]
		y
		\[<e_1>=<v_1>\]
		Notar que $v_1\neq 0$ porque $v_1,...,v_n$ es linealmente independente y por ender $e_1$ esta bien definido.\\
		\underline{$j-1\implies j$:}\\
		$e_1,...,e_j$ es ortonormal.\\
		Primero $e_j$ está bien definido
		\[e_j=\frac{v_j-\sum_{i<j}<v_j,e_i>e_i}{\|v_j-\sum_{i<j}<v_j,e_i>e_i\|}\]
		y
		\[v_j\notin <v_1,...,v_{j-1}>\]
		\[\therefore v_j-\sum_{i<j}<v_j,e_i>e_i\neq 0\]
		Para probar ortonormalidad, basta probar que:
		\[<e_j,e_l>=\begin{cases}
			1\quad l=j\\
			0\quad l<j
		\end{cases}\]
		\[<e_j,e_j>=\|e_j\|^2=\frac{\|\cdot\|^2}{\|\cdot\|^2}=1\]
		\[<e_j,e_l>=\frac{1}{c}<v_j-\sum_{i<j}<v_j,e_i>e_i,e_l>\]
		\[<e_j,e_l>=\frac{1}{c}\left[<v_j,e_l>-<\sum_{i<j}<v_j,e_i>e_i,e_l>\right]\]
		\[<e_j,e_l>=\frac{1}{c}\left[<v_j,e_l>-\sum_{i<j}<<v_j,e_i>e_i,e_l>\right]\]
		\[<e_j,e_l>=\frac{1}{c}\left[<v_j,e_l>-\sum_{i<j}<v_j,e_i>\cdot<e_i,e_l>\right]\]
		Por hipótesis de inducción:
		\[<e_j,e_l>=\frac{1}{c}\left[<v_j,e_l>-<v_j,e_l>\cdot 1\right]=0\]
		$\therefore e_1,...,e_n$ ortonormal.\\
		Falta probar
		\[<e_1,...,e_n>=<v_1,...,v_n>\]
		Para probarlo basta ver que\\
		\[\textrm{(Hip. Ind.)} <e_1,...,e_{j-1}>=<v_1,...,v_{j-1}>\]
		Pero además
		\[e_j\in<e_1,..,e_{j-1},v_j>=<v_1,...,v_j>\]
		En conclusión
		\[<e_1,..,e_j>\subseteq<v_1,...,v_j>\]
		\[\therefore<e_1,..,e_j>\subseteq<v_1,...,v_j>\qedhere\]
	\end{proof}
\end{thm}
\begin{thm}[Representación $\triangle$-superior con respecto a base ortonormal]
	Sea $(V,<\cdot,\cdot>)$ espacio con producto interno (real o complejo) y sea $T\in\mathcal{L}(V)$. Si $T$ posee una representación matricial $\triangle$-superior con respecto a una base, entonces posee una representación $\triangle$-superior con respecto a una base ortonormal.
	\begin{proof}
		Sea $B=(v_1,...,v_n)$ base ordenada de $V$ tal que $\mathcal{M}_B(T)$ sea $\triangle$-superior. Entonces:
		\[<v_1>\textrm{ es invariante bajo} T\]
		\[<v_1,v_2>\textrm{ es invariante bajo} T\]
		\[\vdots\]
		\[<v_1,...,v_n>\textrm{ es invariante bajo} T\]
		Aplicando G-S a $(v_1,...,v_n)\implies e_1,...,e_n$ base ortonormal de $V$.\\
		Para concluir basta probar que los subespacios
		\[<e_1,...,e_j>\quad j=1,...,n\]
		son invariantes bajo $T$.\\
		Pero esto es directo, ya que
		\[<v_1,...,v_j>=<e_1,...,e_j>\quad j=1,...,n\qedhere\]
	\end{proof}
\end{thm}
\begin{thm}[Schur]
	Todo operador sobre un espacio con producto interno complejo de dimensión finita posee una representación $\triangle$-superior para alguna base ortonormal de $V$.
	\begin{proof}
		Todo operador sobre un espacio vectorial complejo de dimensión finita posee una representación $\triangle$-superior para alguna base. Con el Teo anterior se concluye.
	\end{proof}
\end{thm}
\section{Funciones lineales sobre Espacio con producto interno}
\begin{defn}[Funcional Lineal]
	Una funcional lineal sobre $(V,<\cdot,\cdot>)$ es una transformación lineal $\func{l}{V}{\set{F}}{}{}$.
\end{defn}
\underline{Ejemplos:}
\begin{enumerate}
	\item $u\in V:$\\
	$\func{l}{V}{\set{F}}{}{}$\\
	$v\mapsto <v,u>$

	\item Sea $\func{\varphi}{\mathcal{P}_2(\set{R})}{\set{R}}{}{}$ dado por
	\[\varphi(P)=\int^1_{-1}p(t)\cos(\pi t)\d{t}\]
	Pregunta: $\exists q\in\mathcal{P}_2(\set{R}):\varphi(p)=\int^1_{-1}p(t)q(t)\d{t}$?
\end{enumerate}
\begin{thm}[Representación de Riesz]
	Sea $(V,<\cdot,\cdot>)$ espacio producto interno finito-dimensional y $\varphi$ un funcional lineal. Entonces existe un único $u\in V$
	\[\varphi(v)=<v,u>\quad\forall v\in V\]
	\begin{proof}
		Sea $\dim V=n, e_1,..,e_n$ base de ortonormal de V.\\
		\underline{Existencia} Sea $v\in V$
		\[v=<v,e_1>e_1+...+<v,e_n>e_n\quad/\varphi\]
		\[\varphi(v)=\varphi(<v,e_1>e_1+...+<v,e_n>e_n)\]
		\[\varphi(v)=<v,e_1>\varphi(e_1)+...+<v,e_n>\varphi(e_n)>\]
		\[\varphi(v)=<v,\overline{\varphi(e_1)}e_1+...+\overline{\varphi(e_n)}e_n\]
		\underline{Unicidad}: Sean $u_1,u_2\in V$ tales que
		\[\varphi(v)=<v,u_1>=<v,u_2>\quad\forall v\in V\]
		\[\therefore <v,u_1-u_2>=0\quad\forall v\in V\]
		Tomando $v=u_1-u_2$
		\[<u_1-u_2,u_1-u_2>=0\]
		\[\therefore\|u_1-u_2\|^2=0\]
		\[\iff u_1-u_2=0\qedhere\]
	\end{proof}
\end{thm}
\begin{obs}
	La demostración da una fórmula explícita para calcular $u$: Dada una base ortonormal $e_1,...,e_n$
\end{obs}
\section{Complementos Ortogonales y Problemas de Minimización}
\begin{defn}[Complemento Ortogonal]
	Sea $(V,<\cdot,\cdot>$ espacio con producto interno y $U\subset V$. Se define el complemento ortogonal de $U$, denotado $U^\perp$ como
	\[U^\perp=\{v\in V:<v,u>=\forall u\in U\}\]
\end{defn}
\underline{Propiedades:}
\begin{enumerate}[label=\textbf{(\alph*)}]
	\item $U\subseteq V\implies U^\perp$ subespacio vectorial de $V$\\
	$U^\perp=\{v\in V:<v,u>=0\quad\forall u\in U\}$
	\begin{proof}
		\[U^\perp\neq\emptyset:0\in U^\perp\]
		\[v,w\in U^\perp,\lambda\in\set{F}: u\in U\]
		\[\therefore<v+\lambda w,u>=<v,u>+\lambda<w,u>=0\qedhere\]
	\end{proof}

	\item $\{0\}^\perp=V:$ Ejercicio

	\item $V^\perp=\{0\}$
	\begin{proof}
		Si $v\in V^\perp$ entonces:
		\[<v,u>=0\quad\forall u\in V\]
		Tomando $u=v$
		\[<v,v>=0\iff v=0\]
	\end{proof}

	\item $U<V\implies U\cap U^\perp=\{0\}$
	\begin{proof}
		Sea $v\in U\cap U^\perp$
		\[<v,u>=0\quad\forall u\in U\]
		Tomando $u=v$
		\[<v,v>=0\iff v=0\qedhere\]
	\end{proof}

	\item $U\subseteq W\implies W^\perp\subset U^\perp$
	\begin{proof}
		\[W^\perp=\{v\in V:<v,w>=0\quad\forall w\in W\}\]
		\[W^\perp\subseteq\{v\in V:<v,u>=0\quad\forall u\in U\}\]
		\[W^\perp\subseteq U^\perp\qedhere\]
	\end{proof}
\end{enumerate}
\begin{prop}[Suma directa ortogonal]
	Sea $(V,<\cdot,\cdot>)$ espacio con producto interno, y $U$ subespacio finito-dimensional. Entonces:
	\[U\oplus U^\perp=V\]
	\begin{proof}
		\
		$\underline{U+U^\perp=V}$: Sea $e_1,...,e_m$ base ortonormal de $V$
		\[\forall v\in V\]
		\[v=(\overline{<e_1,v>}e_1+...+\overline{<e_m,v>}e_m)\in U+(v-\overline{<e_1,v>}e_1-...-\overline{<e_m,v>}e_m)\]
		Por demostrar que: $u\in U^\perp$\\
		Basta probar que $\forall i=1,...,m$
		\[<e_i,u>=0\]
		Veamos
		\[<e_i,v-<\overline{e_1,v>}e_1-...-\overline{<e_m,v>}e_m>=<e_i,v>-<e_1,v><e_i,e_1>-...-<e_m,v><e_i,e_m>\]
		\[<e_i,u>=<e_i,v>-<e_i,v>=0\]
		\underline{Es directa:} $U\cap U^\perp=\{0\}$ por propiedad.
	\end{proof}
\end{prop}
\begin{cor}[Dimensión del complemento ortogonal]
	Si $V$ es finito-dimensional $U<V$
	\[\dim U+\dim U^\perp=\dim V\]
	\begin{proof}
		\[U\oplus U^\perp=V\qedhere\]
	\end{proof}
\end{cor}
\begin{thm}
	Sea $U<V$ finito-dimensional. Entonces
	\[(U^\perp)^\perp=U\]
	\begin{proof}
		\underline{$U\subseteq(U^\perp)^\perp:$} Sea $u\in U$
		\[<v,u>=0\quad\forall v\in U^\perp\]
		\[\therefore u\in (U^\perp)^\perp\]
		\underline{$U\supseteq(U^\perp)^\perp$} Sea $v\in (U^\perp)^\perp$
		\[v=u+w\quad u\in U,w\in W\]
		\underline{PDQ:} $w=0$\\
		Como $v\in (U^\perp)^\perp$
		\[<v,z>=o\quad\forall z\in U^\perp\]
		Tomando $z=w$
		\[<u+w,w>=0\]
		\[<u,w>+<w,w>=0\]
		\[u\in U, w\in U^\perp\implies <u,w>=0\implies <w,w>=0\iff w=0\]
		\[\therefore v=u\in U\qedhere\]
	\end{proof}
\end{thm}
\begin{defn}[Proyección ortogonal]
	Sea $U$ subespacio vectorial finito-dimensional de $(V,<\cdot,\cdot>)$. Se define la proyección ortogonal $\mathcal{P}_U\in\mathcal{L}(V)$ tal que
	\[\func{\mathcal{P}_U}{V}{V}{}{}\]
	\[v\mapsto u\]
	donde
	\[v=u+w, u\in U, w\in U^\perp\]
\end{defn}
\underline{Ejemplo:} Sea $x\in V,x\neq 0$ y $U=<x>$
\[\mathcal{P}_Uv=\frac{<v,x>}{\|x\|^2}\cdot x\]
\underline{Propiedades:} Sea $U<V$ finito-dimensional y $v\in V$
\begin{enumerate}[label=\textbf{(\alph*)}]
	\item $\mathcal{P}_U\in\mathcal{L}(V)$

	\item $\mathcal{P}_Uu=u\quad\forall u\in U$

	\item $\mathcal{P}_Uw=0\quad\forall w\in U^\perp$

	\item $\Ima\mathcal{P}_U=U$

	\item $\ker\mathcal{P}_U=U^\perp$

	\item $v-\mathcal{P}_Uv\in U^\perp$

	\item $\mathcal{P}_U^2=\mathcal{P}_U$

	\item $\|\mathcal{P}_Uv\}\leq\|v\|$

	\item Para todoa base ortonormal de $U$, $e_1,...,e_n$
	\[\mathcal{P}_Uv=<v,e_1>e_1+,,,+<v,e_n>e_n>\] 
\end{enumerate}
Demostración de las propiedades
\begin{enumerate}[label=\textbf{(\alph*)}]
	\item \begin{proof}
		Sabemos que $\mathcal{P}_U$ es función ya que $U\oplus U^\perp=V$. Probemos la linealidad. Sean $v_1=u_1+w_1,v_2=u_2+w_2$ y sea $\lambda\in\set{F}$
		\[\mathcal{P}_U(v_1+\lambda v_2)=?\]
		Sabemos que
		\[\mathcal{P}_Uv_1=u_1\quad\mathcal{P}_Uv_2=u_2\]
		Y además
		\[v_1+\lambda v_2=(u_1+w_1)+\lambda (u_2+w_2)=(u_1+\lambda u_2)+(w_1+\lambda w_2)\]
	\end{proof}

	\item \begin{proof}
		\[u=u+0\implies P_Uu=u\]
	\end{proof}
	
	\item \begin{proof}
		\[w=0+w\implies P_Uu=0\]
	\end{proof}

	\item Ejercicio

	\item Ejercicio

	\item \begin{proof}
		\[<v,\mathcal{P}_Uv,u>=0\quad\forall u\in U\]
		Sean
		\[v=u+w\quad u\in U\quad w\in U^\perp\]
		\[\implies u=\mathcal{P}_Uv\]
		\[\therefore w-\mathcal{P}_Uv\in U^\perp\qedhere\]
	\end{proof}

	\item \begin{proof}
		$\mathcal{P}_U^2=\mathcal{P}_U$\\
		Sea $v\in V$:
		\[v=u+w\quad u \in U\quad w\in U^\perp\]
		\[\mathcal{P}_Uv=^{(a)}\mathcal{P}_Uu+\mathcal{P}_Uw=^{(b),(c)}u\]
		\[\mathcal{P}_U^2v=\mathcal{P}_Uu+\mathcal{P}_Uw=\mathcal{P}_Uv\qedhere\]
	\end{proof}

	\item Sea $v=u+w\quad u\in U\quad w\in U^\perp$
	\[v=\mathcal{P}_Uv+w\quad/\|\cdot\|^2\]
	\[\|v\|^2=\|\mathcal{P}_Uv\|^2+\|w\|^2\]
	\[\implies \|v\|^2\geq \|\mathcal{P}_Uv\|^2\]

	\item Ejercicio
\end{enumerate}
\section{Problemas de Minimización}
Dado $v\in V$ y $U<V$
\[(\mathcal{P})\min \|u-v\|\quad u\in U\]

\begin{thm}[Distancia mínima a un subespacio vectorial]
	Sea $(V,<\cdot,\cdot>)$ espacio con producto interno y $U<V$. Entonces para todo $v\in V$
	\[\|v-\mathcal{P}_Uv\|\leq \|v-u\|\quad\forall u\in U\]
	Más aun, la igualdad se alcanza $\iff u=\mathcal{P}_Uv$ 
	\begin{proof}
		
	\end{proof}
\end{thm}
\end{document}